\chapter{Famiglia di funzioni hash uniformi}

\begin{problem*}
    Mostrare che la famiglia di funzioni hash 
    \(H = \left\{h(x) = ((ax + b ) \mod p) \mod m\right\}\) \`e (quasi) uniforme, dove 
    \(a,b\in [m]\) con \(a\neq 0\) e \(p\) \`e un numero primo sufficientemente grande.
\end{problem*}
Dobbiamo mostrare che, preso $h\in H$ e dati $k\neq l$ chiavi, $u, v$ valori, vale \[\Pr[h(k)=u \wedge h(l)=v]\approx\Pr[h(k)=u]\cdot\Pr[h(l)=v.]\] Mostreremo che $\Pr[h(k)=u]\approx\Pr[h(l)=v]\approx\frac{1}{m}$ e che $\Pr[h(k)=u \wedge h(l)=v]\approx\frac{1}{m^2}$.
\[\]
Contiamo anzitutto quante sono le chiavi $k$ tali che $h(k)=u$ per un certo valore $u$.\newline
Dato un generico $r$, l'equazione $ax+b=r \pmod p$ ha un'unica soluzione $x$ in $[p]$ in quando $a\neq0$ è sicuramente invertibile. Questo significa che esiste un unico $k$ tale che $ak+b=r \pmod p$.\newline
Se vogliamo inoltre che $r = u \pmod m$, le scelte possibili per $r$ sono almeno $\left\lfloor\frac{p}{m}\right\rfloor$ e al più $\left\lceil \frac{p}{m}\right\rceil$.\newline
Pertanto, dato $u$, se chiamiamo $t$ il numero di possibili valori per $k$ tali che $h(k)=u$, abbiamo $\left\lfloor\frac{p}{m}\right\rfloor\le t \le \left\lceil \frac{p}{m}\right\rceil$, ossia \[\frac{\left\lfloor\frac{p}{m}\right\rfloor}{p}\le\Pr[h(k)=u]\le\frac{\left\lceil \frac{p}{m}\right\rceil}{p},\] da cui \[\Pr[h(k)=u]\approx\frac{1}{m}.\]
\[\]
Consideriamo ora $\Pr[h(k)=u \wedge h(l)=v]$. Se vogliamo che $h(k)=u$ sono possibili per $k$ $t$ valori distinti, con $t$ come sopra, e altrettanti per $l$ affinché sia $h(l)=v$. In totale le coppie $(k, l)$ possibili sono $p(p-1)$, essendo $h\neq k$, da cui \[\frac{\left(\left\lfloor\frac{p}{m}\right\rfloor\right)^2}{p(p-1)}\le \Pr[h(k)=u \wedge h(l)=v]\le\frac{\left(\left\lceil \frac{p}{m}\right\rceil\right)^2}{p(p-1)},\] da cui \[\Pr[h(k)=u \wedge h(l)=v]\approx\frac{1}{m^2},\] come volevasi dimostrare.

