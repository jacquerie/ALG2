\chapter{Permutazione in memoria esterna}

\begin{problem*}
    Dati tre array \(A, B\) e \(C\) di \(N\) elementi, dove \(A\) \`e l'input e 
    \(C\) una permutazione di \(\left\{0,1,\dots ,n-1\right\}\), descrivere e
    analizzare nel modello EMM un algoritmo ottimo per costruire
    \(B[i]=A[C[i]]\) per \(0\le i\le n-1\).
\end{problem*}

Dall'analisi svolta sul Lower Bound della permutazione, sappiamo che il costo di
I/O è dato da \(=(\min\{n, \sort(n)\})\), dove \(\sort(n) =
O\left(\frac{N}{B}log_\frac{N}{M} \frac{N}{B}\right)\).
 
\begin{algorithm}
    \caption{Permutazione in memoria esterna}
    \begin{algorithmic}[1]
        \State Calcola il minimo fra \(\{n, \sort(n)\}\)
        \State Se il minimo è \(n\) si utilizza un algoritmo banale che semplicemente esegue un'iterazione sugli elementi e scrive la permutazione.
        \State Nel caso opposto invece si utilizza tecnica \mapreduce.
    \end{algorithmic}	
\end{algorithm}

Si può notare che l'algoritmo banale è eseguito solo in poche circostanze,
ad esempio quando la dimensione \(B\) del blocco è molto piccola. Di seguito
è mostrato soltanto l'algoritmo con tecnica \mapreduce.
 
\begin{algorithm}
    \caption{Permutazione in memoria esterna con tecnica \mapreduce}
    \begin{algorithmic}[1]
        \State Creare le coppie \(\langle i,C[i] \rangle\), ovvero le coppie che mettono in relazione una posizione del vettore \(C\) con l'indice di permutazione.
        \State Ordina le coppie in base alla seconda componente (l'indice di permutazione).
        \State Sostituisci nelle coppie \(C[i]\) con \(A[C[i]]\).
        \State Ordina le coppie per la prima componente.
        \State Scrivi nel vettore \(B\) le seconde componenti delle coppie.
    \end{algorithmic}	
\end{algorithm}

Analizziamo il costo dei singoli passi dell'algoritmo:
\begin{itemize}
    \item Il \emph{primo} passo dell'algoritmo esegue una lettura di \(N\) elementi
    e una scrittura di \(N\) coppie. Da qui si deduce che il costo di I/O è
    \(O\left(\frac{N}{B}\right)\).
    \item Il costo di I/O del \emph{secondo} passo è uguale a quello di un
    ordinamento che, utilizzando ad esempio il \(k\)-way merge-sort è
    \(O\left(\frac{N}{B}log_\frac{N}{M} \frac{N}{B}\right)\).
    \item Nel \emph{terzo} passo, poiché le coppie sono ordinate per \(C[i]\), si accede
    al vettore \(A\) in modo sequenziale, quindi il costo di I/O è \(O\left(\frac{N}{B}\right)\).
    \item Anche per il \emph{quarto} il costo di I/O è
    \(O\left(\frac{N}{B}log_\frac{N}{M} \frac{N}{B}\right)\).
    \item Per l'\emph{ultimo} passo il costo di I/O è \(O\left(\frac{N}{B}\right)\) in quanto
    sono letti e scritti \(N\) sequenzialmente.
\end{itemize}

In conclusione il costo di \(I/O\) dell'algoritmo appena esposto è
\(O\left(\frac{N}{B}log_\frac{N}{M} \frac{N}{B}\right)\).
