\chapter{Multi-selezione in memoria esterna}    

\begin{problem*}
    Scrivere tutti i passaggi dell'analisi del costo e della correttezza
    dell'algoritmo di multi-selezione visto a lezione.
\end{problem*}

Vogliamo esibire un algoritmo che selezioni un certo numero di pivot da un
insieme \(S\) di cardinalit\`a \(N\) in modo tale che la distanza fra pivot
consecutivi sia piccola. Il nostro scopo sar\`a usare questo algoritmo per
costruire un analogo del \quicksort in memoria esterna, cos\`i come la \(k\)
-way merge ci ha permesso di costruire l'algoritmo di \mergesort in 
memoria esterna.

Ci potremmo aspettare di dover trovare \(m\) pivot, in analogia a quanto
facciamo per la Merge. In realt\`a \`e sufficiente determinarne \(\sqrt{m}\).
Diamo di seguito l'algoritmo e due lemmi. Nel primo dimostreremo il costo
lineare, nel secondo la correttezza dell'algoritmo.

\begin{algorithm}
    \caption{Multi-selezione in memoria esterna}
    \begin{algorithmic}[1]
        \State Carico e ordino in memoria principale \(\frac{N}{M}\) run
        di \(M\) elementi ciascuno.
        \State Da ogni run seleziono un elemento ogni 
        \(\frac{\sqrt{m}}{4}\) e chiamo \(G\) (elementi verdi) l'insieme 
        degli elementi selezionati.
        \State Uso l'algoritmo dei cinque autori \(\sqrt{m}\) volte per
        selezionare in \(G\) un elemento ogni \(\frac{4N}{m}\) e chiamo 
        \(R\) (elementi rossi) l'insieme degli elementi selezionati.
        \State Ritorno \(R\).
    \end{algorithmic}
\end{algorithm}

\begin{lemma}[Costo]
    L'algoritmo compie \(O(n)\) I/O.
\end{lemma}
\begin{proof*}
    La prima riga dell'algoritmo comporta soltanto di scandire tutti gli
    elementi: l'ordinamento di ogni run viene infatti svolto in memoria
    principale, e non comporta ulteriori I/O. Anche la seconda riga
    consiste in una scansione di tutti gli elementi. Per stimare il numero
    di I/O della terza riga sfruttiamo invece il fatto che ogni esecuzione
    dell'algoritmo dei cinque autori comporta una scansione di tutti gli
    elementi. Abbiamo dunque \(\sqrt{m}\) scansioni di \(|G|\) elementi,
    perci\`o:
    \[
        \sqrt{m}\cdot O\left(\frac{|G|}{B}\right) = \sqrt{m}\cdot O\left(\frac{4N}{B\sqrt{m}}\right) = O\left(\frac{4N}{B}\right) = O(n)\mbox{,}
    \]
    dove la prima eguaglianza discende dal fatto che, avendo selezionato
    un elemento ogni \(\frac{\sqrt{m}}{4}\), la cardinalit\`a di \(G\) \`e
    \(\frac{4N}{\sqrt{m}}\). Ogni riga contribuisce quindi \(O(n)\) I/O, da cui la tesi.
\end{proof*}

\begin{lemma}[Correttezza]
    Il numero di elementi di \(S\) compresi fra due elementi di \(R\) \`e
    minore di \(\frac{3}{2}\frac{N}{\sqrt{m}}\).
\end{lemma}
\begin{proof*}
    Vogliamo dunque stimare il numero di elementi di \(S\) compresi fra
    due generici elementi rossi \(r_1\) e \(r_2\). Possiamo dividerli in
    tre categorie:
    \begin{itemize}
        \item Gli elementi \emph{verdi} compresi fra i due elementi rossi \(r_1\) e \(r_2\).
        \item Gli elementi \emph{senza colore} compresi fra due elementi verdi.
        \item Gli elementi \emph{senza colore} compresi fra un rosso e un verde.
    \end{itemize}
        
    I primi sono facilmente maggiorati da \(X = \frac{4N}{m}\): nella 
    terza riga dell'algoritmo abbiamo infatti scelto un rosso ogni 
    \(\frac{4N}{m}\) elementi verdi.
        
    I secondi sono invece maggiorati da \(Y = \frac{N}{\sqrt{m}} - \frac{4N}{m}\). Per la seconda riga dell'algoritmo abbiamo infatti 
    \(\frac{\sqrt{m}}{4}-1\) elementi senza colore fra due verdi
    consecutivi appartenenti allo stesso run, avendo scelto un verde ogni
    \(\frac{\sqrt{m}}{4}\) elementi. Per la terza riga abbiamo al pi\`u
    \(\frac{4N}{m}\) elementi verdi fra \(r_1\) e \(r_2\), dunque al
    pi\`u \(\frac{4N}{m}\) coppie di elementi verdi consecutivi nello
    stesso run. Ma allora il numero cercato \`e stimato dal prodotto, e 
    quindi da:
    \[
        \frac{4N}{m}\left(\frac{\sqrt{m}}{4}-1\right) = \frac{N}{\sqrt{m}} - \frac{4N}{m}\mbox{.}
    \]
        
    I terzi sono invece maggiorati da \(Z = \frac{n}{2\sqrt{m}} - \frac{2n}{m}\). Per vedere questo abbiamo bisogno della figura.
    \begin{figure}
        \centering
        \begin{tikzpicture}
            \node (l_boundary_down) at (3, -0.5) [] {};
            \node (l_boundary_up) at (3, 4.5) [] {};
            \draw (l_boundary_down) -- (l_boundary_up) [thick];
                
            \node (r_boundary_down) at (8, -0.5) [] {};
            \node (r_boundary_up) at (8, 4.5) [] {};
            \draw (r_boundary_down) -- (r_boundary_up) [thick];
                
            \fill [fill=yellow!50] (3, -0.5) -- (3, 4.5) -- (4.25, 4.5) -- (4.25, -0.5) -- cycle;             
            \fill [fill=yellow!50] (8, -0.5) -- (8, 4.5) -- (6.75, 4.5) -- (6.75, -0.5) -- cycle;
                
            \node (run_1_begin) at (0,0) [] {};
            \node (run_1_end) at (10,0) [] {};
                
            \draw (run_1_begin) -- (run_1_end) [];
            \node (el_1_1) at (1,0) [green_el] {};
            \node (el_1_2) at (1.5,0) [el] {};
            \node (el_1_3) at (2.5,0) [el] {};
            \node (el_1_4) at (4,0) [green_el] {};
            \node (el_1_5) at (5.5,0) [el] {};
            \node (el_1_6) at (7.5,0) [el] {};
            \node (el_1_7) at (8.5,0) [green_el] {};
            \node (el_1_8) at (9,0) [el] {};
            \node (el_1_9) at (9.5,0) [el] {};
                
            \node (run_2_begin) at (0,1) [] {};
            \node (run_2_end) at (10,1) [] {};
                
            \draw (run_2_begin) -- (run_2_end) [];
            \node (el_2_1) at (0.5,1) [green_el] {};
            \node (el_2_2) at (1,1) [el] {};
            \node (el_2_3) at (2.5,1) [el] {};
            \node (r_1) at (3,1) [red_el] {};
            \node (el_2_5) at (3.5,1) [el] {};
            \node (el_2_6) at (4,1) [el] {};
            \node (el_2_7) at (7.5,1) [green_el] {};
            \node (el_2_8) at (8.5,1) [el] {};
            \node (el_2_9) at (9,1) [el] {};
                
            \node (run_3_begin) at (0,2) [] {};
            \node (run_3_end) at (10,2) [] {};
                
            \draw (run_3_begin) -- (run_3_end) [];
            \node (el_3_1) at (0.5,2) [green_el] {};
            \node (el_3_2) at (1.5,2) [el] {};
            \node (el_3_3) at (2,2) [el] {};
            \node (el_3_4) at (3.5,2) [green_el] {};
            \node (el_3_5) at (5,2) [el] {};
            \node (el_3_6) at (5.5,2) [el] {};
            \node (el_3_7) at (7.5,2) [green_el] {};
            \node (el_3_8) at (9,2) [el] {};
            \node (el_3_9) at (9.5,2) [el] {};
                
            \node (run_4_begin) at (0,3) [] {};
            \node (run_4_end) at (10,3) [] {};
                
            \draw (run_4_begin) -- (run_4_end) [];
            \node (el_4_1) at (1,3) [green_el] {};
            \node (el_4_2) at (1.5,3) [el] {};
            \node (el_4_3) at (2.5,3) [el] {};
            \node (el_4_4) at (4.5,3) [green_el] {};
            \node (el_4_5) at (7,3) [el] {};
            \node (el_4_6) at (7.5,3) [el] {};
            \node (r_2) at (8,3) [red_el] {};
            \node (el_4_8) at (8.5,3) [el] {};
            \node (el_4_9) at (9.5,3) [el] {};
                
            \node (run_5_begin) at (0,4) [] {};
            \node (run_5_end) at (10,4) [] {};
                
            \draw (run_5_begin) -- (run_5_end) [];
            \node (el_5_1) at (1.5,4) [green_el] {};
            \node (el_5_2) at (2,4) [el] {};
            \node (el_5_3) at (2.5,4) [el] {};
            \node (el_5_4) at (4,4) [green_el] {};
            \node (el_5_5) at (4.5,4) [el] {};
            \node (el_5_6) at (6.5,4) [el] {};
            \node (el_5_7) at (7,4) [green_el] {};
            \node (el_5_8) at (7.5,4) [el] {};
            \node (el_5_9) at (9,4) [el] {};
        \end{tikzpicture}
        \caption{Ogni riga orizzontale rappresenta un run ordinato, e i 
        cerchietti gli elementi di ogni run. Cerchietti rossi e verdi 
        rappresentano rispettivamente gli elementi di \(R\) e \(G\), 
        cerchietti neri i restanti elementi senza colore. Sono inoltre 
        raffigurati in giallo i bordi definiti dalla posizione degli 
        elementi rossi nei quali possiamo avere elementi senza colore 
        compresi fra un rosso e un verde.}
    \end{figure}
    Osserviamo infatti che la posizione dei due elementi rossi definisce
    un paio di ``bordi'' (raffigurati in giallo in figura) in cui \`e
    possibile trovare gli elementi del terzo tipo. Ognuno di questi 
    bordi può contenere al pi\`u \(\frac{\sqrt{m}}{4}-1\) elementi,
    perché se ne contenesse di pi\`u conterrebbe sicuramente due verdi,
    quindi elementi compresi fra due verdi. Per ognuno degli 
    \(\frac{N}{M}\) run abbiamo insomma \(2\) bordi contenenti al pi\`u 
    \(\frac{\sqrt{m}}{4}-1\) elementi, perci\`o possiamo stimare il
    numero di elementi del terzo tipo con il loro prodotto, cio\`e:
    \[
        2 \cdot \frac{n}{m} \cdot \left( \frac{\sqrt{m}}{4}-1 \right) = \frac{n}{2\sqrt{m}} - \frac{2n}{m}\mbox{,}
    \]
    dove abbiamo usato che \(\frac{N}{M}\) = \(\frac{n}{m}\) essendo
    \(n = \frac{N}{B}\) e \(m = \frac{M}{B}\).
        
    Di conseguenza il numero totale degli elementi compresi fra \(r_1\) 
    e \(r_2\) è maggiorato da:
    \begin{align}
        X + Y + Z &= \frac{4N}{m} + \frac{N}{\sqrt{m}} - \frac{4N}{m} + \frac{n}{2\sqrt{m}} - \frac{2n}{m} \nonumber \\
        &\le \frac{N}{\sqrt{m}} + \frac{n}{2\sqrt{m}} \nonumber
    \end{align}
    nella quale abbiamo cancellato i termini uguali di segno opposto e un
    termine negativo, ottenendo un'ulteriore maggiorazione. Abbiamo
    inoltre:
    \begin{align}
        \frac{N}{\sqrt{m}} + \frac{n}{2\sqrt{m}} &\le \frac{N}{\sqrt{m}} + \frac{N}{2\sqrt{m}} \nonumber \\
        &= \frac{3}{2}\frac{N}{\sqrt{m}} \nonumber
    \end{align}
    essendo \(n = \frac{N}{B}\), da cui la tesi.
\end{proof*}