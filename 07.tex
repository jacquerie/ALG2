\chapter{Layout di alberi binari}

\begin{problem*}
  Riportare tutti i passaggi dell'analisi della paginazione descritta nella
  tesi di David Clark (reperibile nella pagina del corso) per un generico
  albero binario in blocchi di dimensioni \(B\). Opzionale: per chi vuole,
  esiste una versione pi\`u impegnativa di questo esercizio dove occorre
  analizzare una paginazione in cui un cammino radice-nodo di lunghezza \(\ell
  \) attraversa \(O(\ell\,/\log{B})\) pagine; contattarmi per discutere questa
  opzione.
\end{problem*}

Sia \(T\) un albero binario di altezza \(H\) su \(n\) nodi. Vogliamo 
partizionare i vertici in sottoinsiemi (detti \emph{pagine}) in modo che:
\begin{enumerate}
  \item Ogni pagina sia una componente connessa dell'albero contenente al 
  pi\`u \(B\) nodi interni.
  \item Sia minimo il numero massimo di pagine attraversate in un cammino
  dalla radice ad una foglia.
\end{enumerate}
