\chapter{Elementi distinti}

\begin{problem*}
    Progettare e analizzare un algoritmo di data streaming che permetta di
    approssimare il numero di elementi distinti.
\end{problem*}

Cominciamo fissando le notazioni usate nel seguito. Sia \(\chi = x_1 x_2
\dots x_n\) uno stream di \(n\) caratteri scelti da un insieme di cardinalit\`a
\(m\le n\) di simboli distinti. Denotiamo con \(D(\chi)\) il numero di elementi
distinti effettivamente presenti nello stream.

Per prima cosa risolviamo un problema pi\`u semplice: dato un numero \(t\)
vogliamo decidere con alta probabili\`a se \(D(\chi) \gg t\) oppure \(D(\chi) \ll
t\). Pi\`u precisamente vogliamo risolvere il seguente
\begin{problem*}
  Data una soglia \(t\) e uno stream \(\chi\), rispondere
  \begin{itemize}
    \item S\`i se \(D(\chi)\ge t\),
    \item No se \(D(\chi)<\frac{t}{2}\),
    \item indifferentemente S\`i o No altrimenti,
  \end{itemize}
  con probabilit\`a maggiore di \(\delta\) di essere corretto.
\end{problem*}

Supponiamo di avere una funzione hash \(h:X\rightarrow [1,t]\) ideale, tale
cio\`e che \(\forall i\mbox{, }Pr[h(x) = i] = \frac{1}{t}\). In termini di
questa funzione siamo quasi in grado di risolvere il precedente problema.
Abbiamo infatti il seguente
\begin{algorithm}
  \caption{Contatore con rumore}
  \begin{algorithmic}[1]
    \For {each \(x_i\in X\) do}
      \If {\(h(x_i) = t\)}
        \State \Return {S\`i}
      \EndIf
    \EndFor
    \State \Return {No}
  \end{algorithmic}
\end{algorithm}

\begin{lemma}
  Supponiamo che \(D(\chi)\ge t\) o \(D(\chi)<\frac{t}{2}\). Allora il precedente
  algoritmo \`e corretto con probabilit\`a maggiore di \(0.6\).
\end{lemma}
\begin{proof}
  \begin{itemize}
    \item Supponiamo \(D(\chi)\ge t\). TODO
    \item Supponiamo invece \(D(\chi)<\frac{t}{2}\). TODO
  \end{itemize}
\end{proof}

