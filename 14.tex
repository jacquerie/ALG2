\chapter {Cuckoo hashing}

\begin{problem*}
    Scrivere tutti i passaggi dell'analisi del costo dell'inserimento
    di un elemento in una tabella di cuckoo hashing. Discutere anche
    della cancellazione e della sua complessità.
\end{problem*}

Dobbiamo fare hashing da un insieme di $n$ elementi ad un insieme di $r$
elementi. Invece di procedere nel modo classico usando le liste di adiacenza
(che nel caso pessimo potrebbero contenere tutti gli elementi), vogliamo un
modo che ci permetta di avere esattamente un elemento in associato ad ognuno
degli $r$ valori (quindi che sia possibile salvare gli $n$ elementi su un array
di dimensione $r$).
Per farlo, ci avvaliamo di due funzioni hash, 2-wise indipendenti, $h_1(x)$ e
$h_2(x)$, in questo modo:

\begin{algorithm}
    \caption{Inserimento in Cuckoo hashing}
    \begin{algorithmic}[1]
        \State provo ad inserire x in $h_1(x)$
        \State se la cella è libera lo inserisco semplicemente
        \State se la cella non è libera tolgo l'elemento y dalla cella $h_1(x)$
        e ripeto il procedimento per y, provando a inserirlo in $h_1(y)$ se
        $h_1(x) = h_2(y)$ o in $h_2(y)$ se $h_1(x) = h_1(y)$, per un massimo di
        $n$ volte.
        \State se il valore non è stato ancora inserito si cambiano le funzioni
        di hashing, e si riprova ad inserire il valore che si stava cercando di
        inserire alla $n$-esima iterazione.
    \end{algorithmic}
\end{algorithm}

Si nota, come vedremo, che il numero di iterazioni massimo nel punto 3 serve
per evitare di andare in loop quando si tenta di inserire un valore.

Per fare l'analisi è necessario introdurre i concetti di grafo Cuckoo e di
bucket, più un lemma:

\begin{definition*}[Grafo Cuckoo]
    Il grafo Cuckoo è un grafo che ha per nodi le celle dell'array, con un arco
    uscente in ogni cella contenente un valore, che punta alla cella alternativa
    secondo le funzioni $h_1(x)$, $h_2(x)$. Ovvero, se il nodo $i$ contiene il
    valore $x$, $i$ avrà un arco uscente verso $h_2(x)$ se $i = h_1(x)$ o
    viceversa un arco verso $h_1(x)$ se $i = h_2(x)$.
\end{definition*}

\begin{definition*}[Bucket]
    Si dice \emph{bucket} di un valore $x$ l'insieme dei nodi raggiungibili dai
    nodi $\left\{h_1(x), h_2(x)\right\}$ nel grafo Cuckoo. Ossia tutti i nodi
    con cui potremmo avere a che fare nel caso volessimo inserire $x$.
\end{definition*}

\begin{lemma}
    Per ogni nodo $i$ e $j$, e ogni $c > 1$, se $r \ge 2cn$, la probabilità che
    esista un cammino tra $i$ e $j$ di lunghezza $l$ è al più $\frac{c^{-l}}{r}
    = \frac{1}{c^lr}$. Ovvero, se il numero di celle nell'array è
    sufficientemente più grande del numero di valori salvati, la probabilità
    che esista un cammino di lunghezza $l$ tra due nodi è $O(\frac{1}{r})$, e
    decresce esponenzialmente.
\end{lemma}

\begin{proof*}
    Procediamo per induzione sulla lunghezza del percorso:
    \begin{itemize}
    \item per $l = 1$: un percorso di lunghezza 1 tra due nodi $i$ e $j$ esiste
    sse per qualche $x$ $h_1(x) = i \land h_2(x) = j$ oppure $h_1(x) = j \land
    h_2(x) = i$, si ha:
    
    \begin{align}
        &\Pr\left[(h_1(x) = i \land h_2(x) = j) \lor (h_1(x) = i \land h_2(x) = j) \right] = \nonumber \\
        &= \Pr\left[h_1(x) = i \land h_2(x) = j\right] + \Pr\left[h_1(x) = j \land h_2(x) = i\right] = \nonumber \\
        &= 2 \Pr\left[h_1(x) = i\right]\Pr\left[h_2(x) = j\right] = \nonumber \\
        &= 2 \frac{1}{r} \frac{1}{r} = \frac{2}{r^2} \nonumber
    \end{align}

    Siccome il numero di elementi per cui vale la proprietà vista sopra è al
    più n, si ha:

    \begin{align}
        & \Pr\left[\exists \mbox{percorso di lunghezza 1 tra $i$ e $j$}\right] \nonumber \\
        &\le n \frac{2}{r^2} = \frac{2n}{r} \frac{1}{r} \nonumber \\
        &\left\{r \ge 2cn \mbox{ per ipotesi} \Rightarrow c \le \frac{r}{2n}
            \Rightarrow \frac{1}{c} \ge \frac{2n}{r} \right\} \nonumber \\
        &\le \frac {1}{cr} = \frac{c^{-1}}{r} \nonumber
    \end{align}

    \item per $l > 1$ è necessario che:
        \begin{enumerate}
            \item Esista un percorso ottimo lungo $l-1$ da $i$ a $k$.
            \item Esista un arco tra $k$ e $j$.
        \end{enumerate}
        Abbiamo
            \[\Pr[(1)] = \frac{c^{1-l}}{r}\]
        per ipotesi induttiva. Inoltre, usando lo stesso ragionamento di prima
        otteniamo che
            \[\Pr[(2)]= \frac{c^{-1}}{r}.\]

        Notiamo che i valori possibili di k sono $r$ e che quindi la probabilità
        totale è data da:
        \[r \Pr[(1)]\Pr[(2)] = r\frac{c^{-l}}{r^2} = \frac{c^{-l}}{r}\] \qed
    \end{itemize}
\end{proof*}

La probabilità che al punto 3 dell'inserimento avvenga un reashing, è maggiorata
dalla probabilità che per qualche elemento esista un ciclo. Se notiamo che
\emph{esiste un ciclo di lunghezza $l$}$\Leftrightarrow$\emph{esiste un percorso di
lunghezza $l$ tra $i$ e $i$ } otteniamo:
\begin{align*}
    &\Pr\left[\exists \mbox{un ciclo per il nodo i nel grafo Cuckoo}\right] \\
    &= \sum_{l=1}^{\infty} \Pr\left[\exists \mbox{ciclo di lunghezza $l$ nel grafo Cuckoo}\right] \\
    &= \sum_{l=1}^{\infty} \Pr\left[\exists \mbox{percorso di lunghezza $l$ tra $i$ e $i$}\right] \\
    &\le \sum_{l=1}^{\infty} \frac{c^{-l}}{r} = \frac{1}{r(c-1)}
\end{align*}

da cui:
\begin{align*}
    &\Pr\left[\exists \mbox{un ciclo nel grafo cuckoo}\right] \\ 
    &= \sum_{i=1}^{r} \Pr\left[\exists \mbox{un ciclo per il nodo i nel grafo cuckoo}\right] \\
    &= r*\frac{1}{r(c-1)} = \frac{1}{c-1}
\end{align*}

notiamo inoltre che se abbiamo esattamente n cicli, possiamo avere al più
n rehash, quindi:

\begin{align*}
    &\Pr\left[\exists \mbox{un ciclo nel grafo cuckoo}\right]^n
        \Pr\left[\not\exists \mbox{nessun ciclo nel grafo cuckoo}\right] \\
    &= \frac{1}{(c-1)^n} \frac{c-2}{c-1} = \frac{c-2}{(c-1)^{n+1}}\\
    &\ge \Pr\left[\mbox{n rehash}\right]
\end{align*}

Ponendo $c=3$ la probabilità di un rehash è al più $\frac{1}{4}$, e di n
rehash è al più $\frac{1}{2^{n+1}}$, quindi il numero atteso di rehash ad ogni
inserimento è

\[ \sum_{i=1}^{n} i \frac{1}{2^{i+1}} = 1. \]

Il costo medio di un inserimento quindi è dato dal costo di un rehashing, 
ognuno da $\Theta(n)$, quindi a sua volta $\Theta(n)$, mentre il costo
ammortizzato per ogni inserimento è $O(1)$.

La cancellazione avviene in $O(1)$, cercando il valore da cancellare nelle sole
due celle possibili ed eliminandolo, è possibile osservare infatti che la
struttura che si ottiene è ancora un cuckoo hashing dove la funzione $h_1(x)$
è quella che mette tutti gli elementi esattamente dove sono e la funzione
$h_2(x)$ è una qualunque (o varianti equivalenti).