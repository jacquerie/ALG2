\chapter{Ordinamento in memoria esterna}

\begin{problem*}
    Mostrare come implementare il \kwaymerge in EMM, ossia la fusione di
    \(n\) sequenze individualmente ordinate e di lunghezza totale \(N\), con
    costo I/O di \(O\left(\frac{N}{B}\right)\), dove \(B\) \`e la dimensione 
    del blocco. Minimizzare e valutare il costo di CPU. Analizzare il costo 
    del \mergesort (I/O complexity, CPU complexity) che utilizza tale 
    \kwaymerge.
\end{problem*}

Date \(k\) sequenze individualmente ordinate, dette \emph{run}, il \kwaymerge 
mantiene in memoria principale \(k\) buffer di input e un buffer di 
output, tutti di dimensione \(B\). A ogni passo seleziona fra i buffer di 
input attivi l'elemento pi\`u piccolo e lo sposta nel buffer di output; 
quando questo risulta pieno il suo contenuto viene copiato in memoria 
esterna. Ogni volta che un buffer di input \`e arrivato al termine 
l'algoritmo copia un nuovo blocco dal corrispondente run; se questo non \`e 
pi\`u possibile il buffer viene marcato come inattivo e non viene pi\`u 
considerato nella selezione dell'elemento pi\`u piccolo. Quando infine rimane 
un solo run attivo viene copiato il suo contenuto in memoria esterna.

\begin{algorithm}
  \caption{\kwaymerge in EMM}
  \begin{algorithmic}[1]
    \State Inizializza \(k\) buffer di input contenenti ciascuno il primo 
    blocco del corrispondente run e marca ogni buffer come attivo
    \While{ci sono almeno due buffer di input attivi}
      \State trova il pi\`u piccolo degli elementi ancora inutilizzati nei 
      buffer attivi
      \State sposta tale elemento nella prima posizione libera del buffer di 
      output
      \If{il buffer di output risulta pieno}
        \State svuotalo in memoria esterna e reinizializzalo
      \EndIf
      \If{il buffer di input da cui ho prelevato l'elemento è giunto alla 
      fine}
        \State copia in tale buffer il prossimo blocco del run corrispondente
        \If {il run \`e terminato}
          \State marca quel buffer come inattivo
        \EndIf
      \EndIf
    \EndWhile
    \State copia l'ultimo run attivo in memoria esterna.
  \end{algorithmic}	
\end{algorithm}

Osserviamo facilmente che il costo in termini di I/O del precedente algoritmo
è \(O\left(\frac{N}{B}\right)\), infatti ciascun elemento \`e coinvolto in 
\(O\left(1\right)\) trasferimenti fra memoria principale e secondaria: uno 
quando entra in un buffer di input e uno quando esce dal buffer di output.

\`E pi\`u delicato valutare il costo in CPU: se infatti al passo \(3\)
troviamo ripetutamente il minimo con una scansione lineare dei primi elementi
di ogni buffer, paghiamo a ogni passo \(O(k)\) confronti, per un costo totale
di \(O(Nk)\). Mantenendo invece un \minheap dei primi \(k\) elementi 
paghiamo a ogni passo costo logaritmico, per un costo totale in CPU di 
\(O(N\log_{2}{k})\).

Osserviamo inoltre che abbiamo tacitamente supposto che la memoria principale 
sia abbastanza grande da contenere \(k+1\) blocchi, ossia \(k \le \frac{M}{B} 
- 1\). Ci\`o \`e rilevante nell'analisi del \mergesort: infatti il caso base 
di questo algoritmo consiste in \(\left\lceil\frac{N}{M}\right\rceil\)
run ordinati grandi quanto la memoria principale. Se questi sono meno di 
\(\frac{M}{B} - 1\) allora basta eseguire una sola volta il \kwaymerge
per ottenere l'ordinamento in memoria esterna.

Se invece cos\`i non fosse dobbiamo eseguire pi\`u volte il \kwaymerge.
A ogni esecuzione il numero di run si riduce di un fattore \(k\), dunque
viene eseguito \(\log_{\frac{M}{B}}{\frac{N}{M}}\) volte. Di conseguenza il 
\mergesort che fa uso del \kwaymerge ha un costo di
\[O\left(\frac{N}{B}\log_{\frac{M}{B}}{\frac{N}{M}}\right) = 
O\left(\frac{N}{B}\log_{\frac{M}{B}}{\left(\frac{N}{M}\cdot\frac{M}{B}\right)}
\right) = O\left(n\log_{m}{n}\right)\mbox{ I/O,}\] che sappiamo essere ottimo.

TODO
