\chapter{Applicazioni di LZ77}

\begin{problem*}
  Sfruttando le caratteristiche dell'algoritmo LZ77 di Lempel e Ziv per
  suddividere un testo in una sequenza di frasi:
  \begin{enumerate}[(a)]
    \item mostrare come utilizzare LZ77 per nascondere dei bit all'interno 
    del file compresso risultante senza aumentarne la dimensione (nel caso
    in cui questo sia possibile)
    \item utilizzare il suffix tree (per costruire LZ77) per trovare la pi\`u 
    lunga sottostringa che si ripete, ossia che appare almeno due volte nel 
    testo.
  \end{enumerate}
\end{problem*}

\begin{enumerate}[(a)]
  \item TODO
  \item Chiamiamo \emph{suffix tree} di una stringa la trie compatta dei suoi
  suffissi. Ad esempio il suffix tree della stringa 
  \texttt{\string"abracadabra\$\string"} \`e:
  \begin{figure}[H]
    \centering
    \begin{tikzpicture}
      \node (_) at (-3,0) [] {};
      \node (a_) at (1,0) [] {};
      \node (abra_) at (2,0) [] {};
      \node (abracadabra_) at (3,0) [] {};
      \node (acadabra_) at (4,0) [] {};
      \node (adabra_) at (5,0) [] {};
      \node (bra_) at (6,0) [] {};
      \node (bracadabra_) at (7,0) [] {};
      \node (cadabra_) at (8,0) [] {};
      \node (dabra_) at (9,0) [] {};
      \node (ra_) at (10,0) [] {};
      \node (racadabra_) at (11,0) [] {};
      
      \node (abra) at (2.5, 1) [] {};
      
      \node (a) at (2.5, 3) [] {};
      \node (bra) at (6.5, 3) [] {};
      \node (ra) at (10.5, 3) [] {};
      
      \node (root) at (7.5, 7) [] {};
      
      \draw (abra_) -- (abra) [];
      \draw (abracadabra_) -- (abra) [];
      
      \draw (a) -- (a_) [];
      \draw (a) -- (abra) [];
      \draw (a) -- (acadabra_) [];
      \draw (a) -- (adabra_) [];
      
      \draw (bra) -- (bra_) [];
      \draw (bra) -- (bracadabra_) [];
      
      \draw (ra) -- (ra_) [];
      \draw (ra) -- (racadabra_) [];
      
      \draw (root) -- (_) [];
      \draw (root) -- (a) [];
      \draw (root) -- (bra) [];
      \draw (root) -- (cadabra_) [];
      \draw (root) -- (dabra_) [];
      \draw (root) -- (ra);
    \end{tikzpicture}
  \end{figure}
  
  Chiamiamo \emph{profondit\`a} di un nodo la distanza di tale nodo dalla 
  radice, misurata per\`o nel numero di caratteri totali delle etichette
  del cammino dalla radice al nodo. Per esempio, nel precedente albero, la 
  profondit\`a del figlio pi\`u a destra della radice \`e \(2\).
  
  Affermiamo allora che la lunghezza della massima sottostringa ripetuta \`e
  la profondit\`a massima di un nodo interno. Infatti perché un nodo sia
  interno \`e necessario che esistano almeno due sottostringhe i cui suffissi
  coincidano per qualche carattere, mentre la condizione di massimalit\`a
  garantisce che si tratti della massima sottostringa ripetuta.
  
  Dato il suffix tree possiamo visitare in qualche ordine i suoi vertici ed
  annotare per ciascuno di essi la profondit\`a. Nella stessa visita possiamo 
  mantenere il nodo interno che realizzi la profondit\`a massima per
  restituirlo a fine visita.
\end{enumerate}
