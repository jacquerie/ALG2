\chapter{Random search tree}

\begin{problem*}
  Scrivere l'algoritmo per inserire una chiave in un random search tree
  con una sola discesa dalla radice (i.e., senza dover risalire poi dalla
  foglia appena inserita verso la radice mediante le rotazioni).
\end{problem*}

L'algoritmo per l'inserimento è il seguente:

\begin{algorithm}
    \caption{Inserimento in un albero random di ricerca binario}
    \begin{algorithmic}[1]
        \Function {insert}{$x, T$}
            \State $n \gets T.size$
            \State $r \gets random(0, n)$

            \If {$r == n$}
                \State $insert\_root(x, T)$
            \EndIf
            \If {$x < T.key$}
                \State $insert(x, T.left)$
            \Else
                \State $insert(x, T.right)$
            \EndIf
        \EndFunction
    \end{algorithmic}
\end{algorithm}

a questo punto si pone il prolbema di inserire l'elemento nel sottoalbero
che lo dovrà contenere in modo da mantenere l'integrità della struttura,
per fare questo usiamo il seguente algoritmo, che separa gli elementi più
piccoli di $x$ e quelli più grandi di $x$, e al posto del sottoalbero di
partenza mette un sottoalbero con radice $x$, figlio sinistro il sottoalbero
contenente tutti valori più piccoli di $x$ e figlio destro contenente tutti
quelli più grandi:
begin
\begin{algorithm}
    \caption{Inserimento nella radice di un sottoalbero}
    \begin{algorithmic}[1]
        \Function {insert\_root}{$x, T$}
            \State $rbintree$ $S, G$
            \State $split(x, T, \&S, \&G)$
            \State $T \gets new\_node()$
            \State $T.left = S, T.right = G$
            \State \Return{$T$}
        \EndFunction

        \Function{split}{$x, T, L, R$}
            \If {$empty(T)$}
                \State $*L \gets empty\_tree(), *R \gets empty\_tree()$
                \State \Return {}
            \EndIf
            \If {$x < T.key$}
                \State $*R \gets T$
                \State $split(x, T.left, L, \&(*R.left))$
            \Else
                \State $*L \gets T$
                \State $split(x, T.right, \&(*L.right), R)$
            \EndIf
            \State \Return {}
        \EndFunction
    \end{algorithmic}
\end{algorithm}