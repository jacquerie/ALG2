\chapter{Count-min sketch: interval query}

\begin{problem*}
    Mostrare come utilizzare il paradigma del count-min sketch per rispondere alle
    interval query (i.e., approssimare \(\sum_{k=i}^j{F[k]}\)).
\end{problem*}

Per risolvere questo problema suddividiamo ogni range in modo canonico in
sottorange di dimensione $2^k$, e usiamo $\log n$ count min sketch per
memorizzare separatamente i sottorange di dimensione diversa.

La suddivisione avviene in intervalli di tipo $[x2^k+1,(x+1)2^k]$ (\emph{diadici})
per $k \in [0, \log n-1]$, e ogni volta che ci arriva un dato andiamo ad
incrementare tutti i count min sketch negli intervalli corrispondenti.

\begin{lemma}[Suddivisione canonica]
    Qualunque intervallo può essere suddiviso in al più $2 \log{n}$ intervalli
    diadici.
\end{lemma}

\begin{proof*}
    Sia $[l^*, r^*]$ l'intervallo diadico più grande che può stare dentro il
    nostro intervallo $[l, r]$. Evidentemente vale $|[l^*, r^*]| < n$ e
    quindi l'intervallo diadico è al più della forma
    $[x2^{\log n-1}+1,(x+1)2^{\log n-1}]$, cioè $k < \log{n}$.

    Consideriamo i due intervalli indotti $[l, x2^{k}+1]$ e $[(x+1)2^{k}, r]$,
    ci sono due casi possibili:
    \begin{itemize}
        \item uno dei due ha dimensione $\ge 2^k$, questo non è un assurdo
        perché i due intervalli di dimensione $2^k$ potrebbero essere spostati
        di $2^k$ rispetto all'intervallo diadico canonico più vicino di
        dimensione $2^{k+1}$.

        In questo caso ci limitiamo a togliere l'intervallo di dimensione $2^k$
        o dal fondo del primo intervallo indotto o dall'inizio del secondo
        (è semplice immaginare perché si deve trovare proprio lì), e ci
        riconduciamo al secondo caso.

        \item entrambi gli intervalli indotti hanno dimensione $< 2^k$, questo
        vuol dire che ogni intervallo indotto può contenere al più un intervallo
        di dimensione $2^{k-1}$, altrimenti sarebbe di dimensione $2^k$ e quindi
        assurdo. Inoltre l'intervallo di dimensione $2^{k-1}$ può essere solo in
        fondo al primo intervallo o in testa al secondo (altrimenti potrebbero
        essercene due). Quindi nel caso non si trovasse nessun intervallo di
        dimensione $2^{k-1}$ contenuto vorrebbe dire che lo stesso intervallo
        indotto ha dimensione $< 2^{k-1}$, quindi ripetiamo l'osservazione con
        $k-2$, e così via.
    \end{itemize}

    Il numero totale di intervalli è quindi dato da $2 \log{n}$, in quanto
    per $k_{max}$ ce n'è al più due, in accordo al primo caso, e per valori
    di $k$ minori ce ne può essere al più uno per ogni intervallo indotto,
    siccome andiamo a togliere sempre dalla testa o dalla coda, siccome
    $k_{max} < \log{n}$ abbiamo la tesi.

\end{proof*}

\begin{lemma}[Approssimazione]
    Per la range query con valori non negativi valgono le seguenti proprietà:
    \begin{itemize}
        \item $\mathcal{Q}(l, r) \le \tilde{\mathcal{Q}}(l, r)$
        \item $Pr \left[\tilde{\mathcal{Q}}(l, r) \le \mathcal{Q}(l, r)
            + 2 \varepsilon \log{n} \vectornorm{F}\right] \ge 1-\delta$.
    \end{itemize}
\end{lemma}

\begin{proof*}
    Abbiamo
    \[
        \tilde{\mathcal{Q}}(l, r) = \sum_{k = 0}^{\log{n}-1} \tilde{\mathcal{Q}}(d_{2^k, i_{k, 1}}) +
            \tilde{\mathcal{Q}}(d_{2^k, i_{k, 2}})
    \]
    dove $d_{2^k, i_{k, 1}}$ e $d_{2^k, i_{k, 1}}$ sono i due intervalli
    diadici di dimensione $2^k$ in cui abbiamo scomposto il nostro intervallo
    $[l, r]$, supponendo per tutti gli intervalli non esistenti di avere $d_{a,b} = []$
    e $\mathcal{Q}(d_{a, b}) = 0$.

    Inoltre, per ogni intervallo diadico, dalle proprietà dei count min sketch
    per valori non negativi si ha:
    \[
        \tilde{\mathcal{Q}}(d_{2^k, i}) =
            \mathcal{Q}(d_{2^k, i}) + \sum_{l=1}^{\frac{n}{2^k}} I_{\hat{i}, i, l} \mathcal{Q}(d_{2^k, l})
    \]

    da cui segue la prima disuguaglianza.

    Per la seconda disuguaglianza abbiamo, ponendo $X_{j,i} =
    \sum_{l=1}^{\frac{n}{2^k}} I_{\hat{i}, i, l} \mathcal{Q}(d_{2^k, l})$

    \[
        E[X_{j, i}] \le \frac{\varepsilon}{e} \vectornorm{F}
    \]


    Quindi, su tutti gli intervalli
    \[
        E[\tilde{\mathcal{Q}}(l, r) - \mathcal{Q}(l, r)] \le 2 \frac{\varepsilon}{e} \log{n} \vectornorm{F}
    \]

%    Quindi, per la disuguaglianza di Markov
%    \begin{align}
%        &Pr[\tilde{\mathcal{Q}}(d_{2^k, i}) \ge \mathcal{Q}(d_{2^k, i})
%            + 2 \varepsilon \log{n} \vectornorm{F}] \nonumber \\
%        &= Pr[X_{j, i} \ge 2 \varepsilon \log{n} \vectornorm{F}] \nonumber \\
%        &\le \frac{E[X_{j, i}]}{2 \varepsilon \log{n} \vectornorm{F}} \nonumber \\
%        &\le \frac{2 \frac{\varepsilon}{e} \log{n} \vectornorm{F}}{2 \varepsilon \log{n} \vectornorm{F}} \nonumber \\
%        &= \frac{1}{e} < \frac{1}{2} \nonumber
%    \end{align}

    Da cui, per la disuguaglianza di Markov:
    \begin{align}
        &Pr \left[\tilde{\mathcal{Q}}(l, r) \ge \mathcal{Q}(l, r)
            + 2 \varepsilon \log{n} \vectornorm{F}\right] \nonumber \\
        &=
        Pr \left[\mathcal{Q}(l, r) + \tilde{\mathcal{Q}}(l, r) - \mathcal{Q}(l, r) \ge \mathcal{Q}(l, r)
            + 2 \varepsilon \log{n} \vectornorm{F}\right] \nonumber \\
        &=
        Pr \left[\tilde{\mathcal{Q}}(l, r) - \mathcal{Q}(l, r) \ge
            2 \varepsilon \log{n} \vectornorm{F}\right] \nonumber \\
        &\le
        \left(
            \frac{E[\tilde{\mathcal{Q}}(l, r) - \mathcal{Q}(l, r)]}{2 \varepsilon \log{n} \vectornorm{F}}
        \right) ^ r
            \nonumber \\
        &=
        \left(
            \frac{2 \frac{\varepsilon}{e} \log{n}}{2 \varepsilon \log{n} \vectornorm{F}}
        \right)^r = \left(\frac{1}{e}\right)^r < \left(\frac{1}{2}\right)^r = \delta \nonumber 
    \end{align}
    da cui segue la tesi.

\end{proof*}