\chapter{Prefix tree del codice di Huffman}

\begin{problem*}
  Impostare un algoritmo per costruire il prefix tree del codice di
  Huffman. Dimostrare l'ottimalit\`a di tale albero in termini di numero
  di bit utilizzati per codici prefix free dei simboli.
\end{problem*}

Dato un testo, costruiamo il codice di Huffman come segue

\begin{algorithm}
    \caption{Algoritmo per la costruzione del prefix tree del codice di Huffman}
    \begin{algorithmic}[1]
        \State scorro la stringa mantenendo tante triple della forma $<p, t>$ per ogni carattere $c$, dove $c$ rappresenta il carattere, $p$ la probabilità empirica per $c$ e $t$ è l'albero associato al carattere (in questo caso una foglia contenente come valore $c$).
        \While {ci sono triple}
            \State prendo le due triple $<p1, t1>$, $<p2, t2>$ con minor valore di $p$
            \State creo la nuova tripla $<p', t'>$
            \State $p' \gets p1 + p2$
            \State $t'.left \gets t1$
            \State $t'.right \gets t2$
        \EndWhile
    \end{algorithmic}
\end{algorithm}

Per ottenere i codici associati ai caratteri è sufficiente visitare l'albero, associando ad una ricorsione sul figlio sinistro l'inserimento nel codice di un bit 0 e 1 nel caso del figlio destro, o viceversa.

Si nota che le righe 6 e 7 potevano tranquillamente avere i valori invertiti, in quanto non cambierebbe né la lunghezza dei codici ottenuti, né la proprietà di prefix free, in quanto i valori codificati restano sempre sulle foglie.

\begin{lemma}[Ottimalità]
Sia $H$ l'albero ottenuto con l'algoritmo precedente, e sia $T$ un qualunque altro albero con la stessa struttura ma con le foglie permutate. Vale
\[
    P(H) \le P(T), \mbox{con } P(X) = \sum_{c \in \Sigma} l_{X, c} \, p_c
\]
\end{lemma}

\begin{proof*}
È sufficiente mostrare che scambiando due foglie in $H$ otteniamo un valore più grande di $P(H)$.

Siano $l_a, p_a, l_b, p_b$ lunghezza e probabilità associate a due caratteri $a$ e $b$, vogliamo
dimostrare $l_a p_a + l_b p_b \le l_a p_b + l_b p_a$.

\begin{align*}
    & l_a p_a + l_b p_b \le l_a p_b + l_b p_a \\
    & (l_a - l_b) p_a + (l_b - l_a) p_b \le 0 \\
    & (l_a - l_b) p_a - (l_a - l_b) p_b \le 0 \\
    & (l_a - l_b)(p_a - p_b) \le 0
\end{align*}

che è sempre vero perché $p_a \le p_b \Rightarrow l_b \le l_a$. \qed
\end{proof*}