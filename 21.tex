\chapter{Approssimazione per MAX-SAT}

\begin{problem*}
  Per il problema \maxsat della soddisfacibilità di una formula 
  booleana, si consideri il seguente algoritmo di approssimazione 
  per massimizzare il numero di clausole soddisfatte in una data 
  formula: Sia \(F\) la formula data, \(x_1, x_2,\dots,x_n\) le
  variabili booleane in essa contenute, e \(c_1,c_2,\dots,c_m\)
  le sue clausole. Scegli i valori booleani casuali \(b_1,b_2,\dots,b_n\), 
  ossia ciascun \(b_i\in\{0, 1\}~(1 \le i \le n)\). Calcola il numero \(m_0\) 
  di clausole soddisfatte dall’assegnamento tale che \(x_i := b_i~ (1 \le i 
  \le n)\). Calcola il numero \(m_1\) di clausole soddisfatte 
  dall’assegnamento tale che \(x_i := \overline{b_i}~(1 \le i \le n)\), dove 
  \(\overline{b_i}\) indica la negazione di \(b_i\). Se \(m_0 > m_1\), 
  restituisci l’assegnamento \(x_i := b_i ~(1\le i\le n)\); altrimenti, 
  restituisci l’assegnamento \(x_i := \overline{b_i}~(1 \le i \le n)\). 
  Dimostrare che il suddetto algoritmo è una \(r\)-approssimazione per 
  \maxsat, indicando anche il valore di \(r < 1\) (e motivando l’utilizzo di 
  tale valore). Discutere se, in generale, la scelta di \(b_1,b_2,\dots,b_n\) 
  possa influenzare o meno il valore di \(r\), motivando le argomentazioni 
  addotte. Facoltativo: creare un’istanza di \maxsat in cui il suddetto 
  algoritmo ottiene un costo che è \(r\) volte più piccolo del costo ottimo 
  per una data scelta dei valori di \(b_1,b_2\dots,b_n\).
\end{problem*} 

Vogliamo mostrare che \(r = \frac{1}{2}\). Senza perdita di generalit\`a
possiamo supporre che \(F\) sia in forma normale congiuntiva, cio\`e sia
congiunzione logica di clausole, ciascuna disgiunzione di letterali.
Supponiamo di aver fissato un certo assegnamento \(x_i~:=~b_i~(1\le~i\le~n)\).
Raffiguriamolo con un disegno (TODO)
\begin{figure}[H]
  \centering
  \begin{tikzpicture}
    \node (x_bar_1) at (0,0) [literal,label=below:{\(\overline{x_1}\)}] {};
    \node (x_2) at (1,0) [literal,label=below:{\(x_2\)}] {};
    \node (x_3) at (2,0) [literal,label=below:{\(x_3\)}] {};
    \node (x_bar_4) at (3,0) [literal,label=below:{\(\overline{x_4}\)}] {};
    \node (x_1) at (0,1) [literal,label=above:{\(x_1\)}] {};
    \node (x_bar_2) at (1,1) [literal,label=above:{\(\overline{x_2}\)}] {};
    \node (x_bar_3) at (2,1) [literal,label=above:{\(\overline{x_3}\)}] {};
    \node (x_4) at (3,1) [literal,label=above:{\(x_4\)}] {};
    
    \node (begin_line) at (-2, 0.5) [] {};
    \node (end_line) at (5, 0.5) [] {};
    
    \draw (begin_line) -- (end_line) [thick];
    
    \begin{scope}[on background layer]
      \draw ($(x_bar_2)!0.5!(x_bar_3)$) ellipse 
            [x radius=0.9cm, y radius=0.2cm];   
    \end{scope}
  \end{tikzpicture}
\end{figure}

Siano \(P\) ed \(N\) il numero di clausole che giacciono rispettivamente sopra
e sotto la linea evidenziata, e sia \(C\) il numero di quelle che la 
attraversano. Osserviamo allora che l'algoritmo proposto soddisfa 
\(\text{APPROX} = C+\max\{P,N\}\) clausole, e che questo \`e sicuramente 
minore dell'ottimo OPT, a propria volta minore di \(C+P+N\). Inoltre 
\(C+\max\{P,N\}\) \`e maggiore di \(C+\frac{P+N}{2}\), che \`e maggiore di 
\(\frac{\text{OPT}}{2}\). Mettendo insieme le disugaglianze fin qui scritte 
abbiamo: \[\frac{\text{OPT}}{2}\le C+\frac{P+N}{2}\le\text{APPROX}\le
\text{OPT}\le C+P+N\text{,}\]
