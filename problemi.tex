\documentclass[a4paper, 12pt]{book}

\usepackage[italian]{babel}
\usepackage[utf8]{inputenc}

% Chapters are now called problems.
\addto\captionsitalian{
    \renewcommand\chaptername{Problema}}

\usepackage{xspace}
\usepackage[table]{xcolor}
\usepackage{enumerate}
\usepackage{epigraph}
\usepackage{fullpage}

\usepackage{amsmath}
\usepackage{amsthm}
\usepackage{amsfonts}

\usepackage{tikz}

% Styles used in a drawing in 04.tex
\tikzstyle{el}=[circle,draw=black!100,fill]
\tikzstyle{red_el}=[circle,draw=red!100,fill=red!100]
\tikzstyle{green_el}=[circle,draw=green!100,fill=green!100]

\usepackage{algorithm}
%\usepackage{algorithmicx}
\usepackage[noend]{algpseudocode}

% Hacks needed to get the algorithm package to work.
\renewcommand{\algorithmiccomment}[1]{/\(\!\)/ #1}
\floatname{algorithm}{Algoritmo}

\theoremstyle{plain}
\newtheorem{definition}{Definizione}
\newtheorem*{definition*}{Definizione}
\newtheorem{problem}{Problema}
\newtheorem*{problem*}{Problema}
\newtheorem{lemma}{Lemma}
\newtheorem*{proof*}{Dimostrazione}

\newcommand{\quicksort}{\textsc{Quicksort}\xspace}
\newcommand{\mergesort}{\textsc{Merge Sort}\xspace}
\newcommand{\vectornorm}[1]{\left|\left|#1\right|\right|}

% Command used in 03.tex
\newcommand{\sort}{\mbox{sort}}

% Commands used in 05.tex
\newcommand{\map}{\textsc{Map}\xspace}
\newcommand{\reduce}{\textsc{Reduce}\xspace}
\newcommand{\mapreduce}{\textsc{MapReduce}\xspace}
\newcommand{\parser}{\textsc{Parser}\xspace}

% Command used in 08.tex
\newcommand{\dc}{\textsc{Dc3}\xspace}

% Command used in 22.tex
\newcommand{\E}{\mathbb{E}}

\begin{document}
    \title{Problemi di Algoritmica 2}
    \author{
        Alessandro Ambrosano
        \and
        Jacopo Notarstefano
        \and
        Francesco Salvatori
    }
    
    \maketitle

    \input{quote.tex}
    \chapter{Ordinamento in memoria esterna}

\begin{problem*}
    Mostrare come implementare il \kwaymerge in EMM, ossia la fusione di
    \(n\) sequenze individualmente ordinate e di lunghezza totale \(N\), con
    costo I/O di \(O\left(\frac{N}{B}\right)\), dove \(B\) \`e la dimensione 
    del blocco. Minimizzare e valutare il costo di CPU. Analizzare il costo 
    del \mergesort (I/O complexity, CPU complexity) che utilizza tale 
    \kwaymerge.
\end{problem*}

Date \(k\) sequenze individualmente ordinate, dette \emph{run}, il \kwaymerge 
mantiene in memoria principale \(k\) buffer di input e un buffer di 
output, tutti di dimensione \(B\). A ogni passo seleziona fra i buffer di 
input attivi l'elemento pi\`u piccolo e lo sposta nel buffer di output; 
quando questo risulta pieno il suo contenuto viene copiato in memoria 
esterna. Ogni volta che un buffer di input \`e arrivato al termine 
l'algoritmo copia un nuovo blocco dal corrispondente run; se questo non \`e 
pi\`u possibile il buffer viene marcato come inattivo e non viene pi\`u 
considerato nella selezione dell'elemento pi\`u piccolo. Quando infine rimane 
un solo run attivo viene copiato il suo contenuto in memoria esterna.

\begin{algorithm}
  \caption{\kwaymerge in EMM}
  \begin{algorithmic}[1]
    \State Inizializza \(k\) buffer di input contenenti ciascuno il primo 
    blocco del corrispondente run e marca ogni buffer come attivo
    \While{ci sono almeno due buffer di input attivi}
      \State trova il pi\`u piccolo degli elementi ancora inutilizzati nei 
      buffer attivi
      \State sposta tale elemento nella prima posizione libera del buffer di 
      output
      \If{il buffer di output risulta pieno}
        \State svuotalo in memoria esterna e reinizializzalo
      \EndIf
      \If{il buffer di input da cui ho prelevato l'elemento è giunto alla 
      fine}
        \State copia in tale buffer il prossimo blocco del run corrispondente
        \If {il run \`e terminato}
          \State marca quel buffer come inattivo
        \EndIf
      \EndIf
    \EndWhile
    \State copia l'ultimo run attivo in memoria esterna.
  \end{algorithmic}	
\end{algorithm}

Osserviamo facilmente che il costo in termini di I/O del precedente algoritmo
è \(O\left(\frac{N}{B}\right)\), infatti ciascun elemento \`e coinvolto in 
\(O\left(1\right)\) trasferimenti fra memoria principale e secondaria: uno 
quando entra in un buffer di input e uno quando esce dal buffer di output.

\`E pi\`u delicato valutare il costo in CPU: se infatti al passo \(3\)
troviamo ripetutamente il minimo con una scansione lineare dei primi elementi
di ogni buffer, paghiamo a ogni passo \(O(k)\) confronti, per un costo totale
di \(O(Nk)\). Mantenendo invece un \minheap dei primi \(k\) elementi 
paghiamo a ogni passo costo logaritmico, per un costo totale in CPU di 
\(O(N\log_{2}{k})\).

Osserviamo inoltre che abbiamo tacitamente supposto che la memoria principale 
sia abbastanza grande da contenere \(k+1\) blocchi, ossia \(k \le \frac{M}{B} 
- 1\). Ci\`o \`e rilevante nell'analisi del \mergesort: infatti il caso base 
di questo algoritmo consiste in \(\left\lceil\frac{N}{M}\right\rceil\)
run ordinati grandi quanto la memoria principale. Se questi sono meno di 
\(\frac{M}{B} - 1\) allora basta eseguire una sola volta il \kwaymerge
per ottenere l'ordinamento in memoria esterna.

Se invece cos\`i non fosse dobbiamo eseguire pi\`u volte il \kwaymerge.
A ogni esecuzione il numero di run si riduce di un fattore \(k\), dunque
viene eseguito \(\log_{\frac{M}{B}}{\frac{N}{M}}\) volte. Di conseguenza il 
\mergesort che fa uso del \kwaymerge ha un costo di
\[O\left(\frac{N}{B}\log_{\frac{M}{B}}{\frac{N}{M}}\right) = 
O\left(\frac{N}{B}\log_{\frac{M}{B}}{\left(\frac{N}{M}\cdot\frac{M}{B}\right)}
\right) = O\left(n\log_{m}{n}\right)\mbox{ I/O,}\] che sappiamo essere ottimo.

TODO

    \chapter{Limite inferiore per la permutazione}

\begin{problem*}
    Estendere l'argomentazione usata per il limite inferiore del problema 
    dell'ordinamento in memoria esterna a quello della permutazione: dati
    \(N\) elementi \(e_1,e_2,\dots ,e_N\) e un array \(\pi\) contenente
    una permutazione degli interi in \([1,2,\dots ,N]\), disporre gli
    elementi secondo la permutazione in \(\pi\). Dopo tale operazione,
    la memoria esterna deve contenerli nell'ordine
    \(e_{\pi[1]},e_{\pi[2]},\dots ,e_{\pi[N]}\).
\end{problem*}

    \chapter{Permutazione in memoria esterna}

\begin{problem*}
    Dati due array \(A\) e \(C\) di \(N\) elementi, dove \(A\) \`e l'input e 
    \(C\) una permutazione di \(\left\{0,1,\dots ,n-1\right\}\), descrivere e
    analizzare nel modello EMM un algoritmo ottimo per costruire \(A[C[i]]\)
    per \(0\le i\le n-1\).
\end{problem*}
 
    \chapter{Multi-selezione in memoria esterna}    

\begin{problem*}
    Scrivere tutti i passaggi dell'analisi del costo e della correttezza
    dell'algoritmo di multi-selezione visto a lezione.
\end{problem*}

Vogliamo esibire un algoritmo che selezioni un certo numero di pivot da un
insieme \(S\) di cardinalit\`a \(N\) in modo tale che la distanza fra pivot
consecutivi sia piccola. Il nostro scopo sar\`a usare questo algoritmo per
costruire un analogo del \quicksort in memoria esterna, cos\`i come la \(k\)
-way merge ci ha permesso di costruire l'algoritmo di \mergesort in 
memoria esterna.

Ci potremmo aspettare di dover trovare \(m\) pivot, in analogia a quanto
facciamo per la Merge. In realt\`a \`e sufficiente determinarne \(\sqrt{m}\).
Diamo di seguito l'algoritmo e due lemmi. Nel primo dimostreremo il costo
lineare, nel secondo la correttezza dell'algoritmo.

\begin{algorithm}
    \caption{Multi-selezione in memoria esterna}
    \begin{algorithmic}[1]
        \State Carico e ordino in memoria principale \(\frac{N}{M}\) run
        di \(M\) elementi ciascuno.
        \State Da ogni run seleziono un elemento ogni 
        \(\frac{\sqrt{m}}{4}\) e chiamo \(G\) (elementi verdi) l'insieme 
        degli elementi selezionati.
        \State Uso l'algoritmo dei cinque autori \(\sqrt{m}\) volte per
        selezionare in \(G\) un elemento ogni \(\frac{4N}{m}\) e chiamo 
        \(R\) (elementi rossi) l'insieme degli elementi selezionati.
        \State Ritorno \(R\).
    \end{algorithmic}
\end{algorithm}

\begin{lemma}[Costo]
    L'algoritmo compie \(O(n)\) I/O.
\end{lemma}
\begin{proof}
    La prima riga dell'algoritmo comporta soltanto di scandire tutti gli
    elementi: l'ordinamento di ogni run viene infatti svolto in memoria
    principale, e non comporta ulteriori I/O. Anche la seconda riga
    consiste in una scansione di tutti gli elementi. Per stimare il numero
    di I/O della terza riga sfruttiamo invece il fatto che ogni esecuzione
    dell'algoritmo dei cinque autori comporta una scansione di tutti gli
    elementi. Abbiamo dunque \(\sqrt{m}\) scansioni di \(|G|\) elementi,
    perci\`o:
    \[
        \sqrt{m}\cdot O\left(\frac{|G|}{B}\right) = \sqrt{m}\cdot O\left(\frac{4N}{B\sqrt{m}}\right) = O\left(\frac{4N}{B}\right) = O(n)\mbox{,}
    \]
    dove la prima eguaglianza discende dal fatto che, avendo selezionato
    un elemento ogni \(\frac{\sqrt{m}}{4}\), la cardinalit\`a di \(G\) \`e
    \(\frac{4N}{\sqrt{m}}\). Ogni riga contribuisce quindi \(O(n)\) I/O, da cui la tesi.
\end{proof}

\begin{lemma}[Correttezza]
    Il numero di elementi di \(S\) compresi fra due elementi di \(R\) \`e
    minore di \(\frac{3}{2}\frac{N}{\sqrt{m}}\).
\end{lemma}
\begin{proof}
    Vogliamo dunque stimare il numero di elementi di \(S\) compresi fra
    due generici elementi rossi \(r_1\) e \(r_2\). Possiamo dividerli in
    tre categorie:
    \begin{itemize}
        \item Gli elementi \emph{verdi} compresi fra i due elementi rossi \(r_1\) e \(r_2\).
        \item Gli elementi \emph{senza colore} compresi fra due elementi verdi.
        \item Gli elementi \emph{senza colore} compresi fra un rosso e un verde.
    \end{itemize}
        
    I primi sono facilmente maggiorati da \(X = \frac{4N}{m}\): nella 
    terza riga dell'algoritmo abbiamo infatti scelto un rosso ogni 
    \(\frac{4N}{m}\) elementi verdi.
        
    I secondi sono invece maggiorati da \(Y = \frac{N}{\sqrt{m}} - \frac{4N}{m}\). Per la seconda riga dell'algoritmo abbiamo infatti 
    \(\frac{\sqrt{m}}{4}-1\) elementi senza colore fra due verdi
    consecutivi appartenenti allo stesso run, avendo scelto un verde ogni
    \(\frac{\sqrt{m}}{4}\) elementi. Per la terza riga abbiamo al pi\`u
    \(\frac{4N}{m}\) elementi verdi fra \(r_1\) e \(r_2\), dunque al
    pi\`u \(\frac{4N}{m}\) coppie di elementi verdi consecutivi nello
    stesso run. Ma allora il numero cercato \`e stimato dal prodotto, e 
    quindi da:
    \[
        \frac{4N}{m}\left(\frac{\sqrt{m}}{4}-1\right) = \frac{N}{\sqrt{m}} - \frac{4N}{m}\mbox{.}
    \]
        
    I terzi sono invece maggiorati da \(Z = \frac{n}{2\sqrt{m}} - \frac{2n}{m}\). Per vedere questo abbiamo bisogno della figura.
    \begin{figure}
        \centering
        \begin{tikzpicture}
            \node (l_boundary_down) at (3, -0.5) [] {};
            \node (l_boundary_up) at (3, 4.5) [] {};
            \draw (l_boundary_down) -- (l_boundary_up) [thick];
                
            \node (r_boundary_down) at (8, -0.5) [] {};
            \node (r_boundary_up) at (8, 4.5) [] {};
            \draw (r_boundary_down) -- (r_boundary_up) [thick];
                
            \fill [fill=yellow!50] (3, -0.5) -- (3, 4.5) -- (4.25, 4.5) -- (4.25, -0.5) -- cycle;             
            \fill [fill=yellow!50] (8, -0.5) -- (8, 4.5) -- (6.75, 4.5) -- (6.75, -0.5) -- cycle;
                
            \node (run_1_begin) at (0,0) [] {};
            \node (run_1_end) at (10,0) [] {};
                
            \draw (run_1_begin) -- (run_1_end) [];
            \node (el_1_1) at (1,0) [green_el] {};
            \node (el_1_2) at (1.5,0) [el] {};
            \node (el_1_3) at (2.5,0) [el] {};
            \node (el_1_4) at (4,0) [green_el] {};
            \node (el_1_5) at (5.5,0) [el] {};
            \node (el_1_6) at (7.5,0) [el] {};
            \node (el_1_7) at (8.5,0) [green_el] {};
            \node (el_1_8) at (9,0) [el] {};
            \node (el_1_9) at (9.5,0) [el] {};
                
            \node (run_2_begin) at (0,1) [] {};
            \node (run_2_end) at (10,1) [] {};
                
            \draw (run_2_begin) -- (run_2_end) [];
            \node (el_2_1) at (0.5,1) [green_el] {};
            \node (el_2_2) at (1,1) [el] {};
            \node (el_2_3) at (2.5,1) [el] {};
            \node (r_1) at (3,1) [red_el] {};
            \node (el_2_5) at (3.5,1) [el] {};
            \node (el_2_6) at (4,1) [el] {};
            \node (el_2_7) at (7.5,1) [green_el] {};
            \node (el_2_8) at (8.5,1) [el] {};
            \node (el_2_9) at (9,1) [el] {};
                
            \node (run_3_begin) at (0,2) [] {};
            \node (run_3_end) at (10,2) [] {};
                
            \draw (run_3_begin) -- (run_3_end) [];
            \node (el_3_1) at (0.5,2) [green_el] {};
            \node (el_3_2) at (1.5,2) [el] {};
            \node (el_3_3) at (2,2) [el] {};
            \node (el_3_4) at (3.5,2) [green_el] {};
            \node (el_3_5) at (5,2) [el] {};
            \node (el_3_6) at (5.5,2) [el] {};
            \node (el_3_7) at (7.5,2) [green_el] {};
            \node (el_3_8) at (9,2) [el] {};
            \node (el_3_9) at (9.5,2) [el] {};
                
            \node (run_4_begin) at (0,3) [] {};
            \node (run_4_end) at (10,3) [] {};
                
            \draw (run_4_begin) -- (run_4_end) [];
            \node (el_4_1) at (1,3) [green_el] {};
            \node (el_4_2) at (1.5,3) [el] {};
            \node (el_4_3) at (2.5,3) [el] {};
            \node (el_4_4) at (4.5,3) [green_el] {};
            \node (el_4_5) at (7,3) [el] {};
            \node (el_4_6) at (7.5,3) [el] {};
            \node (r_2) at (8,3) [red_el] {};
            \node (el_4_8) at (8.5,3) [el] {};
            \node (el_4_9) at (9.5,3) [el] {};
                
            \node (run_5_begin) at (0,4) [] {};
            \node (run_5_end) at (10,4) [] {};
                
            \draw (run_5_begin) -- (run_5_end) [];
            \node (el_5_1) at (1.5,4) [green_el] {};
            \node (el_5_2) at (2,4) [el] {};
            \node (el_5_3) at (2.5,4) [el] {};
            \node (el_5_4) at (4,4) [green_el] {};
            \node (el_5_5) at (4.5,4) [el] {};
            \node (el_5_6) at (6.5,4) [el] {};
            \node (el_5_7) at (7,4) [green_el] {};
            \node (el_5_8) at (7.5,4) [el] {};
            \node (el_5_9) at (9,4) [el] {};
        \end{tikzpicture}
        \caption{Ogni riga orizzontale rappresenta un run ordinato, e i 
        cerchietti gli elementi di ogni run. Cerchietti rossi e verdi 
        rappresentano rispettivamente gli elementi di \(R\) e \(G\), 
        cerchietti neri i restanti elementi senza colore. Sono inoltre 
        raffigurati in giallo i bordi definiti dalla posizione degli 
        elementi rossi nei quali possiamo avere elementi senza colore 
        compresi fra un rosso e un verde.}
    \end{figure}
    Osserviamo infatti che la posizione dei due elementi rossi definisce
    un paio di ``bordi'' (raffigurati in giallo in figura) in cui \`e
    possibile trovare gli elementi del terzo tipo. Ognuno di questi 
    bordi può contenere al pi\`u \(\frac{\sqrt{m}}{4}-1\) elementi,
    perché se ne contenesse di pi\`u conterrebbe sicuramente due verdi,
    quindi elementi compresi fra due verdi. Per ognuno degli 
    \(\frac{N}{M}\) run abbiamo insomma \(2\) bordi contenenti al pi\`u 
    \(\frac{\sqrt{m}}{4}-1\) elementi, perci\`o possiamo stimare il
    numero di elementi del terzo tipo con il loro prodotto, cio\`e:
    \[
        2 \cdot \frac{n}{m} \cdot \left( \frac{\sqrt{m}}{4}-1 \right) = \frac{n}{2\sqrt{m}} - \frac{2n}{m}\mbox{,}
    \]
    dove abbiamo usato che \(\frac{N}{M}\) = \(\frac{n}{m}\) essendo
    \(n = \frac{N}{B}\) e \(m = \frac{M}{B}\).
        
    Di conseguenza il numero totale degli elementi compresi fra \(r_1\) 
    e \(r_2\) è maggiorato da:
    \begin{align}
        X + Y + Z &= \frac{4N}{m} + \frac{N}{\sqrt{m}} - \frac{4N}{m} + \frac{n}{2\sqrt{m}} - \frac{2n}{m} \nonumber \\
        &\le \frac{N}{\sqrt{m}} + \frac{n}{2\sqrt{m}} \nonumber
    \end{align}
    nella quale abbiamo cancellato i termini uguali di segno opposto e un
    termine negativo, ottenendo un'ulteriore maggiorazione. Abbiamo
    inoltre:
    \begin{align}
        \frac{N}{\sqrt{m}} + \frac{n}{2\sqrt{m}} &\le \frac{N}{\sqrt{m}} + \frac{N}{2\sqrt{m}} \nonumber \\
        &= \frac{3}{2}\frac{N}{\sqrt{m}} \nonumber
    \end{align}
    essendo \(n = \frac{N}{B}\), da cui la tesi.
\end{proof}
    \chapter{MapReduce}

\begin{problem*}
    Utilizzare il paradigma Scan \& Sort mediante la MapReduce per calcolare la
    distribuzione dei gradi in ingresso delle pagine Web. In particolare, specificare
    quanti passi di tipo MapReduce sono necessari e quali sono le funzioni Map e
    Reduce impiegate. Ipotizzare di avere gi\`a tali pagine a disposizione.
\end{problem*}


    \chapter{Navigazione implicita in vEB}

\begin{problem*}
    Dato un albero completo memorizzato secondo il layout di van Emde Boas (vEB) in
    modo implicito, ossia senza l'ausilio di puntatori (come succede nello heap binario
    implicito), trovare la regola per navigare in tale albero senza usare puntatori
    espliciti.
\end{problem*}


    \chapter{Layout di alberi binari}

\begin{problem*}
  Riportare tutti i passaggi dell'analisi della paginazione descritta nella
  tesi di David Clark (reperibile nella pagina del corso) per un generico
  albero binario in blocchi di dimensioni \(B\). Opzionale: per chi vuole,
  esiste una versione pi\`u impegnativa di questo esercizio dove occorre
  analizzare una paginazione in cui un cammino radice-nodo di lunghezza \(\ell
  \) attraversa \(O(\ell\,/\log{B})\) pagine; contattarmi per discutere questa
  opzione.
\end{problem*}

Sia \(T\) un albero binario di altezza \(H\) su \(n\) nodi. Vogliamo 
partizionare i vertici in sottoinsiemi (detti \emph{pagine}) in modo che:
\begin{enumerate}
  \item Ogni pagina sia una componente connessa dell'albero contenente al 
  pi\`u \(B\) nodi interni.
  \item Sia minimo il numero massimo di pagine attraversate in un cammino
  dalla radice ad una foglia.
\end{enumerate}

    \chapter{Suffix array in memoria esterna}

\begin{problem*}
  Utilizzando la costruzione del suffix array basata sul \mergesort e la tecnica
  \dc vista a lezione, progettare un algoritmo per EMM per costruire il suffix
  array di un testo che abbia la stessa complessit\`a del \mergesort in EMM.
\end{problem*}

Diamo di seguito un esempio di applicazione dell'algoritmo \dc alla stringa
\texttt{\string"abracadabra\string"}. Per ragioni tecniche aggiungiamo tre 
caratteri speciali \texttt{\$} in fondo alla stringa, ottenendo 
\texttt{\string"abracadabra\$\$\$\string"}. Scandiamo la stringa e sostituiamo 
ogni carattere con il proprio rango nell'ordine alfabetico dei caratteri della 
stringa stessa, con la convenzione che il carattere \texttt{\$} precede ogni altro 
carattere. Otteniamo dunque:
\begin{table}[h]
  \begin{tabular}{l*{14}{c}r}
    Carattere              & \texttt{a} & \texttt{b} & \texttt{r} & \texttt{a}
                           & \texttt{c} & \texttt{a} & \texttt{d} & \texttt{a}
                           & \texttt{b} & \texttt{r} & \texttt{a} & \texttt{\$}
                           & \texttt{\$} & \texttt{\$} \\
    \hline
    Rango del carattere    & 1 & 2 & 5 & 1 & 3 & 1 & 4 & 1 & 2 & 5 & 1 & 0 & 0 & 0 \\
    Indice del suffisso    & 0 & 1 & 2 & 3 & 4 & 5 & 6 & 7 & 8 & 9 & 10 & 11 & 12 & 13 \\
  \end{tabular}
\end{table}

Dividiamo i caratteri in \(3\) gruppi secondo la congruenza modulo \(3\) della 
posizione nella stringa. Assegnamo rispettivamente i colori verde, rosso e giallo 
alle \(3\) classi di congruenza:
\begin{table}[h]
  \begin{tabular}{l*{14}{c}r}
    Carattere              & \texttt{a} & \texttt{b} & \texttt{r} & \texttt{a}
                           & \texttt{c} & \texttt{a} & \texttt{d} & \texttt{a}
                           & \texttt{b} & \texttt{r} & \texttt{a} & \texttt{\$}
                           & \texttt{\$} & \texttt{\$} \\
    \hline
    Rango del carattere    & 1 & 2 & 5 & 1 & 3 & 1 & 4 & 1 & 2 & 5 & 1 & 0 & 0 & 0 \\
    Indice del suffisso    & 0 \cellcolor{green} & 1 \cellcolor{red} & 2 \cellcolor{yellow} 
                           & 3 \cellcolor{green} & 4 \cellcolor{red} & 5 \cellcolor{yellow} 
                           & 6 \cellcolor{green} & 7 \cellcolor{red} & 8 \cellcolor{yellow} 
                           & 9 \cellcolor{green} & 10 \cellcolor{red} & 11 \cellcolor{yellow} 
                           & 12 \cellcolor{green} & 13 \cellcolor{red} \\
  \end{tabular}
\end{table}

Trascuriamo per il momento il gruppo verde e sostituiamo ogni carattere di indice 
\(i\) dei gruppi giallo e rosso con la tripla dei caratteri di indice \(i, i+1\) e 
\(i+2\). Riordiniamo inoltre la tabella secondo la divisione in colori,
tralasciando le triple che cominciano per il simbolo \texttt{\$}:
\begin{table}[h]
  \begin{tabular}{l*{12}{c}r}
                        & \multicolumn{4}{c}{Gruppo 0 \cellcolor{green} } 
                        & \multicolumn{4}{c}{Gruppo 1 \cellcolor{red} } 
                        & \multicolumn{4}{c}{Gruppo 2 \cellcolor{yellow} }\\
    \hline
    Caratteri           & \multicolumn{12}{c}{} \\
    Indice del suffisso & & & &
                        & 1 & 4 & 7 & 10
                        & 2 & 5 & 8 & 11 \\
  \end{tabular}
\end{table}
    \chapter{Famiglia di funzioni hash uniformi}

\begin{problem*}
    Mostrare che la famiglia di funzioni hash 
    \(H = \left\{h(x) = ((ax + b ) \mod p) \mod m\right\}\) \`e (quasi) uniforme, dove 
    \(a,b\in [m]\) con \(a\neq 0\) e \(p\) \`e un numero primo sufficientemente grande.
\end{problem*}
Siano $k\neq l$ due chiavi, $h\in H$ e $r=ak+b \pmod p$, $s=al+b\pmod p$. Abbiamo che $r-s = a(k-l) \pmod p$ da cui, essendo $a\neq0$ e, per $p$ sufficientemente grande, $(k-l)\neq 0 \pmod p$, segue che $r-s\neq0$.\newline
È dunque possibile una collisione tra gli hash delle due chiavi se e solo se $r \pmod m = s \pmod m$.\newline
Notiamo che per $a$ sono possibili $p-1$ valori (lo $0$ non va bene) mentre per $b$ ve ne sono $p$: in tutto abbiamo $p(p-1)$ possibili coppie $(a, b)$; ma anche per $(r, s)$ vi sono $p(p-1)$ possibilità (fissato $r$, per $s$ vanno bene tutti i valori tranne quello assunto da $r$): segue che le coppie $(a, b)$ e le coppie $(r, s)$ possono essere messe in corrispondenza biunivoca. Inoltre se la coppia $(a, b)$ è scelta a caso con probabilità uniforme ciascuna coppia $(r, s)$ ha la stessa probabilità di essere l'immagine di due date chiavi $l$ e $k$. Questo significa che \[\Pr[~h(k)=h(l)~]=\Pr[~r = s \pmod m].\]
Fissiamo $r$: i possibili valori per $s$ che originano una "collisione" sono al più \[\left\lceil \frac{p}{m} \right\rceil-1 \le \frac{p+m-1}{m} - 1= \frac{p-1}{m}.\]
Da questo segue che \[\Pr[~r=s\pmod m] \le \frac{\frac{p-1}{m}}{p-1}=\frac{1}{m},\] il che significa \[\Pr[h(k)=h(l)]\le\frac{1}{m},\] come volevasi dimostrare.


    \chapter{Count-min sketch: estensione}

\begin{problem*}
    Estendere l'analisi vista a lezione permettendo di incrementare e decrementare
    i contatori con valori arbitrari.
\end{problem*}

\begin{lemma}[Approssimazione]
Per il count min sketch a valori arbitrari vale:

\[
    Pr\left[F[i] - 3\varepsilon\vectornorm{F} \le \tilde{F}[i]
        \le F[i] + 3\varepsilon\vectornorm{F}\right] \ge 1 - \delta^{\frac{1}{4}}
\]

con $\tilde{F}[i] = median_{\substack{j}}\{T[j, h_j(i)]\}$.
\end{lemma}

\begin{proof*}
    Osserviamo innanzi tutto che $\tilde{F}[i] = T[\hat{i}, h_{\hat{i}}(i)]$
    per qualche $\hat{i}$, e vale
    \[
        \tilde{F}[i] = F[i] + \sum_{k = 1}^{n}I_{\hat{i}, i, k}F[k]
    \]
    dove $I_{j, i, k}$ è la variabile indicatrice così definita:
    \[
        I_{j, i, k} =
        \begin{cases}
            1 & \mbox{se } i \neq k \land h_j(i)=h_j(k) \\
            0 & \mbox{altrimenti}
        \end{cases}
    \]

    Ponendo $X_{j, i} = \sum_{k=1}^n I_{j, i, k} F[k]$ abbiamo:
    
    \begin{align*}
        & Pr\left[F[i] - 3\varepsilon\vectornorm{F} \le \tilde{F}[i]
            \le F[i] + 3\varepsilon\vectornorm{F}\right] \nonumber \\
        &= Pr\left[F[i] - 3\varepsilon\vectornorm{F}
            \le F[i] + \sum_{k = 1}^{n}I_{\hat{i}, i, k}F[k]
            \le F[i] + 3\varepsilon\vectornorm{F}\right] \nonumber \\
        &= Pr\left[F[i] - 3\varepsilon\vectornorm{F}
            \le F[i] + X_{\hat{i},i}
            \le F[i] + 3\varepsilon\vectornorm{F}\right] \nonumber \\
        &= Pr\left[- 3\varepsilon\vectornorm{F}
            \le X_{\hat{i},i}
            \le 3\varepsilon\vectornorm{F}\right] \nonumber \\
        &= Pr\left[|X_{\hat{i},i}| \le 3\varepsilon\vectornorm{F}\right] \nonumber
    \end{align*}

    \begin{align*}
        E[I_{j, i, k}] &= Pr\left[i \neq k \land h_j(i) = h_j(k)\right] \\ 
        &= Pr\left[\cup_{a=1}^c i \neq k \land h_j(i) = a \land h_j(k) = a \right] \\
        &= \sum_{a=1}^c \underbrace{Pr\left[i \neq k \land h_j(i) = a \land h_j(k) = a \right]}_
            {\mbox{Sono funzioni di hash uniformi}}\\
        &\le \sum_{a=1}^c \frac{1}{c} = \frac{c}{c^2} = \frac{1}{c} = \frac{\varepsilon}{e}
    \end{align*}

    da cui
    \begin{align*}
        E[|X_{j, i}|] &= E\left[|\sum_{k=1}^n I_{j, i, k} F[k]|\right] \\
        &\le \sum_{k=1}^n E[|I_{j, i, k}|] |F[k]| 
        = \frac{\varepsilon}{e} \sum_{k=1}^n |F[k]|]
        = \frac{\varepsilon}{e} \vectornorm{F}
    \end{align*}

    e quindi per ogni $j$, usando la disuguaglianza di Markov
    \[
        Pr\left[|X_{j, i}| \ge 3\varepsilon\vectornorm{F}\right] \le
            \frac{E[|X_{j,i}|]}{3\varepsilon\vectornorm{F}} 
        = \frac{\frac{\varepsilon}{e} \vectornorm{F}}{3\varepsilon\vectornorm{F}}
        = \frac{1}{3e}
    \]

    Notiamo che perché valga $|X_{\hat{i}, i}| \ge 3\varepsilon\vectornorm{F}$,
    deve valere $|X_{j, i}| \ge 3\varepsilon\vectornorm{F}$, per questo scopo
    introduciamo le seguenti variabili casuali:

    \[
        Y_j = 
        \begin{cases}
            1 & \mbox{se } |X_{j, i}| \ge 3\varepsilon\vectornorm{F} \\
            0 & \mbox{altrimenti}
        \end{cases}
    \]

    Se poniamo $p = E[Y_j] = Pr[|X_{j,i}| \ge 3 \varepsilon\vectornorm{F}|]$,
    e $Y = Y_1 + \dots + Y_r$, abbiamo $E[Y] = rp$. \\
    Noi vogliamo $Y \ge \frac{r}{2}$, per calcolarne la probabilità dobbiamo
    introdurre la disuguaglianza di Chernoff.

    \begin{definition*}[Chernoff bound]
        Se $X_1, \dots, X_n$ sono n prove di Poisson indipendenti, identicamente
        distribuite, ossia $\forall i. P[X_i = 1] = p, P[X_i = 0] = 1-p$, se
        prendiamo $X=\sum_{i=1}^n X_i$ e $\mu = E[X]$, per ogni $\lambda > 0$ si ha:
        \[ Pr\left[X \ge (1+\lambda)\mu\right] <
            \left(\frac{e^\lambda}{(1+\lambda)^{1+\lambda}}\right)^\mu\]
    \end{definition*}

    Quindi ponendo $\mu = E[Y] = rp$, $(1+\lambda)\mu = \frac{r}{2} \Rightarrow
    1+\lambda = \frac{1}{2p}$, abbiamo
    \begin{align*}
        Pr\left[Y > \frac{r}{2}\right] &< \left(\frac{e^\lambda}{(1+\lambda)^{1+\lambda}}\right)^\mu
        = \frac{1}{e^\mu}\left(\frac{e}{1+\lambda}\right)^{(1+\lambda)\mu} \\
        &= \frac{1}{e^{rp}}\left(\frac{e}{\frac{1}{2p}}\right)^{\frac{r}{2}}
        = \frac{1}{e^{rp}}(2pe)^{\frac{r}{2}}
    \end{align*}

    Quindi se $\frac{1}{e^{rp}}(2pe)^{\frac{r}{2}} \le \frac{1}{2^{\frac{r}{4}}} (= \delta^{\frac{1}{4}})$
    abbiamo concluso:

    \begin{align*}
        \frac{1}{e^{rp}}(2pe)^{\frac{r}{2}} \le \frac{1}{2^{\frac{r}{4}}}
        & \equiv \underbrace{2^{\frac{r}{4}} \le e^{rp} \frac{1}{(2pe)^{\frac{r}{2}}}}
            _{rp \ge 0 \Rightarrow e^{rp} \ge 1}
        \Leftarrow 2^{\frac{r}{4}} \le \frac{1}{(2pe)^{\frac{r}{2}}} \\
        & \equiv 2^{\frac{1}{2}} \le \frac{1}{2pe}
        \equiv p \le \frac{1}{2\sqrt{2}e}
    \end{align*}

    che vale in quanto $2\sqrt{2}e\sim 7.668$ e $p < \frac{1}{8}$. \qed
\end{proof*}
    \chapter{Count-min sketch: prodotto scalare}

\begin{problem*}
    Mostrare come utilizzare il paradigma del count-min sketch per approssimare il
    prodotto scalare (i.e., approssimare \(\sum_{k=1}^n{F_a[k]\cdot F_b[k]}\)).
\end{problem*}

\begin{lemma}[Approssimazione]
    Per il prodotto scalare, con valori non negativi per $F_a$ e $F_b$ valgono
    \begin{itemize}
        \item $a \cdot b \le \tilde{a \cdot b}$
        \item $Pr[\tilde{a \cdot b} \le a \cdot b
            + \varepsilon \vectornorm{a}\vectornorm{b}] \ge 1 - \delta$
    \end{itemize}
\end{lemma}

con $\displaystyle \tilde{a \cdot b} = \min_j\{\sum_{i=1}^{c}T_a[j,h_j(i)]T_b[j,h_j(i)]\}$.

\begin{proof*}
Osserviamo innanzi tutto che per un certo $\hat{i}$ vale

\[
    \tilde{a \cdot b} = \sum_{i=1}^c T_a[\hat{i}, h_{\hat{i}}(i)] T_b[\hat{i}, h_{\hat{i}}(i)]
\]

e se definiamo la variabile $I_{j, i, k}$ come

\[
    I_{j, i, k} =
    \begin{cases}
        1 & \mbox{se } i \neq k \land h_j(i) = h_j(k) \\
        0 & \mbox{altrimenti}
    \end{cases}
\]

otteniamo la seguente uguaglianza:

\[
    \tilde{a \cdot b} = a \cdot b + \sum_{p, q} I_{\hat{i}, p, q} F_a[p] F_b[q]
\]

\begin{itemize}
\item Non vale ridere perché ho scritto fap
\item Non vale tornare indietro per leggere se l'ho scritto davvero
\end{itemize}

pertanto, vista la non negatività degli elementi di $F_a$ e $F_b$ la prima
disuguaglianza vale banalmente.

Per la seconda, ragioniamo col complementare e cerchiamo di calcolare
\[ Pr[\tilde{a \cdot b} \ge a \cdot b + \varepsilon \vectornorm{a}\vectornorm{b}] \]

Che, se poniamo $X_{j, i} = \sum_{p, q} I_{j, p, q} F_a[p] F_b[q]$ equivale a calcolare

\[ Pr[X_{\hat{i}, i} \ge \varepsilon \vectornorm{a}\vectornorm{b}] \]

Per poterla calcolare ci servono innanzi tutto le speranze di $E[I_{j, i, k}]$
e $E[X_{j,i}]$, date da:

\begin{align}
    E[I_{j, i, k}] &= Pr\left[i \neq k \land h_j(i) = h_j(k)\right] \nonumber \\
    &= Pr\left[\cup_{a=1}^c i \neq k \land h_j(i) = a \land h_j(k) = a \right] \nonumber \\
    &= \sum_{a=1}^c Pr\left[i \neq k \land h_j(i) = a \land h_j(k) = a \right] \nonumber \\
    &= \sum_{a=1}^c Pr\left[h_j(i) = a\right]Pr\left[i \neq k \land h_j(k) = a \right] \nonumber \\
    &= \frac{c}{c^2} = \frac{1}{c} = \frac{\varepsilon}{e} \nonumber
\end{align}

\begin{align}
    E[X_{j, i}] &= E[\sum_{i, k} I_{j, i, k} F_a[i] F_b[k]] \nonumber \\
    &= \sum_{i, k} E[I_{j, i, k}] F_a[i] F_b[k] \nonumber \\
    &= \frac{\varepsilon}{e} \sum_{i, k} F_a[i]F_b[k] \nonumber \\
    &\le \frac{\varepsilon}{e} \vectornorm{a}\vectornorm{b} \nonumber
\end{align}

Si ha quindi, per la disuguaglianza di Markov:

\begin{align}
    Pr[X_{j, i} \ge \varepsilon \vectornorm{a}\vectornorm{b}] &\le
        \frac{E[X_{j, i}]}{\varepsilon \vectornorm{a}\vectornorm{b}} \nonumber \\
        &= \frac{\frac{\varepsilon}{e} \vectornorm{a}\vectornorm{b}}
            {\varepsilon \vectornorm{a}\vectornorm{b}} \nonumber \\
        &= \frac{1}{e} < \frac{1}{2} \nonumber
\end{align}

Osserviamo che affinché valga $X_{\hat{i}, i} \ge \varepsilon \vectornorm{a}\vectornorm{b}$,
è necessario che valga per tutte le $r$ $X_{j, i}$, in quanto $X_{\hat{i}, i}$
è la minima tra queste, trattandosi le $X_{j, i}$ di variabili casuali
indipendenti, abbiamo che:
\[ Pr[X_{\hat{i}, i} \ge \varepsilon \vectornorm{a}\vectornorm{b}] < \frac{1}{2^r} = \delta, \]
da cui
\[ Pr[X_{\hat{i}, i} \le \varepsilon \vectornorm{a}\vectornorm{b}] \ge 1 - \delta. \]
\qed
\end{proof*}
    \chapter{Count-min sketch: interval query}

\begin{problem*}
    Mostrare come utilizzare il paradigma del count-min sketch per rispondere alle
    interval query (i.e., approssimare \(\sum_{k=i}^j{F[k]}\)).
\end{problem*}


    \chapter{Elementi distinti}

\begin{problem*}
    Progettare e analizzare un algoritmo di data streaming che permetta di
    approssimare il numero di elementi distinti.
\end{problem*}

Cominciamo fissando le notazioni usate nel seguito. Sia \(\chi = x_1 x_2
\dots x_n\) uno stream di \(n\) caratteri scelti da un insieme di cardinalit\`a
\(m\le n\) di simboli distinti. Denotiamo con \(D(\chi)\) il numero di elementi
distinti effettivamente presenti nello stream.

Per prima cosa risolviamo un problema pi\`u semplice: dato un numero \(t\)
vogliamo decidere con alta probabili\`a se \(D(\chi) \gg t\) oppure \(D(\chi) \ll
t\). Pi\`u precisamente vogliamo risolvere il seguente
\begin{problem*}
  Data una soglia \(t\) e uno stream \(\chi\), rispondere
  \begin{itemize}
    \item S\`i se \(D(\chi)\ge t\),
    \item No se \(D(\chi)<\frac{t}{2}\),
    \item indifferentemente S\`i o No altrimenti,
  \end{itemize}
  con probabilit\`a maggiore di \(1 - \delta\) di essere corretto.
\end{problem*}

Supponiamo di avere una funzione hash \(h:X\rightarrow [1,t]\) ideale, tale
cio\`e che \(\forall i\mbox{, Pr}[h(x)=i]=\frac{1}{t}\). In termini di
questa funzione siamo quasi in grado di risolvere il precedente problema.
Abbiamo infatti il seguente
\begin{algorithm}
  \caption{Contatore con rumore}
  \begin{algorithmic}[1]
    \Function{NoisyCounter}{h}
    \For {each \(x_i\in X\)}
      \If {\(h(x_i) = t\)}
        \State \Return {S\`i}
      \EndIf
    \EndFor
    \State \Return {No}
    \EndFunction
  \end{algorithmic}
\end{algorithm}

\begin{lemma}
  Supponiamo che \(D(\chi)\ge t\) o \(D(\chi)<\frac{t}{2}\). Allora il precedente
  algoritmo \`e corretto con probabilit\`a maggiore di \(0.6\).
\end{lemma}
\begin{proof}
  Osserviamo che basta stimare la probabilit\`a di rispondere ``No''. Infatti
  questa sar\`a la probabilit\`a d'errore qualora \(D(\chi)\) fosse maggiore di
  \(t\) e la probabilit\`a di successo nell'altro caso.
  \begin{itemize}
    \item Supponiamo \(D(\chi)\ge t\). TODO
    \item Supponiamo invece \(D(\chi)<\frac{t}{2}\). TODO
  \end{itemize}
\end{proof}

Abbiamo quindi un modo per ottenere una risposta corretta in pi\`u del \(60\%\)
dei casi. Possiamo superare qualunque soglia di confidenza semplicemente
ripetendo abbastanza volte il precedente algoritmo e scegliendo il risultato
ottenuto pi\`u frequentemente. Sia dunque \(H = \left\{h_j\,|\,j\in[k]\right\}\) 
una famiglia di funzioni hash ideali e indipendenti.
\begin{algorithm}
  \caption{}
  \begin{algorithmic}[1]
    \State {yes\_votes = 0}
    \For {each \(h_j\in H\)}
      \If {\(\textsc{NoisyCounter}(h_j)\) = S\`i}
        \State {yes\_votes\,++}
      \EndIf
    \EndFor
    \If {yes\_votes \(>\frac{k}{2}\)}
      \State \Return {S\`i}
    \EndIf
    \State \Return {No}
  \end{algorithmic}
\end{algorithm}

\begin{lemma}
  TODO
\end{lemma}
\begin{proof}
  Sia \(Z_i\) l'indicatrice dell'evento \(\{\textsc{NoisyCounter}(h_j)=\mbox{S\`i}\}\).
  Il precedente lemma garantisce che \(\mbox{Pr}[Z_i]\ge 0.6\). Sia allora \(Z=
  \sum_{i=1}^k{Z_i}\). Osserviamo che dalla linearit\`a della speranza segue 
  immediatamente che \(\E[Z]\ge 0.6k\). Di conseguenza abbiamo:
  \[\mbox{Pr}\left[Z\le\frac{k}{2}\right]\le\mbox{Pr}\left[Z\le\frac{0.5}{0.6}
  \E[Z]\right]\mbox{.}\]
  Possiamo maggiorare il membro di destra con una delle disuguaglianze di
  Chernoff, cio\`e:
  \[\mbox{Pr}[Z\le(1-\xi)\E[Z]]\le\exp{\left(-\frac{\E[Z]\cdot\xi^2}{2}\right)}\mbox{.}\]
  Abbiamo dunque, ponendo \(1-\xi = \frac{0.5}{0.6}\), che:
  \[\mbox{Pr}\left[Z\le\frac{k}{2}\right]\le\exp{\left(-\frac{0.6 k\cdot{\frac{0.1}{0.6}}^2}{2}\right)}\mbox{,}\]
  perci\`o possiamo rendere arbitrariamente  piccola la probabilit\`a di rispondere
  ``No''. Di conseguenza, ragionando come nella dimostrazione del precedente lemma,
  possiamo rendere arbitrariamente vicina a \(1\) la probabilit\`a di successo.
\end{proof}

    \chapter {Cuckoo hashing}

\begin{problem*}
    Scrivere tutti i passaggi dell'analisi del costo dell'inserimento
    di un elemento in una tabella di cuckoo hashing. Discutere anche
    della cancellazione e della sua complessità.
\end{problem*}

Dobbiamo fare hashing da un insieme di $n$ elementi ad un insieme di $r$
elementi. Invece di procedere nel modo classico usando le liste di adiacenza
(che nel caso pessimo potrebbero contenere tutti gli elementi), vogliamo un
modo che ci permetta di avere esattamente un elemento in associato ad ognuno
degli $r$ valori (quindi che sia possibile salvare gli $n$ elementi su un array
di dimensione $r$).
Per farlo, ci avvaliamo di due funzioni hash, 2-wise indipendenti, $h_1(x)$ e
$h_2(x)$, in questo modo:

\begin{algorithm}
    \caption{Inserimento in Cuckoo hashing}
    \begin{algorithmic}[1]
        \State provo ad inserire x in $h_1(x)$
        \State se la cella è libera lo inserisco semplicemente
        \State se la cella non è libera tolgo l'elemento y dalla cella $h_1(x)$
        e ripeto il procedimento per y, provando a inserirlo in $h_1(y)$ se
        $h_1(x) = h_2(y)$ o in $h_2(y)$ se $h_1(x) = h_1(y)$, per un massimo di
        $n$ volte.
        \State se il valore non è stato ancora inserito si cambiano le funzioni
        di hashing, e si riprova ad inserire il valore che si stava cercando di
        inserire alla $n$-esima iterazione.
    \end{algorithmic}
\end{algorithm}

Si nota, come vedremo, che il numero di iterazioni massimo nel punto 3 serve
per evitare di andare in loop quando si tenta di inserire un valore.

Per fare l'analisi è necessario introdurre i concetti di grafo Cuckoo e di
bucket, più un lemma:

\begin{definition*}[Grafo Cuckoo]
    Il grafo Cuckoo è un grafo che ha per nodi le celle dell'array, con un arco
    uscente in ogni cella contenente un valore, che punta alla cella alternativa
    secondo le funzioni $h_1(x)$, $h_2(x)$. Ovvero, se il nodo $i$ contiene il
    valore $x$, $i$ avrà un arco uscente verso $h_2(x)$ se $i = h_1(x)$ o
    viceversa un arco verso $h_1(x)$ se $i = h_2(x)$.
\end{definition*}

\begin{definition*}[Bucket]
    Si dice \emph{bucket} di un valore $x$ l'insieme dei nodi raggiungibili dai
    nodi $\left\{h_1(x), h_2(x)\right\}$ nel grafo Cuckoo. Ossia tutti i nodi
    con cui potremmo avere a che fare nel caso volessimo inserire $x$.
\end{definition*}

\begin{lemma}
    Per ogni nodo $i$ e $j$, e ogni $c > 1$, se $r \ge 2cn$, la probabilità che
    esista un cammino tra $i$ e $j$ di lunghezza $l$ è al più $\frac{c^{-l}}{r}
    = \frac{1}{c^lr}$. Ovvero, se il numero di celle nell'array è
    sufficientemente più grande del numero di valori salvati, la probabilità
    che esista un cammino di lunghezza $l$ tra due nodi è $O(\frac{1}{r})$, e
    decresce esponenzialmente.
\end{lemma}

\begin{proof*}
    Procediamo per induzione sulla lunghezza del percorso:
    \begin{itemize}
    \item per $l = 1$: un percorso di lunghezza 1 tra due nodi $i$ e $j$ esiste
    sse per qualche $x$ $h_1(x) = i \land h_2(x) = j$ oppure $h_1(x) = j \land
    h_2(x) = i$, si ha:
    
    \begin{align}
        &Pr\left[(h_1(x) = i \land h_2(x) = j) \lor (h_1(x) = i \land h_2(x) = j) \right] = \nonumber \\
        &= Pr\left[h_1(x) = i \land h_2(x) = j\right] + Pr\left[h_1(x) = j \land h_2(x) = i\right] = \nonumber \\
        &= 2 Pr\left[h_1(x) = i\right]Pr\left[h_2(x) = j\right] = \nonumber \\
        &= 2 \frac{1}{r} \frac{1}{r} = \frac{2}{r^2} \nonumber
    \end{align}

    Siccome il numero di elementi per cui vale la proprietà vista sopra è al
    più n, si ha:

    \begin{align}
        & Pr\left[\exists \mbox{percorso di lunghezza 1 tra $i$ e $j$}\right] \nonumber \\
        &\le n \frac{2}{r^2} = \frac{2n}{r} \frac{1}{r} \nonumber \\
        &\left\{r \ge 2cn \mbox{ per ipotesi} \Rightarrow c \le \frac{r}{2n}
            \Rightarrow \frac{1}{c} \ge \frac{2n}{r} \right\} \nonumber \\
        &\le \frac {1}{cr} = \frac{c^{-1}}{r} \nonumber
    \end{align}

    \item per $l > 1$ è necessario che:
        \begin{enumerate}
            \item Esista un percorso ottimo lungo $l-1$ da $i$ a $k$.
            \item Esista un arco tra $k$ e $j$.
        \end{enumerate}
        Abbiamo
            \[Pr[(1)] = \frac{c^{1-l}}{r}\]
        per ipotesi induttiva. Inoltre, usando lo stesso ragionamento di prima
        otteniamo che
            \[Pr[(2)]= \frac{c^{-1}}{r}.\]

        Notiamo che i valori possibili di k sono $r$ e che quindi la probabilità
        totale è data da:
        \[r Pr[(1)]Pr[(2)] = r\frac{c^{-l}}{r^2} = \frac{c^{-l}}{r}\] \qed
    \end{itemize}
\end{proof*}

La probabilità che al punto 3 dell'inserimento avvenga un reashing, è maggiorata
dalla probabilità che per qualche elemento esista un ciclo. Se notiamo che
\emph{esiste un ciclo di lunghezza $l$}$\Leftrightarrow$\emph{esiste un percorso di
lunghezza $l$ tra $i$ e $i$ } otteniamo:
\begin{align}
    &Pr\left[\exists \mbox{un ciclo per il nodo i nel grafo Cuckoo}\right] \nonumber \\
    &= \sum_{l=1}^{\infty} Pr\left[\exists \mbox{ciclo di lunghezza $l$ nel grafo Cuckoo}\right] \nonumber \\
    &= \sum_{l=1}^{\infty} Pr\left[\exists \mbox{percorso di lunghezza $l$ tra $i$ e $i$}\right] \nonumber \\
    &\le \sum_{l=1}^{\infty} \frac{c^{-l}}{r} = \frac{1}{r(c-1)} \nonumber
\end{align}

da cui:
\begin{align}
    &Pr\left[\mbox{rehashing}\right] \nonumber \\
    &= \sum_{i=1}^{r} Pr\left[\exists \mbox{un ciclo per il nodo i nel grafo cuckoo}\right] \nonumber \\
    &= r*\frac{1}{r(c-1)} = \frac{1}{c-1} \nonumber
\end{align}

Ponendo $c=3$ la probabilità di un rehash è $\frac{1}{2}$, e di n rehash è
$\frac{1}{2^n}$, quindi il numero atteso di rehash ad ogni inserimento è

\[ \sum_{i=1}^{n} i*\frac{1}{2^i} = 2. \]

Il costo medio di un inserimento quindi è dato dal costo di due rehashing, 
ognuno da $\Theta(n)$, quindi a sua volta $\Theta(n)$, mentre il costo
ammortizzato per ogni inserimento è $O(1)$.

La cancellazione avviene in $O(1)$, cercando il valore da cancellare nelle sole
due celle possibili ed eliminandolo, è possibile osservare infatti che la
struttura che si ottiene è ancora un cuckoo hashing dove la funzione $h_1(x)$
è quella che mette tutti gli elementi esattamente dove sono e la funzione
$h_2(x)$ è una qualunque (o varianti equivalenti).
    \chapter{Random search tree}

\begin{problem*}
  Scrivere l'algoritmo per inserire una chiave in un random search tree
  con una sola discesa dalla radice (i.e., senza dover risalire poi dalla
  foglia appena inserita verso la radice mediante le rotazioni).
\end{problem*}


    \chapter{Lista invertita compressa}

\begin{problem*}
  Prendiamo una sequenza ordinata crescente di \(n\) interi
  \(i_1, i_2,\dots ,i_n\), come per esempio una lista invertita.
  La rappresentazione compressa differenziale \`e la sequenza \(S\) di
  \(|S|\) bit ottenuti concatenando
  \(\gamma(i_1),\gamma(i_2 - i_1),\dots ,\gamma(i_{n} - i_{n-1})\), dove
  \(\gamma(x)\) rappresenta il gamma code di Elias per la codifica
  dell'intero \(x\ge 1\) in \(2\lfloor\log_{2}{x}\rfloor + 1\) bit.
  Mostrare come aggiungere un'opportuna directory di spazio \(O(|S|)\) bit
  (meglio ancora, di \(o(|S|)\) bit) per poter accedere velocemente, dato
  \(j\in [2\dots n]\), alla codifica \(\gamma(i_{j} - i_{j-1})\).
  Estendere tale approccio per accedere velocemente a \(i_j\) (e quindi
  poter eseguire una ricerca binaria sugli interi della lista invertita
  compressa).
\end{problem*}


    \chapter{Prefix tree del codice di Huffman}

\begin{problem*}
  Impostare un algoritmo per costruire il prefix tree del codice di
  Huffman. Dimostrare l'ottimalit\`a di tale albero in termini di numero
  di bit utilizzati per codici prefix free dei simboli.
\end{problem*}

Dato un testo, costruiamo il codice di Huffman come segue:
\begin{algorithm}
    \caption{Algoritmo per la costruzione del prefix tree del codice di Huffman}
    \begin{algorithmic}[1]
        \State scorro la stringa mantenendo tante triple della forma \(\langle p,
        t\rangle\) per ogni carattere \(c\), dove \(c\) rappresenta il carattere,
        \(p\) la probabilità empirica per \(c\) e \(t\) è l'albero associato al
        carattere (in questo caso una foglia contenente come valore \(c\)).
        \While {ci sono almeno due triple}
            \State prendo le due triple \(\langle p_1, t_1\rangle\), \(\langle
            p_2, t_2\rangle\) con minor valore di \(p\)
            \State p' \(\gets \text{p}_1 + \text{p}_2\)
            \State t'.left \(\gets \text{t}_1\)
            \State t'.right \(\gets \text{t}_2\)
            \State creo la nuova tripla \(\langle p', t'\rangle\)
        \EndWhile
    \end{algorithmic}
\end{algorithm}

Per ottenere i codici associati ai caratteri è sufficiente visitare l'albero,
associando ad una ricorsione sul figlio sinistro l'inserimento nel codice di un
bit \texttt{0} e \texttt{1} nel caso del figlio destro, o viceversa.

Osserviamo che le righe \(6\) e \(7\) potevano tranquillamente avere i valori
invertiti, in quanto non cambierebbe n\'e la lunghezza dei codici ottenuti,
n\'e la proprietà di prefix free, in quanto i valori codificati restano sempre
sulle foglie.

\begin{lemma}[Ottimalità]
Sia \(H\) l'albero ottenuto con l'algoritmo precedente, e sia \(T\) un qualunque altro albero con la stessa struttura ma con le foglie permutate. Allora abbiamo:
\[
    P(H) \le P(T), \mbox{con } P(X) = \sum_{c \in \Sigma} \ell_{X, c} \, p_c\text{.}
\]
\end{lemma}

\begin{proof}
  È sufficiente mostrare che scambiando due foglie in \(H\) otteniamo un valore
  più grande di \(P(H)\). Siano \(\ell_a, p_a, \ell_b, p_b\) lunghezza e
  probabilità associate a due caratteri \(a\) e \(b\), vogliamo dimostrare 
  \(\ell_a p_a + \ell_b p_b \le \ell_a p_b + \ell_b p_a\). Ma infatti abbiamo:
  \begin{align*}
    & \ell_a p_a + \ell_b p_b \le \ell_a p_b + \ell_b p_a \\
    & (\ell_a - \ell_b) p_a + (\ell_b - \ell_a) p_b \le 0 \\
    & (\ell_a - \ell_b) p_a - (\ell_a - \ell_b) p_b \le 0 \\
    & (\ell_a - \ell_b)(p_a - p_b) \le 0\text{,}
  \end{align*}
  che è sempre vero perché \(p_a \le p_b \Rightarrow \ell_b \le \ell_a\).
\end{proof}

    \chapter{Applicazioni di LZ77}

\begin{problem*}
  Sfruttando le caratteristiche dell'algoritmo \LZuno di Lempel e Ziv per
  suddividere un testo in una sequenza di frasi:
  \begin{enumerate}[(a)]
    \item mostrare come utilizzare \LZuno per nascondere dei bit all'interno 
    del file compresso risultante senza aumentarne la dimensione (nel caso
    in cui questo sia possibile)
    \item utilizzare il suffix tree (per costruire \LZuno) per trovare la pi\`u 
    lunga sottostringa che si ripete, ossia che appare almeno due volte nel 
    testo.
  \end{enumerate}
\end{problem*}

\begin{enumerate}[(a)]
  \item Nel corso della compressione di un file con l'algoritmo \LZuno pu\`o
  capitare che sia possibile scegliere fra pi\`u prefissi identici a posizioni
  diverse. La scelta di un prefisso al posto di un altro genera qualche bit
  di informazione, che pu\`o essere usato per codificare un messaggio segreto.
  Supponiamo che Alice voglia inviare un messaggio nascosto a Bob tramite la
  compressione della stringa \texttt{\string"abracadabra\$\string"}.
  Raffiguriamo in tabella i primi passi dell'esecuzione dell'algoritmo:
  \begin{table}[H]
    \centering
    \begin{tabularx}{5cm}{*{3}{X}}
      \texttt{a} & \(\rightarrow\) & (\textbf{0},\textbf{0},\textbf{a})\ \\
      \texttt{b} & \(\rightarrow\) & (\textbf{0},\textbf{0},\textbf{b})\ \\
      \texttt{r} & \(\rightarrow\) & (\textbf{0},\textbf{0},\textbf{r})\ \\
      \texttt{a} & \(\rightarrow\) & (\textbf{3},\textbf{1},\textbf{c})\ \\
    \end{tabularx}
  \end{table}
  
  Per comprimere la successiva \texttt{a} Alice potrebbe scegliere di tornare
  indietro di \(5\) caratteri oppure di \(2\). La scelta di una delle due
  alternative comporta un bit di informazione, che Bob pu\`o ricostruire in
  fase di decodifica.

  \item Chiamiamo \emph{suffix tree} di una stringa la trie compatta dei suoi
  suffissi. Ad esempio il suffix tree della stringa 
  \texttt{\string"abracadabra\$\string"} \`e: (TODO)
  \begin{figure}[H]
    \centering
    \begin{tikzpicture}
      \node (_) at (-3,0) [] {};
      \node (a_) at (1,0) [] {};
      \node (abra_) at (2,0) [] {};
      \node (abracadabra_) at (3,0) [] {};
      \node (acadabra_) at (4,0) [] {};
      \node (adabra_) at (5,0) [] {};
      \node (bra_) at (6,0) [] {};
      \node (bracadabra_) at (7,0) [] {};
      \node (cadabra_) at (8,0) [] {};
      \node (dabra_) at (9,0) [] {};
      \node (ra_) at (10,0) [] {};
      \node (racadabra_) at (11,0) [] {};
      
      \node (abra) at (2.5, 1) [] {};
      
      \node (a) at (2.5, 3) [] {};
      \node (bra) at (6.5, 3) [] {};
      \node (ra) at (10.5, 3) [] {};
      
      \node (root) at (7.5, 7) [] {};
      
      \draw (abra_) -- (abra) [];
      \draw (abracadabra_) -- (abra) [];
      
      \draw (a) -- (a_) [];
      \draw (a) -- (abra) [];
      \draw (a) -- (acadabra_) [];
      \draw (a) -- (adabra_) [];
      
      \draw (bra) -- (bra_) [];
      \draw (bra) -- (bracadabra_) [];
      
      \draw (ra) -- (ra_) [];
      \draw (ra) -- (racadabra_) [];
      
      \draw (root) -- (_) [];
      \draw (root) -- (a) [];
      \draw (root) -- (bra) [];
      \draw (root) -- (cadabra_) [];
      \draw (root) -- (dabra_) [];
      \draw (root) -- (ra);
    \end{tikzpicture}
  \end{figure}
  
  Chiamiamo \emph{profondit\`a} di un nodo la distanza di tale nodo dalla 
  radice, misurata per\`o nel numero di caratteri totali delle etichette
  del cammino dalla radice al nodo. Per esempio, nel precedente albero, la 
  profondit\`a del figlio pi\`u a destra della radice \`e \(2\).
  
  Affermiamo allora che la lunghezza della massima sottostringa ripetuta \`e
  la profondit\`a massima di un nodo interno. Infatti perché un nodo sia
  interno \`e necessario che esistano almeno due sottostringhe i cui suffissi
  coincidano per qualche carattere, mentre la condizione di massimalit\`a
  garantisce che si tratti della massima sottostringa ripetuta.
  
  Dato il suffix tree possiamo visitare in qualche ordine i suoi vertici ed
  annotare per ciascuno di essi la profondit\`a. Nella stessa visita possiamo 
  mantenere il nodo interno che realizzi la profondit\`a massima per
  restituirlo a fine visita.
\end{enumerate}

    \chapter{Dizionario di LZ78}

\begin{problem*}
  Progettare una struttura dati per memorizzare e interrogare velocemente
  il dizionario delle frasi ottenute incrementalmente con l'algoritmo
  LZ78. Valutare il costo delle soluzioni proposte.
\end{problem*}


    \chapter{Approssimazione per MIN-VC}

\begin{problem*}
    Il problema del MIN-VC (minimum vertex-cover) richiede,
    per un grafo $G = (V, E)$, di trovare un sottoinsieme $S \subseteq V$ di cardinalità
    minima tale che ogni arco sia incidente ad almeno un vertice di S, cioè
    per ogni $(u, v)\in  E$ vale $u \in S$ oppure $v \in S$. Mostrare come il seguente approccio 
    greedy fornisca una 2-approssimazione: inizializza $S$ all’insieme vuoto e, per ogni arco 
    $(u, v) \in E$, se $u$ e $v$ non sono entrambi marcati, allora marcali e aggiungili a 
    $S := S \cup \{u, v\}$; altrimenti, scarta l’arco.
\end {problem*}

Siano $A$ e $C$ gli insiemi rispettivamente di vertici e di archi selezionati dall'algoritmo; sia inoltre $C^*$ l'insieme di vertici nella soluzione ottima.\newline
Chiaramente $C^*$ deve coprire gli archi di $A$, dunque deve contenere almeno un vertice per ogni arco in $A$; per di più due qualsiasi archi di $A$ non hanno vertici in comune, pertanto ciascun vertice di $C^*$ copre al più un arco di $A$, da cui \[|C^*|\ge|A|.\]
Inoltre il nostro algoritmo seleziona un arco solo se nessuno dei due vertici sta già in $C$, dunque per ogni arco inserito in $A$ aggiunge due vertici a $C$, ossia \[|C|=2|A|.\]
Unendo questi due risultati abbiamo che \[|C| = 2|A| \le 2|C^*|,\] come volevasi dimostrare.

    \chapter{Approssimazione per MAX-SAT}

\begin{problem*}
    Per il problema MAX-SAT della soddisfacibilità di una formula 
    booleana, si consideri il seguente algoritmo di approssimazione 
    per massimizzare il numero di clausole soddisfatte in una data 
    formula: Sia $F$ la formula data, $x_1 , x_2 , \dots , x_n$ le
    variabili booleane in essa contenute, e $c_1 , c_2 , \dots c_m$
    le sue clausole. Scegli i valori booleani casuali $b_1 , b_2 , 
    \dots, b_n$, ossia ciascun $b_i \in \{0, 1\}~(1 \le i \le n)$. 
    Calcola il numero $m_0$ di clausole soddisfatte dall’assegnamento 
    tale che $x_i := b_i~ (1 \le i \le n)$. Calcola il numero $m_1$ di
    clausole soddisfatte dall’assegnamento tale che $x_i := \bar{b_i}~
    (1 \le i \le n)$, dove $\bar{b_i}$ indica la negazione di $b_i$. 
    Se $m_0 > m_1$, restituisci l’assegnamento $x_i := b_i ~(1 \le i 
    \le n)$; altrimenti, restituisci l’assegnamento $x_i := \bar{b_i} 
    ~(1 \le i \le n)$. Dimostrare che il suddetto algoritmo è una 
    $r$-approssimazione per MAX-SAT, indicando anche il valore di $r > 1$
    (e motivando l’utilizzo di tale valore). Discutere se, in generale, 
    la scelta di $b1 , b2 , \dots , bn$ possa influenzare o meno il valore 
    di $r$, motivando le argomentazioni addotte. Facoltativo: creare 
    un’istanza di MAX-SAT in cui il suddetto algoritmo ottiene un costo che 
    è r volte più piccolo del costo ottimo per una data scelta dei valori 
    di $b_1 , b_2 \dots , b_n$.
\end{problem*}

    \chapter{Approssimazione per MAX-CUT}

\begin{problem*}
    Il problema MAX-CUT è NP-hard ed è definito come segue per 
    un grafo non orientato $G = (V, E)$. Una partizione di nodi $(C, V - C)$
    con $C \in V$ si chiama “cut” o taglio. Un arco $e = (v, w)$ con $v \in C$
    e $w \in V - C$ si chiama arco di taglio (ricordando che $(v, w)$ e $(w, v)$
    denotano lo stesso arco in un grafo non orientato). Il numero di archi di 
    taglio definisce la dimensione del cut $(C, V - C)$. Poiché cambiando taglio, 
    può cambiare la sua dimensione, il problema richiede di trovare il taglio di 
    dimensione massima e quindi gli archi di taglio corrispondenti. Dimostrare che 
    il seguente algoritmo randomizzato è una 2-approssimazione in valore atteso, 
    ossia che il numero medio di archi di taglio così individuati è in media almeno
    la metà di quelli del taglio massimo. (1) Per ogni nodo $v \in V$ , lancia una 
    moneta equiprobabile: se viene testa, inserisci $v$ in $C$; altrimenti (viene
    croce), inserisci $v$ in $V - C$. (2) Inizializza $T$ all’insieme vuoto. Per 
    ogni arco $(v, w) \in E$, tale che $v \in C$ e $w \in V - C$, aggiungi $(v, w)$
    all’insieme $T$ . Restituisci $C$ e $T$ come soluzione approssimata.
\end{problem*} 

\end{document}
