\documentclass[a4paper, 12pt]{book}

\usepackage[italian]{babel}
\usepackage[utf8]{inputenc}

% Chapters are now called problems.
\addto\captionsitalian{
    \renewcommand\chaptername{Problema}}

\usepackage{enumerate}
\usepackage{epigraph}
\usepackage{fullpage}

\usepackage{amsmath}
\usepackage{amsthm}

\usepackage{tikz}

% Styles used in a drawing in 04.tex
\tikzstyle{el}=[circle,draw=black!100,fill]
\tikzstyle{red_el}=[circle,draw=red!100,fill=red!100]
\tikzstyle{green_el}=[circle,draw=green!100,fill=green!100]

\usepackage{algorithm}
%\usepackage{algorithmicx}
\usepackage[noend]{algpseudocode}

% Hacks needed to get the algorithm package to work.
\renewcommand{\algorithmiccomment}[1]{/\(\!\)/ #1}
\floatname{algorithm}{Algoritmo}

\theoremstyle{plain}
\newtheorem{definition}{Definizione}
\newtheorem*{definition*}{Definizione}
\newtheorem{problem}{Problema}
\newtheorem*{problem*}{Problema}
\newtheorem{lemma}{Lemma}
\newtheorem*{proof*}{Dimostrazione}

\newcommand{\quicksort}{\mbox{\sc Quicksort }}
\newcommand{\mergesort}{\mbox{\sc Merge Sort }}
\newcommand{\vectornorm}[1]{\left|\left|#1\right|\right|}

\begin{document}
    \title{Problemi di Algoritmica 2}
    \author{
        Alessandro Ambrosano
        \and
        Jacopo Notarstefano
        \and
        Francesco Salvatori
    }
    
    \maketitle

    \epigraph{If you're having girl problems \\
          I feel bad for you son \\
          I got 99 problems \\
          but a bitch ain't one.}{Jay-Z}
    \chapter{Ordinamento in memoria esterna}

\begin{problem*}
    Nel modello EMM (external memory model), mostrate come implementare il
    \(k\)-way merge, ossia la fusione di \(n\) sequenze individualmente
    ordinate e di lunghezza totale \(N\), con costo I/O di \(O(\frac{N}{B})\)
    dove \(B\) \`e la dimensione del blocco. Minimizzare e valutare il costo
    di CPU. Analizzare il costo del merge (I/O complexity, CPU complexity)
    che utilizza tale \(k\)-way merge.
\end{problem*}

    \chapter{Limite inferiore per la permutazione}

\begin{problem*}
    Estendere l'argomentazione usata per il limite inferiore del problema 
    dell'ordinamento in memoria esterna a quello della permutazione: dati
    \(N\) elementi \(e_1,e_2,\dots ,e_N\) e un array \(\pi\) contenente
    una permutazione degli interi in \([1,2,\dots ,N]\), disporre gli
    elementi secondo la permutazione in \(\pi\). Dopo tale operazione,
    la memoria esterna deve contenerli nell'ordine
    \(e_{\pi[1]},e_{\pi[2]},\dots ,e_{\pi[N]}\).
\end{problem*}

    \chapter{Permutazione in memoria esterna}

\begin{problem*}
    Dati tre array \(A, C\) e \(D\) di \(N\) elementi, dove \(A\) \`e l'input e 
    \(C\) una permutazione di \(\left\{0,1,\dots ,n-1\right\}\), descrivere e
    analizzare nel modello EMM un algoritmo ottimo per costruire
    \(D[i]=A[C[i]]\) per \(0\le i\le n-1\).
\end{problem*}

Dall'analisi svolta sul Lower Bound della permutazione, sappiamo che il costo di
I/O è dato da \(\min\{n, \sort(n)\}\), dove \(\sort(n) =
O\left(\frac{N}{B}log_\frac{N}{M} \frac{N}{B}\right)\).
 
\begin{algorithm}
    \caption{Permutazione in memoria esterna}
    \begin{algorithmic}[1]
        \State Calcola il minimo fra \(\{n, \sort(n)\}\)
        \State Se il minimo è \(n\) si utilizza un algoritmo banale che semplicemente esegue un'iterazione sugli elementi e scrive la permutazione.
        \State Nel caso opposto invece si utilizza tecnica \mapreduce.
    \end{algorithmic}	
\end{algorithm}

Si può notare che l'algoritmo banale è eseguito solo in poche circostanze,
ad esempio quando la dimensione \(B\) del blocco è molto piccola. Di seguito
è mostrato soltanto l'algoritmo con tecnica \mapreduce.
 
\begin{algorithm}
    \caption{Permutazione in memoria esterna con tecnica \mapreduce}
    \begin{algorithmic}[1]
        \State Creare le coppie \(\langle i,C[i] \rangle\), ovvero le coppie che mettono in relazione una posizione del vettore \(C\) con l'indice di permutazione.
        \State Ordina le coppie in base alla seconda componente (l'indice di permutazione).
        \State Sostituisci nelle coppie \(C[i]\) con \(A[C[i]]\).
        \State Ordina le coppie per la prima componente.
        \State Scrivi nel vettore \(D\) le seconde componenti delle coppie.
    \end{algorithmic}	
\end{algorithm}

Analizziamo il costo dei singoli passi dell'algoritmo:
\begin{itemize}
    \item Il \emph{primo} passo dell'algoritmo esegue una lettura di \(N\) elementi
    e una scrittura di \(N\) coppie. Da qui si deduce che il costo di I/O è
    \(O\left(\frac{N}{B}\right)\).
    \item Il costo di I/O del \emph{secondo} passo è uguale a quello di un
    ordinamento che, utilizzando ad esempio il \(k\)-way merge-sort è
    \(O\left(\frac{N}{B}log_\frac{N}{M} \frac{N}{B}\right)\).
    \item Nel \emph{terzo} passo, poiché le coppie sono ordinate per \(C[i]\), si accede
    al vettore \(A\) in modo sequenziale, quindi il costo di I/O è \(O\left(\frac{N}{B}\right)\).
    \item Anche per il \emph{quarto} il costo di I/O è
    \(O\left(\frac{N}{B}log_\frac{N}{M} \frac{N}{B}\right)\).
    \item Per l'\emph{ultimo} passo il costo di I/O è \(O\left(\frac{N}{B}\right)\) in quanto
    sono letti e scritti \(N\) sequenzialmente.
\end{itemize}

In conclusione il costo di I/O dell'algoritmo appena esposto è
\(O\left(\frac{N}{B}log_\frac{N}{M} \frac{N}{B}\right)\).
    \chapter{Multi-selezione in memoria esterna}    

\begin{problem*}
    Scrivere tutti i passaggi dell'analisi del costo e della correttezza
    dell'algoritmo di multi-selezione visto a lezione.
\end{problem*}

Vogliamo esibire un algoritmo che selezioni un certo numero di pivot da un
insieme \(S\) di cardinalit\`a \(N\) in modo tale che la distanza fra pivot
consecutivi sia piccola. Il nostro scopo sar\`a usare questo algoritmo per
costruire un analogo del \quicksort in memoria esterna, cos\`i come la \(k\)
-way merge ci ha permesso di costruire l'algoritmo di \mergesort in 
memoria esterna.

Ci potremmo aspettare di dover trovare \(m\) pivot, in analogia a quanto
facciamo per la Merge. In realt\`a \`e sufficiente determinarne \(\sqrt{m}\).
Diamo di seguito l'algoritmo e due lemmi. Nel primo dimostreremo il costo
lineare, nel secondo la correttezza dell'algoritmo.

\begin{algorithm}
    \caption{Multi-selezione in memoria esterna}
    \begin{algorithmic}[1]
        \State Carico e ordino in memoria principale \(\frac{N}{M}\) run
        di \(M\) elementi ciascuno.
        \State Da ogni run seleziono un elemento ogni 
        \(\frac{\sqrt{m}}{4}\) e chiamo \(G\) (elementi verdi) l'insieme 
        degli elementi selezionati.
        \State Uso l'algoritmo dei cinque autori \(\sqrt{m}\) volte per
        selezionare in \(G\) un elemento ogni \(\frac{4N}{m}\) e chiamo 
        \(R\) (elementi rossi) l'insieme degli elementi selezionati.
        \State Ritorno \(R\).
    \end{algorithmic}
\end{algorithm}

\begin{lemma}[Costo]
    L'algoritmo compie \(O(n)\) I/O.
\end{lemma}
\begin{proof*}
    La prima riga dell'algoritmo comporta soltanto di scandire tutti gli
    elementi: l'ordinamento di ogni run viene infatti svolto in memoria
    principale, e non comporta ulteriori I/O. Anche la seconda riga
    consiste in una scansione di tutti gli elementi. Per stimare il numero
    di I/O della terza riga sfruttiamo invece il fatto che ogni esecuzione
    dell'algoritmo dei cinque autori comporta una scansione di tutti gli
    elementi. Abbiamo dunque \(\sqrt{m}\) scansioni di \(|G|\) elementi,
    perci\`o:
    \[
        \sqrt{m}\cdot O\left(\frac{|G|}{B}\right) = \sqrt{m}\cdot O\left(\frac{4N}{B\sqrt{m}}\right) = O\left(\frac{4N}{B}\right) = O(n)\mbox{,}
    \]
    dove la prima eguaglianza discende dal fatto che, avendo selezionato
    un elemento ogni \(\frac{\sqrt{m}}{4}\), la cardinalit\`a di \(G\) \`e
    \(\frac{4N}{\sqrt{m}}\). Ogni riga contribuisce quindi \(O(n)\) I/O, da cui la tesi.
\end{proof*}

\begin{lemma}[Correttezza]
    Il numero di elementi di \(S\) compresi fra due elementi di \(R\) \`e
    minore di \(\frac{3}{2}\frac{N}{\sqrt{m}}\).
\end{lemma}
\begin{proof*}
    Vogliamo dunque stimare il numero di elementi di \(S\) compresi fra
    due generici elementi rossi \(r_1\) e \(r_2\). Possiamo dividerli in
    tre categorie:
    \begin{itemize}
        \item Gli elementi \emph{verdi} compresi fra i due elementi rossi \(r_1\) e \(r_2\).
        \item Gli elementi \emph{senza colore} compresi fra due elementi verdi.
        \item Gli elementi \emph{senza colore} compresi fra un rosso e un verde.
    \end{itemize}
        
    I primi sono facilmente maggiorati da \(X = \frac{4N}{m}\): nella 
    terza riga dell'algoritmo abbiamo infatti scelto un rosso ogni 
    \(\frac{4N}{m}\) elementi verdi.
        
    I secondi sono invece maggiorati da \(Y = \frac{N}{\sqrt{m}} - \frac{4N}{m}\). Per la seconda riga dell'algoritmo abbiamo infatti 
    \(\frac{\sqrt{m}}{4}-1\) elementi senza colore fra due verdi
    consecutivi appartenenti allo stesso run, avendo scelto un verde ogni
    \(\frac{\sqrt{m}}{4}\) elementi. Per la terza riga abbiamo al pi\`u
    \(\frac{4N}{m}\) elementi verdi fra \(r_1\) e \(r_2\), dunque al
    pi\`u \(\frac{4N}{m}\) coppie di elementi verdi consecutivi nello
    stesso run. Ma allora il numero cercato \`e stimato dal prodotto, e 
    quindi da:
    \[
        \frac{4N}{m}\left(\frac{\sqrt{m}}{4}-1\right) = \frac{N}{\sqrt{m}} - \frac{4N}{m}\mbox{.}
    \]
        
    I terzi sono invece maggiorati da \(Z = \frac{n}{2\sqrt{m}} - \frac{2n}{m}\). Per vedere questo abbiamo bisogno della figura.
    \begin{figure}
        \centering
        \begin{tikzpicture}
            \node (l_boundary_down) at (3, -0.5) [] {};
            \node (l_boundary_up) at (3, 4.5) [] {};
            \draw (l_boundary_down) -- (l_boundary_up) [thick];
                
            \node (r_boundary_down) at (8, -0.5) [] {};
            \node (r_boundary_up) at (8, 4.5) [] {};
            \draw (r_boundary_down) -- (r_boundary_up) [thick];
                
            \fill [fill=yellow!50] (3, -0.5) -- (3, 4.5) -- (4.25, 4.5) -- (4.25, -0.5) -- cycle;             
            \fill [fill=yellow!50] (8, -0.5) -- (8, 4.5) -- (6.75, 4.5) -- (6.75, -0.5) -- cycle;
                
            \node (run_1_begin) at (0,0) [] {};
            \node (run_1_end) at (10,0) [] {};
                
            \draw (run_1_begin) -- (run_1_end) [];
            \node (el_1_1) at (1,0) [green_el] {};
            \node (el_1_2) at (1.5,0) [el] {};
            \node (el_1_3) at (2.5,0) [el] {};
            \node (el_1_4) at (4,0) [green_el] {};
            \node (el_1_5) at (5.5,0) [el] {};
            \node (el_1_6) at (7.5,0) [el] {};
            \node (el_1_7) at (8.5,0) [green_el] {};
            \node (el_1_8) at (9,0) [el] {};
            \node (el_1_9) at (9.5,0) [el] {};
                
            \node (run_2_begin) at (0,1) [] {};
            \node (run_2_end) at (10,1) [] {};
                
            \draw (run_2_begin) -- (run_2_end) [];
            \node (el_2_1) at (0.5,1) [green_el] {};
            \node (el_2_2) at (1,1) [el] {};
            \node (el_2_3) at (2.5,1) [el] {};
            \node (r_1) at (3,1) [red_el] {};
            \node (el_2_5) at (3.5,1) [el] {};
            \node (el_2_6) at (4,1) [el] {};
            \node (el_2_7) at (7.5,1) [green_el] {};
            \node (el_2_8) at (8.5,1) [el] {};
            \node (el_2_9) at (9,1) [el] {};
                
            \node (run_3_begin) at (0,2) [] {};
            \node (run_3_end) at (10,2) [] {};
                
            \draw (run_3_begin) -- (run_3_end) [];
            \node (el_3_1) at (0.5,2) [green_el] {};
            \node (el_3_2) at (1.5,2) [el] {};
            \node (el_3_3) at (2,2) [el] {};
            \node (el_3_4) at (3.5,2) [green_el] {};
            \node (el_3_5) at (5,2) [el] {};
            \node (el_3_6) at (5.5,2) [el] {};
            \node (el_3_7) at (7.5,2) [green_el] {};
            \node (el_3_8) at (9,2) [el] {};
            \node (el_3_9) at (9.5,2) [el] {};
                
            \node (run_4_begin) at (0,3) [] {};
            \node (run_4_end) at (10,3) [] {};
                
            \draw (run_4_begin) -- (run_4_end) [];
            \node (el_4_1) at (1,3) [green_el] {};
            \node (el_4_2) at (1.5,3) [el] {};
            \node (el_4_3) at (2.5,3) [el] {};
            \node (el_4_4) at (4.5,3) [green_el] {};
            \node (el_4_5) at (7,3) [el] {};
            \node (el_4_6) at (7.5,3) [el] {};
            \node (r_2) at (8,3) [red_el] {};
            \node (el_4_8) at (8.5,3) [el] {};
            \node (el_4_9) at (9.5,3) [el] {};
                
            \node (run_5_begin) at (0,4) [] {};
            \node (run_5_end) at (10,4) [] {};
                
            \draw (run_5_begin) -- (run_5_end) [];
            \node (el_5_1) at (1.5,4) [green_el] {};
            \node (el_5_2) at (2,4) [el] {};
            \node (el_5_3) at (2.5,4) [el] {};
            \node (el_5_4) at (4,4) [green_el] {};
            \node (el_5_5) at (4.5,4) [el] {};
            \node (el_5_6) at (6.5,4) [el] {};
            \node (el_5_7) at (7,4) [green_el] {};
            \node (el_5_8) at (7.5,4) [el] {};
            \node (el_5_9) at (9,4) [el] {};
        \end{tikzpicture}
        \caption{Ogni riga orizzontale rappresenta un run ordinato, e i 
        cerchietti gli elementi di ogni run. Cerchietti rossi e verdi 
        rappresentano rispettivamente gli elementi di \(R\) e \(G\), 
        cerchietti neri i restanti elementi senza colore. Sono inoltre 
        raffigurati in giallo i bordi definiti dalla posizione degli 
        elementi rossi nei quali possiamo avere elementi senza colore 
        compresi fra un rosso e un verde.}
    \end{figure}
    Osserviamo infatti che la posizione dei due elementi rossi definisce
    un paio di ``bordi'' (raffigurati in giallo in figura) in cui \`e
    possibile trovare gli elementi del terzo tipo. Ognuno di questi 
    bordi può contenere al pi\`u \(\frac{\sqrt{m}}{4}-1\) elementi,
    perché se ne contenesse di pi\`u conterrebbe sicuramente due verdi,
    quindi elementi compresi fra due verdi. Per ognuno degli 
    \(\frac{N}{M}\) run abbiamo insomma \(2\) bordi contenenti al pi\`u 
    \(\frac{\sqrt{m}}{4}-1\) elementi, perci\`o possiamo stimare il
    numero di elementi del terzo tipo con il loro prodotto, cio\`e:
    \[
        2 \cdot \frac{n}{m} \cdot \left( \frac{\sqrt{m}}{4}-1 \right) = \frac{n}{2\sqrt{m}} - \frac{2n}{m}\mbox{,}
    \]
    dove abbiamo usato che \(\frac{N}{M}\) = \(\frac{n}{m}\) essendo
    \(n = \frac{N}{B}\) e \(m = \frac{M}{B}\).
        
    Di conseguenza il numero totale degli elementi compresi fra \(r_1\) 
    e \(r_2\) è maggiorato da:
    \begin{align}
        X + Y + Z &= \frac{4N}{m} + \frac{N}{\sqrt{m}} - \frac{4N}{m} + \frac{n}{2\sqrt{m}} - \frac{2n}{m} \nonumber \\
        &\le \frac{N}{\sqrt{m}} + \frac{n}{2\sqrt{m}} \nonumber
    \end{align}
    nella quale abbiamo cancellato i termini uguali di segno opposto e un
    termine negativo, ottenendo un'ulteriore maggiorazione. Abbiamo
    inoltre:
    \begin{align}
        \frac{N}{\sqrt{m}} + \frac{n}{2\sqrt{m}} &\le \frac{N}{\sqrt{m}} + \frac{N}{2\sqrt{m}} \nonumber \\
        &= \frac{3}{2}\frac{N}{\sqrt{m}} \nonumber
    \end{align}
    essendo \(n = \frac{N}{B}\), da cui la tesi.
\end{proof*}
    \chapter{MapReduce}

\begin{problem*}
    Utilizzare il paradigma Scan \& Sort mediante la MapReduce per calcolare la
    distribuzione dei gradi in ingresso delle pagine Web. In particolare, specificare
    quanti passi di tipo MapReduce sono necessari e quali sono le funzioni Map e
    Reduce impiegate. Ipotizzare di avere gi\`a tali pagine a disposizione.
\end{problem*}


    \chapter{Navigazione implicita in vEB}

\begin{problem*}
    Dato un albero completo memorizzato secondo il layout di van Emde Boas (vEB) in
    modo implicito, ossia senza l'ausilio di puntatori (come succede nello heap binario
    implicito), trovare la regola per navigare in tale albero senza usare puntatori
    espliciti.
\end{problem*}


    \chapter{Layout di alberi binari}

\begin{problem*}
    Proporre una paginazione di alberi binari in blocchi di dimensione \(B\) per
    realizzare un loro layout in memoria esterna: valutare se un qualunque cammino
    minimo radice-nodo di lunghezza \(l\) attraversa sempre \(O(\frac{l}{\log{B}})\)
    pagine. NOTA: per chi vuole, esiste una versione pi\`u impegnativa di questo
    esercizio, basta contattarmi per averla.
\end{problem*}


    \chapter{Suffix array in memoria esterna}

\begin{problem*}
    Utilizzando la costruzione del suffix array basata sul \mergesort e la tecnica
    DC3 vista a lezione, progettare un algoritmo per EMM per costruire il suffix
    array di un testo che abbia la stessa complessit\`a del \mergesort in EMM.
\end{problem*}


    \chapter{Famiglia di funzioni hash uniformi}

\begin{problem*}
    Mostrare che la famiglia di funzioni hash 
    \(H = \left\{h(x) = ((ax + b ) \mod p) \mod m\right\}\) \`e (quasi) uniforme, dove 
    \(a,b\in [m]\) con \(a\neq 0\) e \(p\) \`e un numero primo sufficientemente grande.
\end{problem*}
Siano $k\neq l$ due chiavi, $h\in H$ e $r=ak+b \pmod p$, $s=al+b\pmod p$. Abbiamo che $r-s = a(k-l) \pmod p$ da cui, essendo $a\neq0$ e, per $p$ sufficientemente grande, $(k-l)\neq 0 \pmod p$, segue che $r-s\neq0$.\newline
È dunque possibile una collisione tra gli hash delle due chiavi se e solo se $r \pmod m = s \pmod m$.\newline
Notiamo che per $a$ sono possibili $p-1$ valori (lo $0$ non va bene) mentre per $b$ ve ne sono $p$: in tutto abbiamo $p(p-1)$ possibili coppie $(a, b)$; ma anche per $(r, s)$ vi sono $p(p-1)$ possibilità (fissato $r$, per $s$ vanno bene tutti i valori tranne quello assunto da $r$): segue che le coppie $(a, b)$ e le coppie $(r, s)$ possono essere messe in corrispondenza biunivoca. Inoltre se la coppia $(a, b)$ è scelta a caso con probabilità uniforme ciascuna coppia $(r, s)$ ha la stessa probabilità di essere l'immagine di due date chiavi $l$ e $k$. Questo significa che \[\Pr[~h(k)=h(l)~]=\Pr[~r = s \pmod m].\]
Fissiamo $r$: i possibili valori per $s$ che originano una "collisione" sono al più \[\left\lceil \frac{p}{m} \right\rceil-1 \le \frac{p+m-1}{m} - 1= \frac{p-1}{m}.\]
Da questo segue che \[\Pr[~r=s\pmod m] \le \frac{\frac{p-1}{m}}{p-1}=\frac{1}{m},\] il che significa \[\Pr[h(k)=h(l)]\le\frac{1}{m},\] come volevasi dimostrare.


    \chapter{Count-min sketch: estensione}

\begin{problem*}
    Estendere l'analisi vista a lezione permettendo di incrementare e decrementare
    i contatori con valori arbitrari.
\end{problem*}

\begin{lemma}[Approssimazione]
Per il count min sketch a valori arbitrari vale:

\[
    Pr\left[F[i] - 3\varepsilon\vectornorm{F} \le \tilde{F}[i]
        \le F[i] + 3\varepsilon\vectornorm{F}\right] \ge 1 - \delta^{\frac{1}{4}}
\]

con $\tilde{F}[i] = median_{\substack{j}}\{T[j, h_j(i)]\}$.
\end{lemma}

\begin{proof*}
    Osserviamo innanzi tutto che $\tilde{F}[i] = T[\hat{i}, h_{\hat{i}}(i)]$
    per qualche $\hat{i}$, e vale
    \[
        \tilde{F}[i] = F[i] + \sum_{k = 1}^{n}I_{\hat{i}, i, k}F[k]
    \]
    dove $I_{j, i, k}$ è la variabile indicatrice così definita:
    \[
        I_{j, i, k} =
        \begin{cases}
            1 & \mbox{se } i \neq k \land h_j(i)=h_j(k) \\
            0 & \mbox{altrimenti}
        \end{cases}
    \]

    Ponendo $X_{j, i} = \sum_{k=1}^n I_{j, i, k} F[k]$ abbiamo:
    
    \begin{align*}
        & Pr\left[F[i] - 3\varepsilon\vectornorm{F} \le \tilde{F}[i]
            \le F[i] + 3\varepsilon\vectornorm{F}\right] \nonumber \\
        &= Pr\left[F[i] - 3\varepsilon\vectornorm{F}
            \le F[i] + \sum_{k = 1}^{n}I_{\hat{i}, i, k}F[k]
            \le F[i] + 3\varepsilon\vectornorm{F}\right] \nonumber \\
        &= Pr\left[F[i] - 3\varepsilon\vectornorm{F}
            \le F[i] + X_{\hat{i},i}
            \le F[i] + 3\varepsilon\vectornorm{F}\right] \nonumber \\
        &= Pr\left[- 3\varepsilon\vectornorm{F}
            \le X_{\hat{i},i}
            \le 3\varepsilon\vectornorm{F}\right] \nonumber \\
        &= Pr\left[|X_{\hat{i},i}| \le 3\varepsilon\vectornorm{F}\right] \nonumber
    \end{align*}

    \begin{align*}
        E[I_{j, i, k}] &= Pr\left[i \neq k \land h_j(i) = h_j(k)\right] \\ 
        &= Pr\left[\cup_{a=1}^c i \neq k \land h_j(i) = a \land h_j(k) = a \right] \\
        &= \sum_{a=1}^c \underbrace{Pr\left[i \neq k \land h_j(i) = a \land h_j(k) = a \right]}_
            {\mbox{Sono funzioni di hash uniformi}}\\
        &\le \sum_{a=1}^c \frac{1}{c} = \frac{c}{c^2} = \frac{1}{c} = \frac{\varepsilon}{e}
    \end{align*}

    da cui
    \begin{align*}
        E[|X_{j, i}|] &= E\left[|\sum_{k=1}^n I_{j, i, k} F[k]|\right] \\
        &\le \sum_{k=1}^n E[|I_{j, i, k}|] |F[k]| 
        = \frac{\varepsilon}{e} \sum_{k=1}^n |F[k]|]
        = \frac{\varepsilon}{e} \vectornorm{F}
    \end{align*}

    e quindi per ogni $j$, usando la disuguaglianza di Markov
    \[
        Pr\left[|X_{j, i}| \ge 3\varepsilon\vectornorm{F}\right] \le
            \frac{E[|X_{j,i}|]}{3\varepsilon\vectornorm{F}} 
        = \frac{\frac{\varepsilon}{e} \vectornorm{F}}{3\varepsilon\vectornorm{F}}
        = \frac{1}{3e}
    \]

    Notiamo che perché valga $|X_{\hat{i}, i}| \ge 3\varepsilon\vectornorm{F}$,
    deve valere $|X_{j, i}| \ge 3\varepsilon\vectornorm{F}$, per questo scopo
    introduciamo le seguenti variabili casuali:

    \[
        Y_j = 
        \begin{cases}
            1 & \mbox{se } |X_{j, i}| \ge 3\varepsilon\vectornorm{F} \\
            0 & \mbox{altrimenti}
        \end{cases}
    \]

    Se poniamo $p = E[Y_j] = Pr[|X_{j,i}| \ge 3 \varepsilon\vectornorm{F}|]$,
    e $Y = Y_1 + \dots + Y_r$, abbiamo $E[Y] = rp$. \\
    Noi vogliamo $Y \ge \frac{r}{2}$, per calcolarne la probabilità dobbiamo
    introdurre la disuguaglianza di Chernoff.

    \begin{definition*}[Chernoff bound]
        Se $X_1, \dots, X_n$ sono n prove di Poisson indipendenti, identicamente
        distribuite, ossia $\forall i. P[X_i = 1] = p, P[X_i = 0] = 1-p$, se
        prendiamo $X=\sum_{i=1}^n X_i$ e $\mu = E[X]$, per ogni $\lambda > 0$ si ha:
        \[ Pr\left[X \ge (1+\lambda)\mu\right] <
            \left(\frac{e^\lambda}{(1+\lambda)^{1+\lambda}}\right)^\mu\]
    \end{definition*}

    Quindi ponendo $\mu = E[Y] = rp$, $(1+\lambda)\mu = \frac{r}{2} \Rightarrow
    1+\lambda = \frac{1}{2p}$, abbiamo
    \begin{align*}
        Pr\left[Y > \frac{r}{2}\right] &< \left(\frac{e^\lambda}{(1+\lambda)^{1+\lambda}}\right)^\mu
        = \frac{1}{e^\mu}\left(\frac{e}{1+\lambda}\right)^{(1+\lambda)\mu} \\
        &= \frac{1}{e^{rp}}\left(\frac{e}{\frac{1}{2p}}\right)^{\frac{r}{2}}
        = \frac{1}{e^{rp}}(2pe)^{\frac{r}{2}}
    \end{align*}

    Quindi se $\frac{1}{e^{rp}}(2pe)^{\frac{r}{2}} \le \frac{1}{2^{\frac{r}{4}}} (= \delta^{\frac{1}{4}})$
    abbiamo concluso:

    \begin{align*}
        \frac{1}{e^{rp}}(2pe)^{\frac{r}{2}} \le \frac{1}{2^{\frac{r}{4}}}
        & \equiv \underbrace{2^{\frac{r}{4}} \le e^{rp} \frac{1}{(2pe)^{\frac{r}{2}}}}
            _{rp \ge 0 \Rightarrow e^{rp} \ge 1}
        \Leftarrow 2^{\frac{r}{4}} \le \frac{1}{(2pe)^{\frac{r}{2}}} \\
        & \equiv 2^{\frac{1}{2}} \le \frac{1}{2pe}
        \equiv p \le \frac{1}{2\sqrt{2}e}
    \end{align*}

    che vale in quanto $2\sqrt{2}e\sim 7.668$ e $p < \frac{1}{8}$. \qed
\end{proof*}
    \chapter{Count-min sketch: prodotto scalare}

\begin{problem*}
    Mostrare come utilizzare il paradigma del count-min sketch per approssimare il
    prodotto scalare (i.e., approssimare \(\sum_{k=1}^n{F_a[k]\cdot F_b[k]}\)).
\end{problem*}

\begin{lemma}[Approssimazione]
    Per il prodotto scalare, con valori non negativi per $F_a$ e $F_b$ valgono
    \begin{itemize}
        \item $a \cdot b \le \tilde{a \cdot b}$
        \item $Pr[\tilde{a \cdot b} \le a \cdot b
            + \varepsilon \vectornorm{a}\vectornorm{b}] \ge 1 - \delta$
    \end{itemize}
\end{lemma}

con $\displaystyle \tilde{a \cdot b} = \min_j\{\sum_{i=1}^{c}T_a[j,h_j(i)]T_b[j,h_j(i)]\}$.

\begin{proof*}
Osserviamo innanzi tutto che per un certo $\hat{i}$ vale

\[
    \tilde{a \cdot b} = \sum_{i=1}^c T_a[\hat{i}, h_{\hat{i}}(i)] T_b[\hat{i}, h_{\hat{i}}(i)]
\]

e se definiamo la variabile $I_{j, i, k}$ come

\[
    I_{j, i, k} =
    \begin{cases}
        1 & \mbox{se } i \neq k \land h_j(i) = h_j(k) \\
        0 & \mbox{altrimenti}
    \end{cases}
\]

otteniamo la seguente uguaglianza:

\[
    \tilde{a \cdot b} = a \cdot b + \sum_{p, q} I_{\hat{i}, p, q} F_a[p] F_b[q]
\]

\begin{itemize}
\item Non vale ridere perché ho scritto fap
\item Non vale tornare indietro per leggere se l'ho scritto davvero
\end{itemize}

pertanto, vista la non negatività degli elementi di $F_a$ e $F_b$ la prima
disuguaglianza vale banalmente.

Per la seconda, ragioniamo col complementare e cerchiamo di calcolare
\[ Pr[\tilde{a \cdot b} \ge a \cdot b + \varepsilon \vectornorm{a}\vectornorm{b}] \]

Che, se poniamo $X_{j, i} = \sum_{p, q} I_{j, p, q} F_a[p] F_b[q]$ equivale a calcolare

\[ Pr[X_{\hat{i}, i} \ge \varepsilon \vectornorm{a}\vectornorm{b}] \]

Per poterla calcolare ci servono innanzi tutto le speranze di $E[I_{j, i, k}]$
e $E[X_{j,i}]$, date da:

\begin{align}
    E[I_{j, i, k}] &= Pr\left[i \neq k \land h_j(i) = h_j(k)\right] \nonumber \\
    &= Pr\left[\cup_{a=1}^c i \neq k \land h_j(i) = a \land h_j(k) = a \right] \nonumber \\
    &= \sum_{a=1}^c Pr\left[i \neq k \land h_j(i) = a \land h_j(k) = a \right] \nonumber \\
    &= \sum_{a=1}^c Pr\left[h_j(i) = a\right]Pr\left[i \neq k \land h_j(k) = a \right] \nonumber \\
    &= \frac{c}{c^2} = \frac{1}{c} = \frac{\varepsilon}{e} \nonumber
\end{align}

\begin{align}
    E[X_{j, i}] &= E[\sum_{i, k} I_{j, i, k} F_a[i] F_b[k]] \nonumber \\
    &= \sum_{i, k} E[I_{j, i, k}] F_a[i] F_b[k] \nonumber \\
    &= \frac{\varepsilon}{e} \sum_{i, k} F_a[i]F_b[k] \nonumber \\
    &\le \frac{\varepsilon}{e} \vectornorm{a}\vectornorm{b} \nonumber
\end{align}

Si ha quindi, per la disuguaglianza di Markov:

\begin{align}
    Pr[X_{j, i} \ge \varepsilon \vectornorm{a}\vectornorm{b}] &\le
        \frac{E[X_{j, i}]}{\varepsilon \vectornorm{a}\vectornorm{b}} \nonumber \\
        &= \frac{\frac{\varepsilon}{e} \vectornorm{a}\vectornorm{b}}
            {\varepsilon \vectornorm{a}\vectornorm{b}} \nonumber \\
        &= \frac{1}{e} < \frac{1}{2} \nonumber
\end{align}

Osserviamo che affinché valga $X_{\hat{i}, i} \ge \varepsilon \vectornorm{a}\vectornorm{b}$,
è necessario che valga per tutte le $r$ $X_{j, i}$, in quanto $X_{\hat{i}, i}$
è la minima tra queste, trattandosi le $X_{j, i}$ di variabili casuali
indipendenti, abbiamo che:
\[ Pr[X_{\hat{i}, i} \ge \varepsilon \vectornorm{a}\vectornorm{b}] < \frac{1}{2^r} = \delta, \]
da cui
\[ Pr[X_{\hat{i}, i} \le \varepsilon \vectornorm{a}\vectornorm{b}] \ge 1 - \delta. \]
\qed
\end{proof*}
    \chapter{Count-min sketch: interval query}

\begin{problem*}
    Mostrare come utilizzare il paradigma del count-min sketch per rispondere alle
    interval query (i.e., approssimare \(\sum_{k=i}^j{F[k]}\)).
\end{problem*}

Per risolvere questo problema suddividiamo ogni range in modo canonico in
sottorange di dimensione $2^k$, e usiamo $\log n$ count min sketch per
memorizzare separatamente i sottorange di dimensione diversa.

La suddivisione avviene in intervalli di tipo $[x2^k+1,(x+1)2^k]$ (\emph{diadici})
per $k \in [0, \log n-1]$, e ogni volta che ci arriva un dato andiamo ad
incrementare tutti i count min sketch negli intervalli corrispondenti.

\begin{lemma}[Suddivisione canonica]
    Qualunque intervallo può essere suddiviso in al più $2 \log{n}$ intervalli
    diadici.
\end{lemma}

\begin{proof*}
    Sia $[l^*, r^*]$ l'intervallo diadico più grande che può stare dentro il
    nostro intervallo $[l, r]$. Evidentemente vale $|[l^*, r^*]| < n$ e
    quindi l'intervallo diadico è al più della forma
    $[x2^{\log n-1}+1,(x+1)2^{\log n-1}]$, cioè $k < \log{n}$.

    Consideriamo i due intervalli indotti $[l, x2^{k}+1]$ e $[(x+1)2^{k}, r]$,
    ci sono due casi possibili:
    \begin{itemize}
        \item uno dei due ha dimensione $\ge 2^k$, questo non è un assurdo
        perché i due intervalli di dimensione $2^k$ potrebbero essere spostati
        di $2^k$ rispetto all'intervallo diadico canonico più vicino di
        dimensione $2^{k+1}$.

        In questo caso ci limitiamo a togliere l'intervallo di dimensione $2^k$
        o dal fondo del primo intervallo indotto o dall'inizio del secondo
        (è semplice immaginare perché si deve trovare proprio lì), e ci
        riconduciamo al secondo caso.

        \item entrambi gli intervalli indotti hanno dimensione $< 2^k$, questo
        vuol dire che ogni intervallo indotto può contenere al più un intervallo
        di dimensione $2^{k-1}$, altrimenti sarebbe di dimensione $2^k$ e quindi
        assurdo. Inoltre l'intervallo di dimensione $2^{k-1}$ può essere solo in
        fondo al primo intervallo o in testa al secondo (altrimenti potrebbero
        essercene due). Quindi nel caso non si trovasse nessun intervallo di
        dimensione $2^{k-1}$ contenuto vorrebbe dire che lo stesso intervallo
        indotto ha dimensione $< 2^{k-1}$, quindi ripetiamo l'osservazione con
        $k-2$, e così via.
    \end{itemize}

    Il numero totale di intervalli è quindi dato da $2 \log{n}$, in quanto
    per $k_{max}$ ce n'è al più due, in accordo al primo caso, e per valori
    di $k$ minori ce ne può essere al più uno per ogni intervallo indotto,
    siccome andiamo a togliere sempre dalla testa o dalla coda, siccome
    $k_{max} < \log{n}$ abbiamo la tesi.

\end{proof*}

\begin{lemma}[Approssimazione]
    Per la range query con valori non negativi valgono le seguenti proprietà:
    \begin{itemize}
        \item $\mathcal{Q}(l, r) \le \tilde{\mathcal{Q}}(l, r)$
        \item $Pr \left[\tilde{\mathcal{Q}}(l, r) \le \mathcal{Q}(l, r)
            + 2 \varepsilon \log{n} \vectornorm{F}\right] \ge 1-\delta$.
    \end{itemize}
\end{lemma}

\begin{proof*}
    Abbiamo
    \[
        \tilde{\mathcal{Q}}(l, r) = \sum_{k = 0}^{\log{n}-1} \tilde{\mathcal{Q}}(d_{2^k, i_{k, 1}}) +
            \tilde{\mathcal{Q}}(d_{2^k, i_{k, 2}})
    \]
    dove $d_{2^k, i_{k, 1}}$ e $d_{2^k, i_{k, 1}}$ sono i due intervalli
    diadici di dimensione $2^k$ in cui abbiamo scomposto il nostro intervallo
    $[l, r]$, supponendo per tutti gli intervalli non esistenti di avere $d_{a,b} = []$
    e $\mathcal{Q}(d_{a, b}) = 0$.

    Inoltre, per ogni intervallo diadico, dalle proprietà dei count min sketch
    per valori non negativi si ha:
    \[
        \tilde{\mathcal{Q}}(d_{2^k, i}) =
            \mathcal{Q}(d_{2^k, i}) + \sum_{l=1}^{\frac{n}{2^k}} I_{\hat{i}, i, l} \mathcal{Q}(d_{2^k, l})
    \]

    da cui segue la prima disuguaglianza.

    Per la seconda disuguaglianza abbiamo, ponendo $X_{j,i} =
    \sum_{l=1}^{\frac{n}{2^k}} I_{\hat{i}, i, l} \mathcal{Q}(d_{2^k, l})$

    \[
        E[X_{j, i}] \le \frac{\varepsilon}{e} \vectornorm{F}
    \]


    Quindi, su tutti gli intervalli
    \[
        E[\tilde{\mathcal{Q}}(l, r) - \mathcal{Q}(l, r)] \le 2 \frac{\varepsilon}{e} \log{n} \vectornorm{F}
    \]

%    Quindi, per la disuguaglianza di Markov
%    \begin{align}
%        &Pr[\tilde{\mathcal{Q}}(d_{2^k, i}) \ge \mathcal{Q}(d_{2^k, i})
%            + 2 \varepsilon \log{n} \vectornorm{F}] \nonumber \\
%        &= Pr[X_{j, i} \ge 2 \varepsilon \log{n} \vectornorm{F}] \nonumber \\
%        &\le \frac{E[X_{j, i}]}{2 \varepsilon \log{n} \vectornorm{F}} \nonumber \\
%        &\le \frac{2 \frac{\varepsilon}{e} \log{n} \vectornorm{F}}{2 \varepsilon \log{n} \vectornorm{F}} \nonumber \\
%        &= \frac{1}{e} < \frac{1}{2} \nonumber
%    \end{align}

    Da cui, per la disuguaglianza di Markov:
    \begin{align}
        &Pr \left[\tilde{\mathcal{Q}}(l, r) \ge \mathcal{Q}(l, r)
            + 2 \varepsilon \log{n} \vectornorm{F}\right] \nonumber \\
        &=
        Pr \left[\mathcal{Q}(l, r) + \tilde{\mathcal{Q}}(l, r) - \mathcal{Q}(l, r) \ge \mathcal{Q}(l, r)
            + 2 \varepsilon \log{n} \vectornorm{F}\right] \nonumber \\
        &=
        Pr \left[\tilde{\mathcal{Q}}(l, r) - \mathcal{Q}(l, r) \ge
            2 \varepsilon \log{n} \vectornorm{F}\right] \nonumber \\
        &\le
        \left(
            \frac{E[\tilde{\mathcal{Q}}(l, r) - \mathcal{Q}(l, r)]}{2 \varepsilon \log{n} \vectornorm{F}}
        \right) ^ r
            \nonumber \\
        &=
        \left(
            \frac{2 \frac{\varepsilon}{e} \log{n}}{2 \varepsilon \log{n} \vectornorm{F}}
        \right)^r = \left(\frac{1}{e}\right)^r < \left(\frac{1}{2}\right)^r = \delta \nonumber 
    \end{align}
    da cui segue la tesi.

    Attenzione: l'ultimo passo della dimostrazione è falso.

\end{proof*}
    \chapter{Elementi distinti}

\begin{problem*}
    Progettare e analizzare un algoritmo di data streaming che permetta di
    approssimare il numero di elementi distinti.
\end{problem*}

Cominciamo fissando le notazioni usate nel seguito. Sia \(\chi = x_1 x_2
\dots x_n\) uno stream di \(n\) caratteri scelti da un insieme di cardinalit\`a
\(m\le n\) di simboli distinti. Denotiamo con \(D(\chi)\) il numero di elementi
distinti effettivamente presenti nello stream.

Per prima cosa risolviamo un problema pi\`u semplice: dato un numero \(t\)
vogliamo decidere con alta probabili\`a se \(D(\chi) \gg t\) oppure \(D(\chi) \ll
t\). Pi\`u precisamente vogliamo risolvere il seguente
\begin{problem*}[Semplificato]
  Data una soglia \(t\) e uno stream \(\chi\), rispondere
  \begin{itemize}
    \item ``S\`i'' se \(D(\chi)\ge t\),
    \item ``No'' se \(D(\chi)<\frac{t}{2}\),
    \item indifferentemente ``S\`i'' o ``No'' altrimenti,
  \end{itemize}
  con probabilit\`a maggiore di \(1 - \delta\) di essere corretto.
\end{problem*}

Supponiamo di avere una funzione hash \(h:\chi\rightarrow [1,t]\) ideale, tale
cio\`e che \(\forall i\mbox{, }\Pr[h(x)=i]=\frac{1}{t}\). In termini di
questa funzione siamo quasi in grado di risolvere il precedente problema.
Abbiamo infatti il seguente
\begin{algorithm}
  \caption{Contatore con rumore}
  \begin{algorithmic}[1]
    \Function{NoisyCounter}{h}
    \For {each \(x_i\in X\)}
      \If {\(h(x_i) = t\)}
        \State \Return {``S\`i''}
      \EndIf
    \EndFor
    \State \Return {``No''}
    \EndFunction
  \end{algorithmic}
\end{algorithm}

\begin{lemma}
  Supponiamo che \(D(\chi)\ge t\) o \(D(\chi)<\frac{t}{2}\). Allora il precedente
  algoritmo \`e corretto con probabilit\`a maggiore di \(0.6\).
\end{lemma}
\begin{proof}
  Osserviamo che basta stimare la probabilit\`a di rispondere ``No''. Infatti
  questa sar\`a la probabilit\`a d'errore qualora \(D(\chi)\) fosse maggiore di
  \(t\) e la probabilit\`a di successo nell'altro caso.
  \begin{itemize}
    \item Supponiamo \(D(\chi)\ge t\). Allora la probabilit\`a di fallimento \`e
    la probabilit\`a che \(D(\chi)\) volte la funzione hash sia diversa da \(t\),
    il che accade con probabilit\`a \(\Pr[h(x_i)\neq t]=1-\frac{1}{t}\).
    Dunque possiamo scrivere
    \begin{align*}
      \Pr[\mbox{fallimento}]&=\overbrace{\left(1-\frac{1}{t}\right)\left(1-
      \frac{1}{t}\right)\cdots\left(1-\frac{1}{t}\right)}^{D(\chi)\mbox{ 
      volte}} \\ &= \left(1-\frac{1}{t}\right)^{D(\chi)} \le \left(1-
      \frac{1}{t}\right)^{t} < \frac{1}{e} \approx 0.37\text{,}
    \end{align*}
    di conseguenza
    \[\Pr[\mbox{successo}]=1-\Pr[\mbox{fallimento}]\approx 0.63\text{.}\]
    \item Supponiamo invece \(D(\chi)<\frac{t}{2}\). La probabilit\`a di successo
    \`e ancora la probabilit\`a che \(D(\chi)\) volte la funzione hash sia diversa
    da \(t\), dunque
    \begin{align*}
      \Pr[\mbox{successo}]&=\overbrace{\left(1-\frac{1}{t}\right)\left(1-\frac{1}{t}
      \right)\cdots\left(1-\frac{1}{t}\right)}^{D(\chi)\mbox{ volte}} \\ &=
      \left(1-\frac{1}{t}\right)^{D(\chi)}\ge \left(1-\frac{1}{t}\right)^
      {\frac{t}{2}} > \frac{1}{\sqrt{e}}\approx 0.60\text{.}
    \end{align*}
  \end{itemize}
  Otteniamo quindi un algoritmo che fornisce la risposta corretta in pi\`u del
  \(60\%\) dei casi. 
\end{proof}

Possiamo superare qualunque soglia di confidenza semplicemente ripetendo
abbastanza volte il precedente algoritmo e scegliendo il risultato ottenuto pi\`u
frequentemente. Sia dunque \(H = \left\{h_j\,|\,j\in[k]\right\}\)  una famiglia
di funzioni hash ideali e indipendenti.
\begin{algorithm}
  \caption{Referendum di \textsc{NoisyCounter}}
  \begin{algorithmic}[1]
    \State {yes\_votes = 0}
    \For {each \(h_j\in H\)}
      \If {\(\textsc{NoisyCounter}(h_j)\) = ``S\`i''}
        \State {yes\_votes\,++}
      \EndIf
    \EndFor
    \If {yes\_votes \(>\frac{k}{2}\)}
      \State \Return {``S\`i''}
    \EndIf
    \State \Return {``No''}
  \end{algorithmic}
\end{algorithm}

\begin{lemma}
  \(\forall\,\delta>0\), il precedente algoritmo risolve il problema semplificato
  con probabilit\`a di essere corretto \(1-\delta\).
\end{lemma}
\begin{proof}
  Sia \(Z_i\) l'indicatrice dell'evento \(\{\textsc{NoisyCounter}(h_j)=\mbox{``S\`i''}\}\).
  Il precedente lemma garantisce che \(\mbox{Pr}[Z_i]\ge 0.6\). Sia allora \(Z=
  \sum_{i=1}^k{Z_i}\). Osserviamo che dalla linearit\`a della speranza segue 
  immediatamente che \(\E[Z]\ge 0.6k\). Di conseguenza abbiamo:
  \[\mbox{Pr}\left[Z\le\frac{k}{2}\right]\le\mbox{Pr}\left[Z\le\frac{0.5}{0.6}
  \E[Z]\right]\mbox{.}\]
  Possiamo maggiorare il membro di destra con una delle disuguaglianze di
  Chernoff, cio\`e:
  \[\mbox{Pr}[Z\le(1-\xi)\E[Z]]\le\exp{\left(-\frac{\E[Z]\cdot\xi^2}{2}\right)}\mbox{.}\]
  Abbiamo dunque, ponendo \(1-\xi = \frac{0.5}{0.6}\), che:
  \[\mbox{Pr}\left[Z\le\frac{k}{2}\right]\le\exp{\left(-\frac{0.6 k\cdot{\frac{0.1}{0.6}}^2}{2}\right)}\mbox{,}\]
  perci\`o possiamo rendere arbitrariamente  piccola la probabilit\`a di rispondere
  ``No''. Di conseguenza, ragionando come nella dimostrazione del precedente lemma,
  possiamo ottenere un algoritmo corretto con probabilit\`a \(1-\delta\) scegliendo
  \(k = O\left(\log{\frac{1}{\delta}}\right)\).
\end{proof}


    \chapter {Cuckoo hashing}

\begin{problem*}
    Scrivere tutti i passaggi dell'analisi del costo dell'inserimento
    di un elemento in una tabella di cuckoo hashing. Discutere anche
    della cancellazione e della sua complessità.
\end{problem*}

Dobbiamo fare hashing da un insieme di $n$ elementi ad un insieme di $r$
elementi. Invece di procedere nel modo classico usando le liste di adiacenza
(che nel caso pessimo potrebbero contenere tutti gli elementi), vogliamo un
modo che ci permetta di avere esattamente un elemento in associato ad ognuno
degli $r$ valori (quindi che sia possibile salvare gli $n$ elementi su un array
di dimensione $r$).
Per farlo, ci avvaliamo di due funzioni hash, 2-wise indipendenti, $h_1(x)$ e
$h_2(x)$, in questo modo:

\begin{algorithm}
    \caption{Inserimento in Cuckoo hashing}
    \begin{algorithmic}[1]
        \State provo ad inserire x in $h_1(x)$
        \State se la cella è libera lo inserisco semplicemente
        \State se la cella non è libera tolgo l'elemento y dalla cella $h_1(x)$
        e ripeto il procedimento per y, provando a inserirlo in $h_1(y)$ se
        $h_1(x) = h_2(y)$ o in $h_2(y)$ se $h_1(x) = h_1(y)$, per un massimo di
        $n$ volte.
        \State se il valore non è stato ancora inserito si cambiano le funzioni
        di hashing, e si riprova ad inserire il valore che si stava cercando di
        inserire alla $n$-esima iterazione.
    \end{algorithmic}
\end{algorithm}

Si nota, come vedremo, che il numero di iterazioni massimo nel punto 3 serve
per evitare di andare in loop quando si tenta di inserire un valore.

Per fare l'analisi è necessario introdurre i concetti di grafo Cuckoo e di
bucket, più un lemma:

\begin{definition*}[Grafo Cuckoo]
    Il grafo Cuckoo è un grafo che ha per nodi le celle dell'array, con un arco
    uscente in ogni cella contenente un valore, che punta alla cella alternativa
    secondo le funzioni $h_1(x)$, $h_2(x)$. Ovvero, se il nodo $i$ contiene il
    valore $x$, $i$ avrà un arco uscente verso $h_2(x)$ se $i = h_1(x)$ o
    viceversa un arco verso $h_1(x)$ se $i = h_2(x)$.
\end{definition*}

\begin{definition*}[Bucket]
    Si dice \emph{bucket} di un valore $x$ l'insieme dei nodi raggiungibili dai
    nodi $\left\{h_1(x), h_2(x)\right\}$ nel grafo Cuckoo. Ossia tutti i nodi
    con cui potremmo avere a che fare nel caso volessimo inserire $x$.
\end{definition*}

\begin{lemma}
    Per ogni nodo $i$ e $j$, e ogni $c > 1$, se $r \ge 2cn$, la probabilità che
    esista un cammino tra $i$ e $j$ di lunghezza $l$ è al più $\frac{c^{-l}}{r}
    = \frac{1}{c^lr}$. Ovvero, se il numero di celle nell'array è
    sufficientemente più grande del numero di valori salvati, la probabilità
    che esista un cammino di lunghezza $l$ tra due nodi è $O(\frac{1}{r})$, e
    decresce esponenzialmente.
\end{lemma}

\begin{proof*}
    Procediamo per induzione sulla lunghezza del percorso:
    \begin{itemize}
    \item per $l = 1$: un percorso di lunghezza 1 tra due nodi $i$ e $j$ esiste
    sse per qualche $x$ $h_1(x) = i \land h_2(x) = j$ oppure $h_1(x) = j \land
    h_2(x) = i$, si ha:
    
    \begin{align}
        &\Pr\left[(h_1(x) = i \land h_2(x) = j) \lor (h_1(x) = i \land h_2(x) = j) \right] = \nonumber \\
        &= \Pr\left[h_1(x) = i \land h_2(x) = j\right] + \Pr\left[h_1(x) = j \land h_2(x) = i\right] = \nonumber \\
        &= 2 \Pr\left[h_1(x) = i\right]\Pr\left[h_2(x) = j\right] = \nonumber \\
        &= 2 \frac{1}{r} \frac{1}{r} = \frac{2}{r^2} \nonumber
    \end{align}

    Siccome il numero di elementi per cui vale la proprietà vista sopra è al
    più n, si ha:

    \begin{align}
        & \Pr\left[\exists \mbox{percorso di lunghezza 1 tra $i$ e $j$}\right] \nonumber \\
        &\le n \frac{2}{r^2} = \frac{2n}{r} \frac{1}{r} \nonumber \\
        &\left\{r \ge 2cn \mbox{ per ipotesi} \Rightarrow c \le \frac{r}{2n}
            \Rightarrow \frac{1}{c} \ge \frac{2n}{r} \right\} \nonumber \\
        &\le \frac {1}{cr} = \frac{c^{-1}}{r} \nonumber
    \end{align}

    \item per $l > 1$ è necessario che:
        \begin{enumerate}
            \item Esista un percorso ottimo lungo $l-1$ da $i$ a $k$.
            \item Esista un arco tra $k$ e $j$.
        \end{enumerate}
        Abbiamo
            \[\Pr[(1)] = \frac{c^{1-l}}{r}\]
        per ipotesi induttiva. Inoltre, usando lo stesso ragionamento di prima
        otteniamo che
            \[\Pr[(2)]= \frac{c^{-1}}{r}.\]

        Notiamo che i valori possibili di k sono $r$ e che quindi la probabilità
        totale è data da:
        \[r \Pr[(1)]\Pr[(2)] = r\frac{c^{-l}}{r^2} = \frac{c^{-l}}{r}\] \qed
    \end{itemize}
\end{proof*}

La probabilità che al punto 3 dell'inserimento avvenga un reashing, è maggiorata
dalla probabilità che per qualche elemento esista un ciclo. Se notiamo che
\emph{esiste un ciclo di lunghezza $l$}$\Leftrightarrow$\emph{esiste un percorso di
lunghezza $l$ tra $i$ e $i$ } otteniamo:
\begin{align*}
    &\Pr\left[\exists \mbox{un ciclo per il nodo i nel grafo Cuckoo}\right] \\
    &= \sum_{l=1}^{\infty} \Pr\left[\exists \mbox{ciclo di lunghezza $l$ nel grafo Cuckoo}\right] \\
    &= \sum_{l=1}^{\infty} \Pr\left[\exists \mbox{percorso di lunghezza $l$ tra $i$ e $i$}\right] \\
    &\le \sum_{l=1}^{\infty} \frac{c^{-l}}{r} = \frac{1}{r(c-1)}
\end{align*}

da cui:
\begin{align*}
    &\Pr\left[\exists \mbox{un ciclo nel grafo cuckoo}\right] \\ 
    &= \sum_{i=1}^{r} \Pr\left[\exists \mbox{un ciclo per il nodo i nel grafo cuckoo}\right] \\
    &= r*\frac{1}{r(c-1)} = \frac{1}{c-1}
\end{align*}

notiamo inoltre che se abbiamo esattamente n cicli, possiamo avere al più
n rehash, quindi:

\begin{align*}
    &\Pr\left[\exists \mbox{un ciclo nel grafo cuckoo}\right]^n
        \Pr\left[\not\exists \mbox{nessun ciclo nel grafo cuckoo}\right] \\
    &= \frac{1}{(c-1)^n} \frac{c-2}{c-1} = \frac{c-2}{(c-1)^{n+1}}\\
    &\ge \Pr\left[\mbox{n rehash}\right]
\end{align*}

Ponendo $c=3$ la probabilità di un rehash è al più $\frac{1}{4}$, e di n
rehash è al più $\frac{1}{2^{n+1}}$, quindi il numero atteso di rehash ad ogni
inserimento è

\[ \sum_{i=1}^{n} i \frac{1}{2^{i+1}} = 1. \]

Il costo medio di un inserimento quindi è dato dal costo di un rehashing, 
ognuno da $\Theta(n)$, quindi a sua volta $\Theta(n)$, mentre il costo
ammortizzato per ogni inserimento è $O(1)$.

La cancellazione avviene in $O(1)$, cercando il valore da cancellare nelle sole
due celle possibili ed eliminandolo, è possibile osservare infatti che la
struttura che si ottiene è ancora un cuckoo hashing dove la funzione $h_1(x)$
è quella che mette tutti gli elementi esattamente dove sono e la funzione
$h_2(x)$ è una qualunque (o varianti equivalenti).
    \chapter{Random search tree}

\begin{problem*}
  Scrivere l'algoritmo per inserire una chiave in un random search tree
  con una sola discesa dalla radice (i.e., senza dover risalire poi dalla
  foglia appena inserita verso la radice mediante le rotazioni).
\end{problem*}


    \chapter{Lista invertita compressa}

\begin{problem*}
  Prendiamo una sequenza ordinata crescente di \(n\) interi
  \(i_1, i_2,\dots ,i_n\), come per esempio una lista invertita.
  La rappresentazione compressa differenziale \`e la sequenza \(S\) di
  \(|S|\) bit ottenuti concatenando
  \(\gamma(i_1),\gamma(i_2 - i_1),\dots ,\gamma(i_{n} - i_{n-1})\), dove
  \(\gamma(x)\) rappresenta il gamma code di Elias per la codifica
  dell'intero \(x\ge 1\) in \(2\lfloor\log_{2}{x}\rfloor + 1\) bit.
  Mostrare come aggiungere un'opportuna directory di spazio \(O(|S|)\) bit
  (meglio ancora, di \(o(|S|)\) bit) per poter accedere velocemente, dato
  \(j\in [2\dots n]\), alla codifica \(\gamma(i_{j} - i_{j-1})\).
  Estendere tale approccio per accedere velocemente a \(i_j\) (e quindi
  poter eseguire una ricerca binaria sugli interi della lista invertita
  compressa).
\end{problem*}


    \chapter{Prefix tree del codice di Huffman}

\begin{problem*}
  Impostare un algoritmo per costruire il prefix tree del codice di
  Huffman. Dimostrare l'ottimalit\`a di tale albero in termini di numero
  di bit utilizzati per codici prefix free dei simboli.
\end{problem*}

Dato un testo, costruiamo il codice di Huffman come segue

\begin{algorithm}
    \caption{Algoritmo per la costruzione del prefix tree del codice di Huffman}
    \begin{algorithmic}[1]
        \State scorro la stringa mantenendo tante triple della forma $<p, t>$ per ogni carattere $c$, dove $c$ rappresenta il carattere, $p$ la probabilità empirica per $c$ e $t$ è l'albero associato al carattere (in questo caso una foglia contenente come valore $c$).
        \While {ci sono triple}
            \State prendo le due triple $<p1, t1>$, $<p2, t2>$ con minor valore di $p$
            \State creo la nuova tripla $<p', t'>$
            \State $p' \gets p1 + p2$
            \State $t'.left \gets t1$
            \State $t'.right \gets t2$
        \EndWhile
    \end{algorithmic}
\end{algorithm}

Per ottenere i codici associati ai caratteri è sufficiente visitare l'albero, associando ad una ricorsione sul figlio sinistro l'inserimento nel codice di un bit 0 e 1 nel caso del figlio destro, o viceversa.

Si nota che le righe 6 e 7 potevano tranquillamente avere i valori invertiti, in quanto non cambierebbe né la lunghezza dei codici ottenuti, né la proprietà di prefix free, in quanto i valori codificati restano sempre sulle foglie.

\begin{lemma}[Ottimalità]
Sia $H$ l'albero ottenuto con l'algoritmo precedente, e sia $T$ un qualunque altro albero con la stessa struttura ma con le foglie permutate. Vale
\[
    P(H) \le P(T), \mbox{con } P(X) = \sum_{c \in \Sigma} l_{X, c} \, p_c
\]
\end{lemma}

\begin{proof*}
È sufficiente mostrare che scambiando due foglie in $H$ otteniamo un valore più grande di $P(H)$.

Siano $l_a, p_a, l_b, p_b$ lunghezza e probabilità associate a due caratteri $a$ e $b$, vogliamo
dimostrare $l_a p_a + l_b p_b \le l_a p_b + l_b p_a$.

\begin{align*}
    & l_a p_a + l_b p_b \le l_a p_b + l_b p_a \\
    & (l_a - l_b) p_a + (l_b - l_a) p_b \le 0 \\
    & (l_a - l_b) p_a - (l_a - l_b) p_b \le 0 \\
    & (l_a - l_b)(p_a - p_b) \le 0
\end{align*}

che è sempre vero perché $p_a \le p_b \Rightarrow l_b \le l_a$. \qed
\end{proof*}
    \chapter{Applicazioni di LZ77}

\begin{problem*}
  Sfruttando le caratteristiche dell'algoritmo LZ77 di Lempel e Ziv per
  suddividere un testo in una sequenza di frasi, mostrare come utilizzare
  LZ77 per 
  \begin{enumerate}[(a)]
    \item nascondere dei bit all'interno del file compresso risultante,
    \item trovare la pi\`u lunga sottostringa che si ripete, ossia che appare almeno due volte nel testo.
  \end{enumerate}
\end{problem*}


    \chapter{Dizionario di LZ78}

\begin{problem*}
  Progettare una struttura dati per memorizzare e interrogare velocemente
  il dizionario delle frasi ottenute incrementalmente con l'algoritmo
  LZ78. Valutare il costo delle soluzioni proposte.
\end{problem*}



\end{document}
