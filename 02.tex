\chapter{Limite inferiore per la permutazione}

\begin{problem*}
    Estendere l'argomentazione usata per il limite inferiore del problema 
    dell'ordinamento in memoria esterna a quello della permutazione: dati
    \(N\) elementi \(e_1,e_2,\dots ,e_N\) e un array \(\pi\) contenente
    una permutazione degli interi in \([1,2,\dots ,N]\), disporre gli
    elementi secondo la permutazione in \(\pi\). Dopo tale operazione,
    la memoria esterna deve contenerli nell'ordine
    \(e_{\pi[1]},e_{\pi[2]},\dots ,e_{\pi[N]}\).
\end{problem*}

Mostreremo che il problema presenta un lower-bound di $\displaystyle\Omega\left(\min\left\{N,~\frac{N}{B}\log_{\frac{M}{B}}\frac{N}{B}\right\}\right)$.
L'idea è la seguente: consideriamo una sequenza di $t$ operazioni di input, e contiamo quante permutazioni diverse possiamo ottenere. Trovando il minimo $t$ che ci garantisca la possibilità di ottenere \textit{tutte} le permutazioni distinte avremo allora il lower-bound desiderato.\[\]
Consideriamo una generica operazione di input: possiamo scegliere tra $N$ pagine in memoria (in realtà, se gli elementi sono disposti in modo contiguo, la scelta si restringe a $\frac{N}{B}\le N$, ma è corretto usare $N$ perché stiamo cercando una sottostima del numero di I/O necessari, e dunque una sovrastima del numero di permutazioni ottenibili data una sequenza di I/O di lunghezza fissata). Distinguiamo ora due casi:
\begin{itemize}
\item Se la pagina che ho scelto viene caricata in memoria per la prima volta (e questo avviene $N \choose B$ volte), posso permutare i suoi elementi in $B!$ modi, e disporli tra gli elementi già in memoria in al più $M\choose{B}$ modi.
\item Se la pagina era già stata caricata in memoria in precedenza, i suoi elementi sono già permutati nell'ordine desirderato, dunque devo solo scegliere gli $M\choose B$ modi in cui disporli rispetto agli elementi già in memoria.
\end{itemize}
Unendo i due casi, dopo una sequenza di $t$ I/O posso produrre al più \[N^t\cdot(B!)^\frac{N}{B}\cdot {M\choose B}^t\] permutazioni distinte; ovviamente vogliamo che questo numero risulti maggiore di $N!$. Passando ai logaritmi, dev'essere \[t\log N + \frac{N}{B}\log B! + t\log {M\choose B} \ge \log N!,\]da cui, usando le approssimazioni $\log x! \sim x\log x$ e $\log{n\choose k} \sim k\log\frac{n}{k}$, \[t\log N + N\log B + tB\log\frac{M}{B} \ge N\log N\]\[t\left(\log N + B\log\frac{M}{B}\right)\ge N(\log N-\log B)\]\[t\ge\frac{N\log\frac{N}{B}}{\log N + B\log\frac{M}{B}}\]
Distinguiamo ancora due casi:
\begin{itemize}
\item Se $\log N \le B \log\frac{M}{B}$ dev'essere \[\displaystyle t \ge \frac{N \log \frac{N}{B}}{2B\log\frac{M}{B}}=\Omega\left(\frac{N}{B}\log_{\frac{M}{B}}\frac{N}{B}\right).\]
\item Se invece $\log N \ge B\log\frac{M}{B}$ allora \[t\ge\frac{N\log\frac{N}{B}}{2\log N}\ge\frac{N\log N - N\log B}{2\log N} =\]\[= \frac12\left(N-N\frac{\log B}{\log N}\right)\ge\frac12(N-\frac12 N)=\Omega(N),\]il che ovviamente vale purché $\displaystyle\frac{\log B}{\log N}\le\frac12$, ossia $B\le \sqrt{N}$, relazione vera in tutti i casi di interesse.
\end{itemize}
<<<<<<< HEAD
Dall'unione dei due casi segue il lower-bound desiderato.  
=======
Dall'unione dei due casi segue il lower-bound desiderato. 
>>>>>>> 222ad24863355b7b6b476dfbe7ceaa4d2faac947
