\chapter{Approssimazione per MAX-CUT}

\begin{problem*}
    Il problema MAX-CUT è NP-hard ed è definito come segue per 
    un grafo non orientato $G = (V, E)$. Una partizione di nodi $(C, V - C)$
    con $C \in V$ si chiama “cut” o taglio. Un arco $e = (v, w)$ con $v \in C$
    e $w \in V - C$ si chiama arco di taglio (ricordando che $(v, w)$ e $(w, v)$
    denotano lo stesso arco in un grafo non orientato). Il numero di archi di 
    taglio definisce la dimensione del cut $(C, V - C)$. Poiché cambiando taglio, 
    può cambiare la sua dimensione, il problema richiede di trovare il taglio di 
    dimensione massima e quindi gli archi di taglio corrispondenti. Dimostrare che 
    il seguente algoritmo randomizzato è una 2-approssimazione in valore atteso, 
    ossia che il numero medio di archi di taglio così individuati è in media almeno
    la metà di quelli del taglio massimo. (1) Per ogni nodo $v \in V$ , lancia una 
    moneta equiprobabile: se viene testa, inserisci $v$ in $C$; altrimenti (viene
    croce), inserisci $v$ in $V - C$. (2) Inizializza $T$ all’insieme vuoto. Per 
    ogni arco $(v, w) \in E$, tale che $v \in C$ e $w \in V - C$, aggiungi $(v, w)$
    all’insieme $T$ . Restituisci $C$ e $T$ come soluzione approssimata.
\end{problem*}
