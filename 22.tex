\chapter{Approssimazione per MAX-CUT}

\begin{problem*}
    Il problema MAX-CUT è NP-hard ed è definito come segue per 
    un grafo non orientato $G = (V, E)$. Una partizione di nodi $(C, V - C)$
    con $C \in V$ si chiama “cut” o taglio. Un arco $e = (v, w)$ con $v \in C$
    e $w \in V - C$ si chiama arco di taglio (ricordando che $(v, w)$ e $(w, v)$
    denotano lo stesso arco in un grafo non orientato). Il numero di archi di 
    taglio definisce la dimensione del cut $(C, V - C)$. Poiché cambiando taglio, 
    può cambiare la sua dimensione, il problema richiede di trovare il taglio di 
    dimensione massima e quindi gli archi di taglio corrispondenti. Dimostrare che 
    il seguente algoritmo randomizzato è una 2-approssimazione in valore atteso, 
    ossia che il numero medio di archi di taglio così individuati è in media almeno
    la metà di quelli del taglio massimo. (1) Per ogni nodo $v \in V$ , lancia una 
    moneta equiprobabile: se viene testa, inserisci $v$ in $C$; altrimenti (viene
    croce), inserisci $v$ in $V - C$. (2) Inizializza $T$ all’insieme vuoto. Per 
    ogni arco $(v, w) \in E$, tale che $v \in C$ e $w \in V - C$, aggiungi $(v, w)$
    all’insieme $T$ . Restituisci $C$ e $T$ come soluzione approssimata.
\end{problem*} 
Definiamo la variabile indicatrice $X_{(u,v)}$ che vale $1$ se l'arco $(u,v)$ appartiene a $T$, $0$ altrimenti. Allora \[\Pr[X_{(u,v)}=1]=\Pr[u \in C \wedge v\in V-C] + \Pr[u\in V-C \wedge v\in C] = \frac12,\] da cui $\mathbb{E}[X_{(u,v)}]=\frac12$.\newline
Definiamo poi $\displaystyle X = \sum_{(u, v)\in E}X_{(u, v)}$. Allora \[\mathbb{E}[X]=\sum_{(u, v)\in E}\mathbb{E}[X_{(u, v)}]=\frac12|E|.\]
Detto $T^*$ il taglio ottimo, si ha ovviamente che $|T^*|\le|E|$, da cui \[\mathbb{E}[|T|]=\mathbb{E}[X]=\frac12|E|\ge\frac12|T^*|,\]come volevasi dimostrare. 
