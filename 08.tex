\chapter{Suffix array in memoria esterna}

\begin{problem*}
  Utilizzando la costruzione del suffix array basata sul \mergesort e la tecnica
  \dc vista a lezione, progettare un algoritmo per EMM per costruire il Suffix
  Array di un testo che abbia la stessa complessit\`a del \mergesort in EMM.
\end{problem*}

Diamo di seguito un esempio di applicazione dell'algoritmo \dc alla stringa
\texttt{\string"abracadabra\string"}. Per ragioni tecniche aggiungiamo tre 
caratteri speciali \texttt{\$} in fondo alla stringa, ottenendo 
\texttt{\string"abracadabra\$\$\$\string"}. Scandiamo la stringa e sostituiamo 
ogni carattere con il proprio rango nell'ordine alfabetico dei caratteri della 
stringa stessa, con la convenzione che il carattere \texttt{\$} precede ogni altro 
carattere. Otteniamo dunque:
\begin{table}[h]
  \begin{tabularx}{\linewidth}{l*{14}{X}}
    Carattere              & \texttt{a} & \texttt{b} & \texttt{r} & \texttt{a}
                           & \texttt{c} & \texttt{a} & \texttt{d} & \texttt{a}
                           & \texttt{b} & \texttt{r} & \texttt{a} & \texttt{\$}
                           & \texttt{\$} & \texttt{\$} \\
    \hline
    Rango del carattere    & 1 & 2 & 5 & 1 & 3 & 1 & 4 & 1 & 2 & 5 & 1 & 0 & 0 & 0 \\
    Indice del suffisso    & 0 & 1 & 2 & 3 & 4 & 5 & 6 & 7 & 8 & 9 & 10 & 11 & 12 & 13 \\
  \end{tabularx}
\end{table}

Dividiamo i caratteri in \(3\) gruppi secondo la congruenza modulo \(3\) della 
posizione nella stringa. Assegnamo rispettivamente i colori verde, rosso e giallo 
alle \(3\) classi di congruenza:
\begin{table}[h]
  \begin{tabularx}{\linewidth}{l*{14}{X}}
    Carattere              & \texttt{a} & \texttt{b} & \texttt{r} & \texttt{a}
                           & \texttt{c} & \texttt{a} & \texttt{d} & \texttt{a}
                           & \texttt{b} & \texttt{r} & \texttt{a} & \texttt{\$}
                           & \texttt{\$} & \texttt{\$} \\
    \hline
    Rango del carattere    & 1 & 2 & 5 & 1 & 3 & 1 & 4 & 1 & 2 & 5 & 1 & 0 & 0 & 0 \\
    Indice del suffisso    & 0 \cellcolor{green} & 1 \cellcolor{red} & 2 \cellcolor{yellow} 
                           & 3 \cellcolor{green} & 4 \cellcolor{red} & 5 \cellcolor{yellow} 
                           & 6 \cellcolor{green} & 7 \cellcolor{red} & 8 \cellcolor{yellow} 
                           & 9 \cellcolor{green} & 10 \cellcolor{red} & 11 \cellcolor{yellow} 
                           & 12 \cellcolor{green} & 13 \cellcolor{red} \\
  \end{tabularx}
\end{table}

Trascuriamo per il momento il gruppo verde e sostituiamo ogni carattere di indice 
\(i\) dei gruppi giallo e rosso con la tripla dei caratteri di indice \(i, i+1\) e 
\(i+2\). Riordiniamo inoltre la tabella secondo la divisione in colori,
tralasciando le triple che cominciano per il simbolo \texttt{\$}:
\begin{table}[h]
  \begin{tabularx}{\linewidth}{l*{11}{X}}
                        & \multicolumn{4}{c}{Gruppo 0 \cellcolor{green} } 
                        & \multicolumn{4}{c}{Gruppo 1 \cellcolor{red} } 
                        & \multicolumn{3}{c}{Gruppo 2 \cellcolor{yellow} }\\
    \hline
    Triple di caratteri & \multicolumn{4}{c}{\cellcolor{gray!25}}
                        & \texttt{bra} & \texttt{cad} & \texttt{abr} & \texttt{a\$\$}
                        & \texttt{rac} & \texttt{ada} & \texttt{bra} \\
    Triple di ranghi    & \multicolumn{4}{c}{\cellcolor{gray!25}}
                        & 251 & 314 & 125 & 100
                        & 513 & 141 & 251 \\
    Indice del suffisso & \multicolumn{4}{c}{\cellcolor{gray!25}}
                        & 1 & 4 & 7 & 10
                        & 2 & 5 & 8 \\
  \end{tabularx}
\end{table}

\newpage

Ordiniamo ora le colonne dei gruppi \(1\) e \(2\) secondo le triple dei ranghi con
\(3\) passaggi di \radixsort. Da questo ordinamento ricaviamo i ranghi delle 
triple, cio\`e la loro posizione nell'ordinamento lessicografico delle
corrispondenti triple di caratteri:
\begin{table}[h]
  \begin{tabularx}{\linewidth}{l*{11}{X}}
                        & \multicolumn{4}{c}{Gruppo 0 \cellcolor{green} } 
                        & \multicolumn{7}{c}{Gruppi 1 \& 2\cellcolor{orange} } \\
    \hline
    Triple di caratteri & \multicolumn{4}{c}{\cellcolor{gray!25}}
                        & \texttt{a\$\$} & \texttt{abr} & \texttt{ada} & \texttt{bra}
                        & \texttt{bra} & \texttt{cad} & \texttt{rac} \\
    Triple di ranghi    & \multicolumn{4}{c}{\cellcolor{gray!25}}
                        & 100 & 125 & 141 & 251
                        & 251 & 314 & 513 \\
    Indice del suffisso & \multicolumn{4}{c}{\cellcolor{gray!25}}
                        & 10 & 7 & 5 & 1
                        & 8 & 4 & 2 \\
    Ranghi delle triple & \multicolumn{4}{c}{\cellcolor{gray!25}}
                        & \textbf{1} & \textbf{2} & \textbf{3} & \textbf{4}
                        & \textbf{4} & \textbf{5} & \textbf{6} \\
  \end{tabularx}
\end{table}

Esistono due triple con lo stesso rango. Dobbiamo quindi ripetere ricorsivamente
l'algoritmo \dc sulle triple di caratteri dei gruppi \(1\) e \(2\). Omettiamo
questo passaggio per brevit\`a e scriviamo direttamente l'indice nel Suffix Array
della corrispondente tripla di caratteri:
\begin{table}[h]
  \begin{tabularx}{\linewidth}{l*{11}{X}}
                        & \multicolumn{4}{c}{Gruppo 0 \cellcolor{green} } 
                        & \multicolumn{4}{c}{Gruppo 1 \cellcolor{red} } 
                        & \multicolumn{3}{c}{Gruppo 2 \cellcolor{yellow} }\\
    \hline
    Triple di caratteri & \multicolumn{4}{c}{\cellcolor{gray!25}}
                        & \texttt{bra} & \texttt{cad} & \texttt{abr} & \texttt{a\$\$}
                        & \texttt{rac} & \texttt{ada} & \texttt{bra} \\
    Triple di ranghi    & \multicolumn{4}{c}{\cellcolor{gray!25}}
                        & 251 & 314 & 125 & 100
                        & 513 & 141 & 251 \\
    Indice del suffisso & \multicolumn{4}{c}{\cellcolor{gray!25}}
                        & 1 & 4 & 7 & 10
                        & 2 & 5 & 8 \\
    Indice nel Suffix Array & \multicolumn{4}{c}{\cellcolor{gray!25}}
                        & \textbf{4} & \textbf{5} & \textbf{1} & \textbf{0}
                        & \textbf{6} & \textbf{2} & \textbf{3} \\
  \end{tabularx}
\end{table}

