\chapter{Navigazione implicita in vEB}

\begin{problem*}
    Dato un albero completo memorizzato secondo il layout di van Emde Boas (vEB) in
    modo implicito, ossia senza l'ausilio di puntatori (come succede nello heap binario
    implicito), trovare la regola per navigare in tale albero senza usare puntatori
    espliciti.
\end{problem*}
Per consentire la navigazione nel layout implicito è necessaro costruire, durante la fase di archiviazione dell'albero, una tabella di dimensione $O(\log n)$, in cui per ogni profondità $d$, memorizziamo:
\begin{itemize}
\item $\mbox{B}[d]$: la dimensione di ogni bottom tree avente radice a profondità $d$;
\item $\mbox{T}[d]$: la dimensione del corrispondente top tree;
\item $\mbox{D}[d]$: la profondità della radice di tale top tree.
\end{itemize}
Inoltre quando effettuiamo una ricerca (a partire dalla radice) di un nodo $v$ di profondità $d$, teniamo traccia di:
\begin{itemize}
\item $i$: la posizione che $v$ occuperebbe in una memorizzazione BFS;
\item $\mbox{Pos}[j]$: la posizione nella memorizzazione vEB del particolare nodo $z$ a profondità $j$ che ho incontrato durante la ricerca.
\end{itemize}
Consideriamo la rappresentazione binaria di $i$: letti da sinistra verso destra, i suoi bit rappresentano le svolte effettuate nella ricerca di $v$ nella rappresentazione BFS (secondo la regola: $0$ svolta a sinistra nella BFS, $1$ svolta a destra; nota: il bit più a sinistra ovviamente vale sempre $1$, e indica il fatto che si parte con la radice).\newline
Consideriamo una suddivisione in cui il nodo $v$ cercato è radice di un bottom tree. Allora se il corrispondente top tree è alto $k$, cioè $\mbox{T}[d]=2^k-1$, i $k$ bit più a destra di $i$ rappresentano l'indice del bottom tree cercato tra tutti quelli della suddivisione in questione. Poiché tutti i bit di $\mbox{T}[d]$ (in totale $k$) valgono $1$, posso calcolare questo indice come $i \And \mbox{T}[d]$. \newline
Vale allora \[\mbox{Pos}[d]=\mbox{Pos}(\mbox{D}[d])+\mbox{T}[d]+(i\And \mbox{T}[d])\mbox{B}[d],\] ossia:
\begin{itemize}
\item[-] Parto da $\mbox{Pos}(\mbox{D}[d])$, posizione della radice del top tree; dopodiché avanzo di:
\item[-] Tanti nodi quanti ve ne sono nel top tree, cioé $\mbox{T}[d]$;
\item[-] Tanti nodi quandi ve ne sono in ogni bottom tree, cioé $\mbox{B}[d]$, e faccio questo tante volte quanti sono i bottom tree che precedono quello di radice $v$ ossia, poiché gli indici partono da 0, $i\And \mbox{T}[d]$.
\end{itemize}
Usando ripetutamente questa formula è dunque possibile effettuare la ricerca di un qualsiasi nodo $v$. 
