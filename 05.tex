\chapter{MapReduce}

\begin{problem*}
    Utilizzare il paradigma Scan \& Sort mediante la \mapreduce per calcolare la
    distribuzione dei gradi in ingresso delle pagine Web. In particolare,
    specificare quanti passi di tipo \mapreduce sono necessari e quali sono le
    funzioni \map e \reduce impiegate. Ipotizzare di avere gi\`a tali pagine a
    disposizione.
\end{problem*}

La prima cosa da fare è estrarre tutti i link nelle pagine Web. Definiamo una
funzione \parser che presa una pagina web restituisce tutti i link presenti in
essa. 

\begin{algorithm}
    \caption{Funzione Map - Creazione coppie dei link}
    \begin{algorithmic}[1]
        \Function {Map}{W}
        	\For {each p in W}
        		\State links \(\gets\) \parser(p)
        		\For {each l in links}
        			\State create \(\langle\mbox{l},1\rangle\)
        		\EndFor
        	\EndFor
        \EndFunction
    \end{algorithmic}
\end{algorithm}

La funzione \map prende la lista delle pagine  Web, estrae da ogni pagina tutti
i suoi link e crea una coppia per ognuno di essi con associata la costante
\(1\), la quale serve a indicare che è presente un link con quel
identificativo.

Prima di richiamare la funzione \reduce, per il paradigma \mapreduce, le
coppie sono ordinate e suddivise per il campo chiave in modo tale da
distribuire al meglio il carico di lavoro sulle macchine. 

\begin{algorithm}
    \caption{Funzione Reduce - Creazione coppie numero link}
    \begin{algorithmic}[1]
        \Function {Reduce}{key, list}
      	  \State sum \(\gets 0\)
        	\For {each p in list}
        		\State sum++
        	\EndFor
        	\State create \(\langle\mbox{sum},1\rangle\)
        \EndFunction
    \end{algorithmic}
\end{algorithm}


Poiché le coppie sono state raggruppate per link, quando passiamo un gruppo
alla funzione \reduce questa calcola il numero di link che ha quella pagina in
ingresso. La funzione restituisce una coppia che indica che c'è una pagina che
ha un certo numero di link in ingresso.

Per ottenere la distribuzione dei gradi in ingresso delle pagine Web,
applichiamo una nuova funzione \reduce alle coppie precedentemente create
(prima della funzione le coppie sono ordinate e suddivise in gruppi come al
passo precedente).

\begin{algorithm}
    \caption{Funzione Reduce - Creazione gradi in ingresso}
    \begin{algorithmic}[1]
        \Function {Reduce}{key, list}
      	  \State sum \(\gets 0\)
        	\For {each p in list}
        		\State sum++
        	\EndFor
        	\State create \(\langle\mbox{key,sum}\rangle\)
        \EndFunction
    \end{algorithmic}
\end{algorithm}

In conclusione bisogna eseguire una \map per trovare i link presenti nelle
pagine Web e due funzione \reduce, una per contare il numero di link che ha in
ingresso una pagina e una per calcolare la distribuzione dei gradi in ingresso.
